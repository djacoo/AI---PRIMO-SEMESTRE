\documentclass[11pt,a4paper]{article}

% ========================================
% PACKAGES
% ========================================
\usepackage[utf8]{inputenc}
\usepackage[T1]{fontenc}
\usepackage[italian]{babel} 
\usepackage[margin=2.5cm]{geometry}

% Mathematics
\usepackage{amsmath, amssymb, amsthm}
\usepackage{mathtools}

% Graphics and colors
\usepackage{graphicx}
\usepackage{xcolor}
\usepackage{tikz}

% Lists and formatting
\usepackage{enumitem}
\usepackage{parskip}
\usepackage{fancyhdr}

% Code listings
\usepackage{listings}
\usepackage{algorithm}
\usepackage{algorithmic}

% Hyperlinks
\usepackage{hyperref}
\hypersetup{
    colorlinks=true,
    linkcolor=blue,
    urlcolor=blue,
    citecolor=blue
}

% ========================================
% THEOREM ENVIRONMENTS
% ========================================
\theoremstyle{definition}
\newtheorem{definition}{Definizione}[section]
\newtheorem{example}{Esempio}[section]
\newtheorem{exercise}{Esercizio}[section]

\theoremstyle{plain}
\newtheorem{theorem}{Teorema}[section]
\newtheorem{lemma}[theorem]{Lemma}
\newtheorem{proposition}[theorem]{Proposizione}
\newtheorem{corollary}[theorem]{Corollario}

\theoremstyle{remark}
\newtheorem*{remark}{Nota}
\newtheorem*{observation}{Osservazione}

% ========================================
% CUSTOM COMMANDS
% ========================================
\newcommand{\R}{\mathbb{R}}
\newcommand{\N}{\mathbb{N}}
\newcommand{\Z}{\mathbb{Z}}
\newcommand{\Q}{\mathbb{Q}}
\newcommand{\C}{\mathbb{C}}

% ========================================
% HEADER AND FOOTER
% ========================================
\pagestyle{fancy}
\setlength{\headheight}{14pt}
\fancyhf{}
\lhead{HCI}
\rhead{HCI}
\cfoot{\thepage}

% ========================================
% DOCUMENT INFORMATION
% ========================================
\title{\textbf{HCI - MultiModal Systems - Theory 2}\\
\large Artificial Intelligence}
\author{Jacopo Parretti}
\date{I Semester 2025-2026}

% ========================================
% DOCUMENT
% ========================================
\begin{document}

\maketitle
\newpage
\tableofcontents
\newpage

\part{Foundations of Multimodal Interaction}

\section{Introduction to Intelligent Multimodal Interfaces}

\subsection{Course Objectives}

This course explores the fundamental theories and concepts of \textbf{Human-Computer Interaction (HCI)}, an interdisciplinary field that synthesizes knowledge from cognitive psychology, computer science, and design. The primary objectives are:

\begin{itemize}
    \item Understanding the theoretical foundations of human-computer interaction
    \item Developing practical skills in designing and implementing multimodal interfaces
    \item Exploring the integration of artificial intelligence techniques in interactive systems
    \item Analyzing nonverbal communication and its role in human-computer interaction
\end{itemize}

\subsection{Focus Areas}

The course places special emphasis on three interconnected dimensions:

\subsubsection{Technological Solutions}

Students will develop computer interfaces with a focus on both methodological and implementation aspects. The course emphasizes hands-on experience in building functional interactive systems, bridging the gap between theory and practice.

\subsubsection{Multimodal Interaction}

Special attention is devoted to \textbf{multimodal solutions} that integrate multiple input and output modalities:
\begin{itemize}
    \item \textbf{Touch}: Tactile and haptic interaction
    \item \textbf{Vision}: Camera-based interaction and visual recognition
    \item \textbf{Natural Language}: Speech and text-based communication
    \item \textbf{Audio}: Sound-based interaction and auditory feedback
\end{itemize}

\subsubsection{Intelligent Systems}

The course explores how \textbf{artificial intelligence techniques} can enhance interaction by:
\begin{itemize}
    \item Inferring user intentions from multimodal input
    \item Predicting expected interactions based on context and user behavior
    \item Adapting interface behavior to individual users
    \item Recognizing and responding to affective and social cues
\end{itemize}

\subsection{Course Program}

\subsubsection{Theoretical Component}

The theoretical component covers the following topics:

\begin{enumerate}
    \item \textbf{Introduction}: Course motivation, professional perspectives, open research issues, program overview, and examination methodology
    
    \item \textbf{Foundations of HCI}: Human factors in interface design, interaction design principles, usability evaluation, gaming, and gamification
    
    \item \textbf{Visual Interaction}: Camera calibration techniques, structure from motion, 3D reconstruction
    
    \item \textbf{Nonverbal Behavior in Communication}: 
    \begin{itemize}
        \item Types of nonverbal behavior: facial expressions, gestures, posture, eye gaze
        \item Data collection methods and protocols
        \item Tools and software for nonverbal behavior analysis
        \item Annotation tools (e.g., ELAN)
    \end{itemize}
    
    \item \textbf{Automated Analysis of Body Language}: Movement tracking, gesture recognition, facial expression analysis, and speech processing. Techniques for data capture, feature extraction, and automatic analysis
    
    \item \textbf{Social Artificial Intelligence}: Applications in social psychology, organizational psychology, and social robotics
    
    \item \textbf{Affective Computing}: Theories of emotion, emotion recognition systems, and applications in HCI
    
    \item \textbf{Multimodal Fusion}: Integration of multimodal nonverbal cues using fusion techniques (late fusion, early fusion)
\end{enumerate}

\subsubsection{Laboratory Component}

The laboratory sessions provide hands-on experience with state-of-the-art tools and techniques:

\begin{enumerate}
    \item \textbf{Deep Image Matching}: Python implementation of feature detection and matching algorithms
    
    \item \textbf{3D Model Reconstruction}: Structure from motion using Zephyr software
    
    \item \textbf{Camera Pose Estimation}: C\# implementation of Fiore's method for camera localization
    
    \item \textbf{3D Graphics}: Modeling and rendering in Unity game engine
    
    \item \textbf{Model-Based Augmented Reality}: Implementation of the complete AR pipeline integrating Python code and Unity
    
    \item \textbf{Advanced Topics}: Deep learning approaches to camera pose estimation and model recognition
\end{enumerate}
 














\end{document}