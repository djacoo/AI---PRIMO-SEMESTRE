\documentclass[11pt,a4paper]{article}

% ========================================
% PACKAGES
% ========================================
\usepackage[utf8]{inputenc}
\usepackage[T1]{fontenc}
\usepackage[italian]{babel}  % Change to your language
\usepackage[margin=2.5cm]{geometry}

% Mathematics
\usepackage{amsmath, amssymb, amsthm}
\usepackage{mathtools}

% Graphics and colors
\usepackage{graphicx}
\usepackage{xcolor}
\usepackage{tikz}
\usepackage{pgfplots}
\pgfplotsset{compat=1.18}
\usetikzlibrary{arrows.meta, positioning, shapes}

% Lists and formatting
\usepackage{enumitem}
\usepackage{parskip}
\usepackage{fancyhdr}

% Code listings (if needed)
\usepackage{listings}
\usepackage{algorithm}
\usepackage{algorithmic}

% CJK support for Chinese, Japanese, Korean characters
\usepackage{CJKutf8}

% Hyperlinks
\usepackage{hyperref}
\hypersetup{
    colorlinks=true,
    linkcolor=blue,
    urlcolor=blue,
    citecolor=blue
}

% ========================================
% THEOREM ENVIRONMENTS
% ========================================
\theoremstyle{definition}
\newtheorem{definition}{Definizione}[section]
\newtheorem{example}{Esempio}[section]
\newtheorem{exercise}{Esercizio}[section]

\theoremstyle{plain}
\newtheorem{theorem}{Teorema}[section]
\newtheorem{lemma}[theorem]{Lemma}
\newtheorem{proposition}[theorem]{Proposizione}
\newtheorem{corollary}[theorem]{Corollario}

\theoremstyle{remark}
\newtheorem*{remark}{Nota}
\newtheorem*{observation}{Osservazione}

% ========================================
% CUSTOM COMMANDS
% ========================================
\newcommand{\R}{\mathbb{R}}
\newcommand{\N}{\mathbb{N}}
\newcommand{\Z}{\mathbb{Z}}
\newcommand{\Q}{\mathbb{Q}}
\newcommand{\C}{\mathbb{C}}

% ========================================
% HEADER AND FOOTER
% ========================================
\setlength{\headheight}{14pt}
\pagestyle{fancy}
\fancyhf{}
\lhead{\leftmark}
\rhead{ML\&DL}
\cfoot{\thepage}
\renewcommand{\sectionmark}[1]{\markboth{#1}{}}

% ========================================
% TABLE OF CONTENTS DEPTH
% ========================================
\setcounter{tocdepth}{2} % Show only parts, sections, and subsections

% ========================================
% DOCUMENT INFORMATION
% ========================================
\title{\textbf{Machine Learning \& Deep Learning}\\
\large Artificial Intelligence}
\author{Jacopo Parretti}
\date{I Semester 2025-2026}

% ========================================
% DOCUMENT
% ========================================
\begin{document}

\maketitle
\newpage
\tableofcontents
\newpage

\part{Introduction to Machine Learning}

\section{What is Machine Learning?}

Machine Learning is a branch of Artificial Intelligence that enables software to use data to find solutions to specific tasks without being explicitly programmed to do so.

\subsection{Machine Learning vs Statistical Modelling}

In traditional statistical modelling, we follow a structured process:
\begin{enumerate}
    \item Collect data
    \item Verify and clean the data (correct or discard if not clean)
    \item Use the clean data to test hypotheses, make predictions and forecasts
\end{enumerate}

In contrast, Machine Learning takes a different approach:
\begin{itemize}
    \item It is the \textbf{data} that determines which analytic techniques to use
    \item The computer uses data to \textbf{train} algorithms to find patterns and make predictions
    \item Algorithms are no longer \textbf{static} but become \textbf{dynamic}
\end{itemize}

\subsection{Conventional Programming vs Machine Learning}

The fundamental difference between conventional programming and machine learning can be understood through their inputs and outputs:

\subsubsection{Conventional Programming}
\begin{itemize}
    \item \textbf{Input}: Program + Data
    \item \textbf{Output}: Result
    \item You write the program, give it data, and get results
\end{itemize}

\subsubsection{Machine Learning}
\begin{itemize}
    \item \textbf{Input}: Data + Result
    \item \textbf{Output}: Program
    \item You give the computer data and desired results, and it learns to generate a program/algorithm
\end{itemize}

\subsection{The Machine Learning Recipe}

The ML recipe consists of 4 fundamental steps:
\begin{enumerate}
    \item Collect data
    \item Define a family of possible models
    \item Define an objective (error) function to quantify how well a model fits the data
    \item Find the model that minimizes the error function (training/learning a model)
\end{enumerate}

\subsubsection{The Key Ingredients}

Every machine learning problem requires five essential ingredients:
\begin{itemize}
    \item \textbf{Task}: The problem we want to solve
    \item \textbf{Data}: The information we use to learn
    \item \textbf{Model hypothesis}: The family of functions we consider
    \item \textbf{Objective function}: How we measure success
    \item \textbf{Learning algorithm}: How we find the best model
\end{itemize}

\section{Machine Learning Tasks}

A task represents the type of prediction being made to solve a problem on some data. We can identify a task with the set of functions that can potentially solve the problem. In general, it consists of functions assigning each \textbf{input} $x \in \mathcal{X}$ an \textbf{output} $y \in \mathcal{Y}$.

\subsection{Classification Task}

\begin{definition}[Classification]
Find a function $f \in \mathcal{Y}^{\mathcal{X}}$ assigning each input $x \in \mathcal{X}$ a \textbf{discrete} label, $f(x) \in \mathcal{Y} = \{c_1, \ldots, c_k\}$.
\end{definition}

In classification, the output space is a finite set of discrete categories or classes. The goal is to learn a decision boundary that separates different classes.

\subsection{Regression Task}

\begin{definition}[Regression]
Find a function $f \in \mathcal{Y}^{\mathcal{X}}$ assigning each input $x \in \mathcal{X}$ a \textbf{continuous} label, $f(x) \in \mathcal{Y} = \mathbb{R}$.
\end{definition}

In regression, the output is a continuous value, allowing for predictions of quantities such as prices, temperatures, or probabilities.

\subsection{Density Estimation Task}

\begin{definition}[Density Estimation]
Find a probability distribution $f \in \Delta(\mathcal{X})$ that best fits the data $x \in \mathcal{X}$.
\end{definition}

There is no reference to an output space (as in classification and regression tasks). Instead, we make a reasoning on the \textbf{input} $x$ itself, trying to model the underlying probability distribution that generated the data.

\subsection{Clustering Task}

\begin{definition}[Clustering]
Find a function $f \in \mathbb{N}^{\mathcal{X}}$ that assigns each input $x \in \mathcal{X}$ a \textbf{cluster index} $f(x) \in \mathbb{N}$.
\end{definition}

All points mapped to the same index form a cluster. The goal is to group similar data points together without prior knowledge of the groups.

\subsection{Dimensionality Reduction Task}

\begin{definition}[Dimensionality Reduction]
Find a function $f \in \mathcal{Y}^{\mathcal{X}}$ that \textbf{maps} each (\textbf{high-dimensional}) input $x \in \mathcal{X}$ to a \textbf{lower-dimensional} embedding $f(x) \in \mathcal{Y}$, where $\dim(\mathcal{Y}) < \dim(\mathcal{X})$.
\end{definition}

This task aims to reduce the number of features while preserving the essential information in the data.

\section{Data in Machine Learning}

\subsection{Data Representation}

We represent inputs of our algorithm as $x \in \mathcal{X}$ and outputs as $y \in \mathcal{Y}$. The dataset is the set of both, where each input is associated with an output.

\textbf{Key considerations:}
\begin{itemize}
    \item The quality of the data is crucial for the performance of the algorithm
    \item It is equally important to have a large amount of data to learn effective models
    \item Poor quality or insufficient data can lead to poor model performance
\end{itemize}

\subsection{Features}

\begin{definition}[Features]
Features are the measurable properties or characteristics of the phenomenon being observed.
\end{definition}

\textbf{Examples:}
\begin{itemize}
    \item To identify a flower: color, petal length, petal width, sepal length, sepal width
    \item To identify a fruit: color, shape, size, texture, weight
    \item To classify images: pixel values, edges, textures, shapes
\end{itemize}

We can say that features are how our algorithm "sees" the data and are in general represented with (fixed-size) vectors.

\section{Data Distributions and Learning Types}

In machine learning, the information about the problem we want to solve is represented in the form of a data distribution, usually denoted as $p_{\text{data}}$.

\subsection{Supervised Learning}

\subsubsection{Classification and Regression}

The data distribution is over pairs of inputs and outputs:
\[
p_{\text{data}} \in \Delta(\mathcal{X} \times \mathcal{Y})
\]

Here:
\begin{itemize}
    \item $\mathcal{X}$ is the input space
    \item $\mathcal{Y}$ is the output space (labels or targets)
    \item The goal is to learn a mapping from inputs to outputs using labeled data
\end{itemize}

\subsection{Unsupervised Learning}

\subsubsection{Density Estimation, Clustering, and Dimensionality Reduction}

The data distribution is only over the input space:
\[
p_{\text{data}} \in \Delta(\mathcal{X})
\]

Here:
\begin{itemize}
    \item We do not have explicit labels or targets
    \item The goal is to discover structure, patterns, or representations within the data itself
\end{itemize}

\subsection{Summary of Learning Types}

\begin{table}[h]
\centering
\begin{tabular}{|l|c|c|}
\hline
\textbf{Task Type} & \textbf{Data Distribution} & \textbf{Learning Type} \\
\hline
Classification, Regression & $p_{\text{data}} \in \Delta(\mathcal{X} \times \mathcal{Y})$ & Supervised Learning \\
\hline
Density Estimation, Clustering, & $p_{\text{data}} \in \Delta(\mathcal{X})$ & Unsupervised Learning \\
Dimensionality Reduction & & \\
\hline
\end{tabular}
\caption{Comparison of learning types based on data distribution}
\end{table}

\begin{remark}
In \textbf{supervised learning}, each data point consists of an input and a corresponding output (label). In \textbf{unsupervised learning}, each data point consists only of the input, and the algorithm tries to find patterns or structure without explicit labels.

This distinction is fundamental in machine learning, as it determines the type of algorithms and approaches used to solve different problems.
\end{remark}

\section{Data Sampling and Dataset Splitting}

\subsection{The Data Distribution Problem}

In both supervised and unsupervised learning, the true data distribution $p_{\text{data}}$ is typically \textbf{unknown}. This presents a fundamental challenge:
\begin{itemize}
    \item We do not have direct access to the underlying probability distribution that generates the data
    \item We only have access to a finite sample of data points drawn from this distribution
\end{itemize}

\textbf{Solution}: We perform \textbf{sampling} from the distribution. By collecting a representative sample of data, we can approximate the true distribution and use it to train our models.

\subsection{Training, Validation, and Test Sets}

To properly evaluate machine learning models, we split our data into three distinct sets. Each set serves a specific purpose in the machine learning pipeline.

\begin{definition}[Dataset Splitting]
A dataset is typically divided into three subsets:
\begin{itemize}
    \item \textbf{Training set} ($D_{\text{train}}$): Used to train the model
    \item \textbf{Validation set} ($D_{\text{val}}$): Used to tune hyperparameters and select the best model
    \item \textbf{Test set} ($D_{\text{test}}$): Used to evaluate the final model performance
\end{itemize}
\end{definition}

\begin{remark}
\textbf{Critical requirement}: All three sets (training, validation, and test) should be sampled from the \textbf{same probability distribution}. This ensures that the model's performance on the test set is a reliable indicator of its performance on new, unseen data.
\end{remark}

\subsection{How to Use the Three Sets}

The three datasets serve different purposes in the machine learning workflow:

\subsubsection{Learning/Training Phase}

During the learning phase, the \textbf{training set} is used to train the machine learning algorithm, which learns patterns and relationships in the data to produce a \textbf{program} (model). 

The \textbf{validation set} plays a crucial role in this phase:
\begin{itemize}
    \item Evaluates different model configurations
    \item Tunes hyperparameters
    \item Selects the best performing model
\end{itemize}

This process involves iterative feedback between training and validation until we find the optimal configuration.

\subsubsection{Testing/Model Evaluation Phase}

Once the best model is selected, we use the \textbf{test set} to evaluate its performance. The trained model is applied to the test data, which it has never seen before. This gives us an unbiased estimate of how well the model will perform in real-world scenarios, telling us whether our model can truly generalize to new data.

\subsection{The Importance of Distribution Similarity}

\subsubsection{Correct Scenario: Same Distribution}

When training, validation, and test sets are sampled from the \textbf{same distribution}:
\begin{itemize}
    \item The sets are \textbf{similar} in terms of statistical properties
    \item They represent the same underlying phenomenon
    \item However, they \textbf{do not overlap} (no data point appears in multiple sets)
\end{itemize}

This ensures that the model can generalize well from training to test data.

\textbf{Example}: If we're classifying fruits, all three sets might contain apples and bananas in similar proportions, but with different specific instances.

\subsubsection{Incorrect Scenario: Different Distributions}

When training/validation and test sets come from \textbf{different distributions}, the model learns patterns from one distribution but is tested on another. This problematic scenario typically leads to:
\begin{itemize}
    \item Poor performance on the test set
    \item Unreliable predictions on new data
    \item Inability to generalize to real-world applications
\end{itemize}

\textbf{Example}: Training on apples and bananas, but testing on oranges and limes. The model cannot classify fruits it has never seen during training.

\begin{remark}
\textbf{Key principle}: The fundamental assumption in machine learning is that training and test data come from the same distribution. Violating this assumption leads to poor generalization and unreliable models.
\end{remark}

\section{Training and Testing: The Role of Features}

\subsection{Training (Learning, Induction)}

During the training phase, the model learns to distinguish between different classes based on the features provided in the training data. 

\textbf{Example}: Consider a fruit classification task with the following training data:
\begin{itemize}
    \item \textit{red, round, leaf, 3oz, ...} $\rightarrow$ apple
    \item \textit{green, round, no leaf, 4oz, ...} $\rightarrow$ apple
    \item \textit{yellow, curved, no leaf, 8oz, ...} $\rightarrow$ banana
    \item \textit{green, curved, no leaf, 7oz, ...} $\rightarrow$ banana
\end{itemize}

The model learns to associate patterns in the features (color, shape, presence of leaf, weight) with the corresponding labels (apple or banana). \textbf{What distinguishes apples and bananas is based on the features!} The model identifies which feature combinations are characteristic of each class.

\subsection{Testing}

Once trained, the model can classify new examples based on the same features it learned during training. When presented with a new example described by the same semantic features, the model makes a prediction.

\textbf{Example}: Given a new fruit with features \textit{red, round, no leaf, 4oz, ...}, the model uses the learned patterns to predict whether it is an apple or a banana.

\begin{remark}
\textbf{Critical assumption}: The features of the training and testing data must be the same! The model can only make predictions based on the features it was trained on. If the test data uses different features or feature representations, the model cannot function properly.
\end{remark}

\subsection{Learning and Generalization}

Learning is fundamentally about \textbf{generalizing} from the training data. The goal is not simply to memorize the training examples, but to learn patterns that apply to new, unseen data.

However, the failure of a machine learning algorithm is often caused by a \textbf{bad selection of training samples}. The key issue is that we might introduce \textbf{unwanted correlations} from which the algorithm derives \textbf{wrong conclusions}.

\textbf{Example}: Consider a model trained to recognize objects in images. If all training images were captured on sunny days, the model might learn to associate sunny weather conditions with the objects. When tested on images taken on cloudy days, the model might fail to recognize the same objects because it never encountered cloudy conditions during training. The model incorrectly learned that sunny weather is a relevant feature for object recognition.

\begin{observation}
The quality and diversity of training data are crucial. Training data should be representative of all conditions the model will encounter in real-world applications, avoiding spurious correlations that lead to poor generalization.
\end{observation}

\section{The Complete Machine Learning Recipe}

We can now revisit the complete machine learning workflow with all its essential ingredients:

\subsection{The Five Ingredients of Machine Learning}

Every machine learning problem requires five fundamental components that work together:

\begin{enumerate}
    \item \textbf{Task}: The specific problem we want to solve (e.g., classify birds vs. non-birds in images)
    
    \item \textbf{Data}: The collection of examples used to train and evaluate the model. Data quality and quantity are critical for success.
    
    \item \textbf{Model Hypothesis}: The family of functions or models we consider as potential solutions. This defines the space of possible models the learning algorithm can explore.
    
    \item \textbf{Objective Function}: A mathematical function that quantifies how well a model performs on the data. The learning algorithm aims to optimize this function.
    
    \item \textbf{Learning Algorithm}: The method used to search through the model hypothesis space and find the model that optimizes the objective function.
\end{enumerate}

\subsection{The Interplay of Ingredients}

These five ingredients are interconnected:
\begin{itemize}
    \item The \textbf{task} determines what kind of \textbf{data} we need and what \textbf{model hypothesis} is appropriate
    \item The \textbf{data} influences which features are available and how we define the \textbf{objective function}
    \item The \textbf{model hypothesis} constrains what the \textbf{learning algorithm} can discover
    \item The \textbf{objective function} guides the \textbf{learning algorithm} toward better models
\end{itemize}

Understanding and carefully selecting each ingredient is essential for building effective machine learning systems.

\section{Model and Hypothesis Space}

\subsection{What is a Model?}

\begin{definition}[Model]
A \textbf{model} is the implementation of a function $f \in \mathcal{F}_{\text{task}}$ that can be tractably computed.
\end{definition}

In practice, when searching for our function $f$, we don't look at everything in $\mathcal{F}_{\text{task}}$. Instead, we look at a subset called the \textbf{hypothesis space}: $\mathcal{H} \subset \mathcal{F}_{\text{task}}$, which is composed of a set of models (e.g., neural networks, decision trees, linear models).

The learning algorithm seeks a \textbf{solution within the hypothesis space}. In other words, a model is a simplified representation of the world that we can actually compute and optimize.

\begin{observation}
The choice of hypothesis space is crucial:
\begin{itemize}
    \item If $\mathcal{H}$ is too small, it may not contain a good solution
    \item If $\mathcal{H}$ is too large, finding the best model becomes computationally intractable
    \item The hypothesis space defines the types of patterns the model can learn
\end{itemize}
\end{observation}

\subsection{Types of Learning Paradigms}

Machine learning can be categorized into different paradigms based on the type and availability of labeled data:

\subsubsection{Supervised Learning}

In supervised learning, we have labeled data where each input is paired with its corresponding output. This includes:
\begin{itemize}
    \item \textbf{Classification}: Predicting discrete labels
    \item \textbf{Regression}: Predicting continuous values
\end{itemize}

\subsubsection{Unsupervised Learning}

In unsupervised learning, we only have input data without labels. The goal is to discover structure in the data. This includes:
\begin{itemize}
    \item \textbf{Clustering}: Grouping similar data points
    \item \textbf{Density Estimation}: Modeling the probability distribution of data
    \item \textbf{Dimensionality Reduction}: Finding lower-dimensional representations
\end{itemize}

\subsubsection{Semi-Supervised Learning}

Semi-supervised learning addresses scenarios where we have a large amount of unlabeled data and only some labeled examples. This paradigm has gained significant popularity in recent years to leverage large datasets through self-supervised learning techniques.

The key idea is to provide the model with tasks different from the one to be solved, but that allow the model to learn the data distribution from the unlabeled data. Once the model understands the general structure of the data, it can then specialize on the few labeled examples to solve the target task.

\textbf{Example}: Training a language model on large amounts of unlabeled text to learn general language patterns, then fine-tuning it on a small labeled dataset for sentiment analysis.

\section{Model-Based vs. Instance-Based Learning}

Machine learning algorithms can be further categorized based on how they make predictions:

\subsection{Model-Based (Parametric) Learning}

\begin{definition}[Parametric Model]
A learning model that summarizes data with a \textbf{set of parameters of fixed size} (independent of the number of training examples) is called a \textbf{parametric model}.
\end{definition}

Key characteristics:
\begin{itemize}
    \item The model learns a fixed set of parameters from the training data
    \item Once trained, the original training data can be discarded
    \item No matter how much data you give to a parametric model, it won't change its mind about how many parameters it needs
    \item The model makes predictions using only the learned parameters
\end{itemize}

\textbf{Examples}: Naive Bayes, Linear Regression, Logistic Regression, Neural Networks

In a model-based approach, the algorithm learns a function (e.g., a decision boundary) described by parameters. For instance, a linear classifier might learn the curve $\alpha x_1 + \beta x_2 + \gamma$ that separates two classes, where $\alpha, \beta, \gamma$ are the parameters.

\subsection{Instance-Based (Nonparametric) Learning}

\begin{definition}[Nonparametric Model]
These algorithms \textbf{do not learn parameters} but instead use the data itself to compare a new example through similarity metrics.
\end{definition}

Key characteristics:
\begin{itemize}
    \item The model stores (some or all of) the training data
    \item Predictions are made by comparing new examples to stored training examples
    \item The model's complexity can grow with the amount of training data
    \item No explicit training phase that produces a fixed set of parameters
\end{itemize}

\textbf{Examples}: K-Nearest Neighbors (KNN), Support Vector Machines (SVM), Random Forest

In an instance-based approach, when a new example arrives, it is compared with nearby training examples and classified according to a similarity policy. For example, in KNN, a new point is classified based on the majority class of its $k$ nearest neighbors in the training data.

\subsection{Comparison}

\begin{table}[h]
\centering
\begin{tabular}{|l|p{5.5cm}|p{5.5cm}|}
\hline
\textbf{Aspect} & \textbf{Model-Based} & \textbf{Instance-Based} \\
\hline
Parameters & Fixed number of parameters & No fixed parameters \\
\hline
Training data & Can be discarded after training & Must be retained for predictions \\
\hline
Prediction speed & Fast (only uses parameters) & Slower (compares with stored data) \\
\hline
Memory & Low (stores only parameters) & High (stores training examples) \\
\hline
Flexibility & Fixed model complexity & Complexity grows with data \\
\hline
\end{tabular}
\caption{Comparison between model-based and instance-based learning}
\end{table}

\section{Error Function (Objective Function)}

To allow an algorithm to learn, we need to provide it with a metric, a measure that helps it understand the direction of the optimal solution we are seeking. For this reason, \textbf{performance measures} are defined to enable the algorithm to learn.

\subsection{What is an Error Function?}

In machine learning, \textbf{training a model} means finding a \textbf{function} which maps a set of values $x$ to a value $y$. We can calculate how well a predictive model is doing by comparing the \textbf{predicted values} with the \textbf{true values} for $y$.

\begin{definition}[Training Error vs. Test Error]
\begin{itemize}
    \item If we apply the model to the data it was trained on, we are calculating the \textbf{training error}
    \item If we calculate the error on data which was \textbf{unknown} in the training phase, we are calculating the \textbf{test error}
\end{itemize}
\end{definition}

The test error is the most important metric because it tells us how well the model generalizes to new, unseen data.

\subsection{Types of Error Functions}

There are different types of error functions (also called loss functions or objective functions), each suited for different tasks:

\begin{itemize}
    \item \textbf{Mean Absolute Error (MAE)}: Measures the average absolute difference between predictions and true values
    \item \textbf{Mean Squared Error (MSE)}: Measures the average squared difference between predictions and true values
    \item \textbf{Binary Cross-Entropy}: Used for binary classification problems
    \item \textbf{Mean Average Precision}: Used for ranking and information retrieval tasks
\end{itemize}

The choice of error function depends on the task and the desired properties of the model.

\section{Example: Polynomial Fitting}

Let's consider a concrete example to understand how models, hypothesis spaces, and error functions work together.

\subsection{Problem Setup}

Consider a regression problem with the following setup:
\begin{itemize}
    \item \textbf{Data}: $\mathcal{D}_n = \{(x_1, y_1), \ldots, (x_n, y_n)\}$
    \item Data generated from $\sin(2\pi x) + \text{noise}$
    \item Training set with $n = 10$ points
    \item \textbf{Goal}: Learn a function (shown as the purple line) that fits the data
\end{itemize}

The correct solution should capture the underlying sinusoidal pattern while being robust to the noise in the training data.

\begin{figure}[h]
\centering
\begin{tikzpicture}
\begin{axis}[
    width=10cm,
    height=7cm,
    xlabel={$x$},
    ylabel={$y$},
    xmin=0, xmax=1,
    ymin=-1.5, ymax=1.5,
    grid=major,
    legend pos=north east,
    legend style={font=\small}
]

% True function (purple, sinusoidal)
\addplot[violet, very thick, smooth, domain=0:1, samples=100] {sin(deg(2*pi*x))};
\addlegendentry{Correct solution}

% Training data points (black dots with some noise)
\addplot[only marks, mark=*, mark size=2pt, black] coordinates {
    (0.05, 0.35)
    (0.15, 0.92)
    (0.25, 0.88)
    (0.35, 0.65)
    (0.45, -0.15)
    (0.55, -0.82)
    (0.65, -0.95)
    (0.75, -0.72)
    (0.85, 0.05)
    (0.95, 0.85)
};
\addlegendentry{Training data}

\end{axis}
\end{tikzpicture}
\caption{Polynomial Fitting Problem: Data generated from $\sin(2\pi x)$ with noise}
\end{figure}

\subsection{Model: Polynomial Functions}

We choose to model the data using polynomial functions:

\[
f_w(x) = \sum_{j=0}^{M} w_j x^j
\]

where:
\begin{itemize}
    \item $M$ is the \textbf{degree of the polynomial} (a hyperparameter)
    \item $w = \{w_0, \ldots, w_M\}$ are the \textbf{parameters} to be learned
    \item The \textbf{hypothesis space} for a fixed degree $M$ is: $\mathcal{H}_M = \{f_w : w \in \mathbb{R}^{M+1}\}$
\end{itemize}

Different values of $M$ give us different hypothesis spaces:
\begin{itemize}
    \item $M = 0$: Constant functions (horizontal lines)
    \item $M = 1$: Linear functions (straight lines)
    \item $M = 2$: Quadratic functions (parabolas)
    \item $M = 3$: Cubic functions
    \item Higher $M$: More complex curves
\end{itemize}

\begin{figure}[h]
\centering
\begin{tikzpicture}

% M=0 (Underfitting)
\begin{scope}[xshift=0cm]
\begin{axis}[
    width=5.5cm,
    height=4.5cm,
    title={$M=0$ (Underfitting)},
    title style={font=\small},
    xlabel={$x$},
    ylabel={$y$},
    xlabel style={font=\small},
    ylabel style={font=\small},
    xmin=0, xmax=1,
    ymin=-1.5, ymax=1.5,
    xtick=\empty,
    ytick=\empty,
]
% Data points
\addplot[only marks, mark=*, mark size=1.5pt, black] coordinates {
    (0.05, 0.35) (0.15, 0.92) (0.25, 0.88) (0.35, 0.65) (0.45, -0.15)
    (0.55, -0.82) (0.65, -0.95) (0.75, -0.72) (0.85, 0.05) (0.95, 0.85)
};
% True function (purple)
\addplot[violet, thick, smooth, domain=0:1, samples=50] {sin(deg(2*pi*x))};
% M=0 fit (green horizontal line)
\addplot[green!70!black, very thick, domain=0:1] {0.1};
\end{axis}
\end{scope}

% M=3 (Good fit)
\begin{scope}[xshift=6.5cm]
\begin{axis}[
    width=5.5cm,
    height=4.5cm,
    title={$M=3$ (Good Fit)},
    title style={font=\small},
    xlabel={$x$},
    ylabel={$y$},
    xlabel style={font=\small},
    ylabel style={font=\small},
    xmin=0, xmax=1,
    ymin=-1.5, ymax=1.5,
    xtick=\empty,
    ytick=\empty,
]
% Data points
\addplot[only marks, mark=*, mark size=1.5pt, black] coordinates {
    (0.05, 0.35) (0.15, 0.92) (0.25, 0.88) (0.35, 0.65) (0.45, -0.15)
    (0.55, -0.82) (0.65, -0.95) (0.75, -0.72) (0.85, 0.05) (0.95, 0.85)
};
% True function (purple)
\addplot[violet, thick, smooth, domain=0:1, samples=50] {sin(deg(2*pi*x))};
% M=3 fit (green, close to true function)
\addplot[green!70!black, very thick, smooth, domain=0:1, samples=50] {sin(deg(2*pi*x)) + 0.1*sin(deg(6*pi*x))};
\end{axis}
\end{scope}

% M=9 (Overfitting)
\begin{scope}[xshift=3.25cm, yshift=-5.5cm]
\begin{axis}[
    width=5.5cm,
    height=4.5cm,
    title={$M=9$ (Overfitting)},
    title style={font=\small},
    xlabel={$x$},
    ylabel={$y$},
    xlabel style={font=\small},
    ylabel style={font=\small},
    xmin=0, xmax=1,
    ymin=-1.5, ymax=1.5,
    xtick=\empty,
    ytick=\empty,
]
% Data points
\addplot[only marks, mark=*, mark size=1.5pt, black] coordinates {
    (0.05, 0.35) (0.15, 0.92) (0.25, 0.88) (0.35, 0.65) (0.45, -0.15)
    (0.55, -0.82) (0.65, -0.95) (0.75, -0.72) (0.85, 0.05) (0.95, 0.85)
};
% True function (purple)
\addplot[violet, thick, smooth, domain=0:1, samples=50] {sin(deg(2*pi*x))};
% M=9 fit (green, oscillating wildly)
\addplot[green!70!black, very thick, smooth, domain=0:1, samples=100] {
    sin(deg(2*pi*x)) + 0.8*sin(deg(20*pi*x)) + 0.3*cos(deg(30*pi*x))
};
\end{axis}
\end{scope}

\end{tikzpicture}
\caption{Polynomial fits with different degrees: $M=0$ (underfitting), $M=3$ (good fit), $M=9$ (overfitting). Purple: true function, Green: fitted polynomial, Black dots: training data.}
\end{figure}

\subsection{Error Function: Mean Squared Error}

To measure how well our polynomial fits the data, we use the Mean Squared Error (MSE):

\[
E(f; \mathcal{D}_n) = \frac{1}{n} \sum_{i=1}^{n} [f(x_i) - y_i]^2
\]

This error function:
\begin{itemize}
    \item Computes the squared difference between the prediction $f(x_i)$ and the ground-truth label $y_i$ for each point
    \item Averages these squared differences over all $n$ training points
    \item Is also called a \textbf{pointwise loss} because it measures the error at each individual data point
\end{itemize}

The learning algorithm's goal is to find the parameters $w$ that minimize this error function:

\[
w^* = \arg\min_{w \in \mathbb{R}^{M+1}} E(f_w; \mathcal{D}_n)
\]

\begin{observation}
The choice of polynomial degree $M$ is crucial:
\begin{itemize}
    \item Too small $M$ (e.g., $M=0$ or $M=1$): The model is too simple and cannot capture the sinusoidal pattern (\textbf{underfitting})
    \item Appropriate $M$ (e.g., $M=2$ or $M=3$): The model captures the underlying pattern well
    \item Too large $M$: The model fits the noise in the training data and doesn't generalize well (\textbf{overfitting})
\end{itemize}
\end{observation}

\section{Underfitting and Overfitting}

Two fundamental problems can occur when training machine learning models: underfitting and overfitting. Understanding these concepts is crucial for building models that generalize well.

\subsection{Definitions}

\begin{definition}[Underfitting]
\textbf{Underfitting} is a scenario where a model is unable to capture the relationship between the input ($x$) and output ($y$) variables accurately, generating a high error rate on both the training set and unseen data (e.g., testing set).
\end{definition}

The model is too simple and has not yet learned from the data. It performs poorly on both training and test data.

\begin{definition}[Overfitting]
\textbf{Overfitting} occurs when the model gives accurate predictions for training data (lower training error) but not for testing data (higher testing error).
\end{definition}

The model is too sensitive to the training data and has essentially memorized it, including its noise and peculiarities, rather than learning the underlying pattern.

\subsection{The Bias-Variance Trade-off}

The relationship between model complexity, training error, and test error can be visualized as follows:

\begin{figure}[h]
\centering
\begin{tikzpicture}
\begin{axis}[
    width=13cm,
    height=8cm,
    xlabel={Model Complexity},
    ylabel={Error},
    xmin=0, xmax=10,
    ymin=0, ymax=5.5,
    xtick=\empty,
    ytick=\empty,
    axis lines=left,
    legend pos=north west,
    legend style={font=\normalsize}
]

% Training Error curve (decreasing)
\addplot[blue, thick, smooth, domain=0:10] {1.5 + 3*exp(-0.5*x)};
\addlegendentry{Training Error}

% Test Error curve (U-shaped)
\addplot[orange, thick, smooth, domain=0:10] {2 + 2*exp(-0.8*x) + 0.08*x^2};
\addlegendentry{Test Error}

% Vertical line for "Best Fit"
\draw[red, dashed, thick] (axis cs:3.5,0) -- (axis cs:3.5,2.2);

\end{axis}
\end{tikzpicture}
\caption{Bias-Variance Trade-off: Training and Test Error vs. Model Complexity}
\end{figure}

\begin{itemize}
    \item \textbf{Left side of the graph (low complexity)}: High error rate in both training and testing $\rightarrow$ \textbf{Underfitting}
    \item \textbf{Middle region (optimal complexity)}: Low error rate in both training and testing $\rightarrow$ \textbf{Best Fit}
    \item \textbf{Right side of the graph (high complexity)}: Low error rate in training but high error rate in testing $\rightarrow$ \textbf{Overfitting}
\end{itemize}

\subsection{Diagnostic Matrix}

We can diagnose the state of our model by examining both training and test errors:

\begin{table}[h]
\centering
\begin{tabular}{|c|c|c|}
\hline
 & \textbf{Low Training Error} & \textbf{High Training Error} \\
\hline
\textbf{Low Test Error} & OK (Good fit) & Underfitting \\
\hline
\textbf{High Test Error} & Overfitting & Underfitting \\
\hline
\end{tabular}
\caption{Diagnostic matrix for model performance}
\end{table}

\subsection{How to Handle Overfitting}

When the model is too sensitive to training data and overfits, we can apply several strategies:

\begin{enumerate}
    \item \textbf{Try a simpler model}: Use a model with fewer parameters or lower complexity
    
    \item \textbf{Try a less powerful model}: Choose a model architecture with reduced capacity
    
    \item \textbf{Increase regularization impact}: Apply techniques that penalize model complexity:
    \begin{itemize}
        \item Early stopping: Stop training before the model overfits
        \item L1/L2 regularization: Add penalty terms to the loss function
    \end{itemize}
    
    \item \textbf{Use a smaller number of features}:
    \begin{itemize}
        \item Remove some features that may be causing overfitting
        \item Apply feature selection techniques to identify the most relevant features
    \end{itemize}
    
    \item \textbf{Get more data}: Increasing the training set size helps the model learn more generalizable patterns
\end{enumerate}

\subsection{How to Handle Underfitting}

When the model has not yet learned from the data and underfits, we can try:

\begin{enumerate}
    \item \textbf{Try more complex models}: Use models with a larger number of parameters that can capture more intricate patterns
    \begin{itemize}
        \item Ensemble learning: Combine multiple models
    \end{itemize}
    
    \item \textbf{Less regularization}: Reduce or remove regularization constraints that may be limiting the model's capacity
    
    \item \textbf{A larger quantity of features}: Add more features or engineer new features that better capture the underlying relationships (get more features)
\end{enumerate}

\subsection{Polynomial Fitting Example Revisited}

Let's examine how different polynomial degrees affect the fit:

\begin{itemize}
    \item \textbf{$M = 0$} (constant function): Severe underfitting. The horizontal line cannot capture any of the sinusoidal pattern. Both training and validation errors are high.
    
    \item \textbf{$M = 1$} (linear function): Still underfitting. A straight line cannot represent the curved pattern. Both errors remain high.
    
    \item \textbf{$M = 3$} (cubic polynomial): Good fit. The model captures the underlying sinusoidal pattern well. Both training and validation errors are low, and the curves are similar.
    
    \item \textbf{$M = 9$} (9th-degree polynomial): Overfitting. The model passes through all training points (very low training error) but oscillates wildly between them. The validation error is high because the model has learned the noise rather than the signal.
\end{itemize}

The error plot shows:

\begin{figure}[h]
\centering
\begin{tikzpicture}
\begin{axis}[
    width=12cm,
    height=8cm,
    xlabel={$M$ (Polynomial Degree)},
    ylabel={Error},
    xlabel style={font=\normalsize},
    ylabel style={font=\normalsize},
    xmin=-0.5, xmax=9.5,
    ymin=0, ymax=4.5,
    xtick={0,1,2,3,4,5,6,7,8,9},
    grid=major,
    legend pos=north east,
    legend style={font=\normalsize}
]

% Training error (blue, decreasing)
\addplot[blue, thick, mark=*, smooth] coordinates {
    (0, 3.5)
    (1, 2.8)
    (2, 1.8)
    (3, 1.2)
    (4, 0.8)
    (5, 0.5)
    (6, 0.3)
    (7, 0.2)
    (8, 0.15)
    (9, 0.1)
};
\addlegendentry{Training}

% Validation error (orange, U-shaped)
\addplot[orange, thick, mark=square*, smooth] coordinates {
    (0, 3.5)
    (1, 2.5)
    (2, 1.5)
    (3, 1.0)
    (4, 1.1)
    (5, 1.3)
    (6, 1.6)
    (7, 2.0)
    (8, 2.5)
    (9, 3.2)
};
\addlegendentry{Validation}


\end{axis}
\end{tikzpicture}
\caption{Training and Validation Error vs. Polynomial Degree $M$}
\end{figure}

\begin{itemize}
    \item \textbf{Training error} (blue line): Decreases monotonically as $M$ increases
    \item \textbf{Validation error} (orange line): Decreases initially, reaches a minimum around $M = 3$, then increases again
    \item The optimal model complexity is where the validation error is minimized
\end{itemize}

\begin{remark}
The key insight is that minimizing training error alone is not sufficient. We must monitor validation/test error to ensure the model generalizes well to new data. The best model is the one that balances fitting the training data with generalizing to unseen data.
\end{remark}

\section{Regularization}

Regularization is one of the most important techniques in machine learning to prevent overfitting and improve model generalization.

\subsection{What is Regularization?}

\begin{definition}[Regularization]
Regularization refers to techniques that are used to calibrate machine learning models in order to minimize the adjusted loss function and prevent overfitting or underfitting.
\end{definition}

In practice, regularization is a modification of the training error function by appending a term $\Omega(f)$ that typically penalizes complex solutions.

\subsection{The Regularized Objective Function}

The regularized objective function has the following form:

\[
E_{\text{reg}}(f; \mathcal{D}_n) = E(f; \mathcal{D}_n) + \lambda_n \Omega(f)
\]

where:
\begin{itemize}
    \item $E(f; \mathcal{D}_n)$ is the \textbf{training error function} (e.g., MSE)
    \item $\Omega(f)$ is the \textbf{regularization term} that penalizes model complexity
    \item $\lambda_n$ is the \textbf{tradeoff hyperparameter} that controls the strength of regularization
\end{itemize}

The regularization term acts as a penalty that discourages the model from becoming too complex. By minimizing this combined objective, we find models that fit the data well while remaining relatively simple.

\subsection{Effect of Regularization}

The regularization term $\Omega(f)$ typically measures some notion of model complexity. When we minimize the regularized objective:
\begin{itemize}
    \item We balance fitting the training data (low $E(f; \mathcal{D}_n)$) with keeping the model simple (low $\Omega(f)$)
    \item The hyperparameter $\lambda_n$ controls this tradeoff
    \item Higher $\lambda_n$ means stronger regularization (simpler models)
    \item Lower $\lambda_n$ means weaker regularization (more complex models allowed)
\end{itemize}

\subsection{Regularization in Polynomial Fitting}

For the polynomial fitting example, we can regularize by penalizing polynomials with large coefficients. The regularized error function becomes:

\[
E_{\text{reg}}(f_w; \mathcal{D}_n) = \frac{1}{n} \sum_{i=1}^{n} [f_w(x_i) - y_i]^2 + \frac{\lambda}{n} \|w\|^2
\]

where:
\begin{itemize}
    \item The first term is the training error (MSE)
    \item $\frac{\lambda}{n} \|w\|^2 = \frac{\lambda}{n} \sum_{j=0}^{M} w_j^2$ is the regularization term (L2 regularization)
    \item $\lambda$ is the tradeoff hyperparameter
\end{itemize}

\subsubsection{Effect of the Regularization Parameter $\lambda$}

The choice of $\lambda$ significantly affects the model:

\begin{itemize}
    \item \textbf{$\lambda \approx 10^{-18}$ (very small)}: Almost no regularization. The model can have large coefficients and may overfit. The polynomial fits the training data very closely, including noise.
    
    \item \textbf{$\lambda = 1$ (moderate)}: Balanced regularization. The model is penalized for having large coefficients, leading to a smoother curve that generalizes better. This often provides the best tradeoff between training and validation error.
    
    \item \textbf{$\lambda$ very large}: Strong regularization. The penalty for large coefficients is so high that the model becomes too simple and may underfit. The curve becomes nearly flat.
\end{itemize}

\subsubsection{Regularization Path}

The plot of error vs. $\ln(\lambda)$ shows:

\begin{figure}[h]
\centering
\begin{tikzpicture}
\begin{axis}[
    width=13cm,
    height=8cm,
    xlabel={$\ln(\lambda)$},
    ylabel={Error},
    xlabel style={font=\normalsize},
    ylabel style={font=\normalsize},
    xmin=-40, xmax=5,
    ymin=0, ymax=5,
    grid=major,
    legend pos=north west,
    legend style={font=\normalsize}
]

% Training error (blue, increasing)
\addplot[blue, thick, smooth, domain=-40:5] {0.1 + 0.05*(x+20)^2/100 + 0.8/(1+exp(-0.3*(x+10)))};
\addlegendentry{Training}

% Validation error (orange, U-shaped)
\addplot[orange, thick, smooth, domain=-40:5] {1.5 + 0.08*(x+5)^2/50 + 1.2*exp(-0.15*(x+20))};
\addlegendentry{Validation}


\end{axis}
\end{tikzpicture}
\caption{Training and Validation Error vs. $\ln(\lambda)$ (Regularization Strength)}
\end{figure}

\begin{itemize}
    \item \textbf{Left side (small $\lambda$)}: Low training error, high validation error $\rightarrow$ Overfitting
    \item \textbf{Middle region}: Both errors are low $\rightarrow$ Good fit
    \item \textbf{Right side (large $\lambda$)}: Both errors increase $\rightarrow$ Underfitting
\end{itemize}

The optimal $\lambda$ is found where the validation error is minimized.

\subsection{Generalization and Data Size}

An important theoretical result in machine learning relates training error to generalization error:

\begin{theorem}[Generalization with Infinite Data]
As the number of training samples approaches infinity, the training error approximates the generalization error:
\[
E(f; \mathcal{D}_n) \to E(f; p_{\text{data}}) \quad \text{as} \quad n \to \infty
\]
\end{theorem}

This means:
\begin{itemize}
    \item With a small dataset ($n = 15$), even a complex model ($M = 9$) may overfit because there isn't enough data to constrain the parameters
    \item With a large dataset ($n = 100$), the same complex model can generalize well because the abundant data prevents overfitting
    \item More data allows us to use more complex models without overfitting
\end{itemize}

\textbf{Visual example}:

\begin{figure}[h]
\centering
\begin{tikzpicture}

% n=15, M=9 (Overfitting)
\begin{scope}[xshift=0cm]
\begin{axis}[
    width=6.5cm,
    height=5.5cm,
    title={$n=15$, $M=9$ (Overfitting)},
    title style={font=\normalsize},
    xlabel={$x$},
    ylabel={$y$},
    xlabel style={font=\small},
    ylabel style={font=\small},
    xmin=0, xmax=1,
    ymin=-1.5, ymax=1.5,
    xtick=\empty,
    ytick=\empty,
]
% Few data points (15 points, but showing representative subset)
\addplot[only marks, mark=*, mark size=2pt, black] coordinates {
    (0.05, 0.35) (0.15, 0.92) (0.25, 0.88) (0.35, 0.65) (0.45, -0.15)
    (0.55, -0.82) (0.65, -0.95) (0.75, -0.72) (0.85, 0.05) (0.95, 0.85)
};
% True function (purple)
\addplot[violet, thick, smooth, domain=0:1, samples=50] {sin(deg(2*pi*x))};
% Overfitted polynomial (blue, oscillating wildly)
\addplot[blue, very thick, smooth, domain=0:1, samples=100] {
    sin(deg(2*pi*x)) + 0.8*sin(deg(20*pi*x)) + 0.3*cos(deg(30*pi*x))
};
\end{axis}
\end{scope}

% n=100, M=9 (Good fit)
\begin{scope}[xshift=7.5cm]
\begin{axis}[
    width=6.5cm,
    height=5.5cm,
    title={$n=100$, $M=9$ (Good Fit)},
    title style={font=\normalsize},
    xlabel={$x$},
    ylabel={$y$},
    xlabel style={font=\small},
    ylabel style={font=\small},
    xmin=0, xmax=1,
    ymin=-1.5, ymax=1.5,
    xtick=\empty,
    ytick=\empty,
]
% Many data points (showing more points)
\addplot[only marks, mark=*, mark size=1pt, black, samples=50, domain=0:1] {sin(deg(2*pi*x)) + 0.15*rand};
% True function (purple)
\addplot[violet, thick, smooth, domain=0:1, samples=50] {sin(deg(2*pi*x))};
% Well-fitted polynomial (blue, smooth)
\addplot[blue, very thick, smooth, domain=0:1, samples=100] {
    sin(deg(2*pi*x)) + 0.05*sin(deg(6*pi*x))
};
\end{axis}
\end{scope}

\end{tikzpicture}
\caption{Effect of data size on generalization: With few data points ($n=15$), a complex model ($M=9$) overfits. With many data points ($n=100$), the same model generalizes well. Purple: true function, Blue: fitted polynomial, Black dots: training data.}
\end{figure}

\begin{itemize}
    \item \textbf{$n = 15$, $M = 9$}: The 9th-degree polynomial overfits, oscillating wildly between the few training points
    \item \textbf{$n = 100$, $M = 9$}: With 100 training points, the same 9th-degree polynomial fits smoothly and generalizes well
\end{itemize}

\begin{observation}
This illustrates why "get more data" is often the most effective solution to overfitting. With sufficient data, even complex models can learn to generalize well. However, collecting more data is not always feasible, which is why regularization and other techniques remain important.
\end{observation}



\newpage

\part{Model Evaluation and Regression Methods}

\section{Model Evaluation for Classification and Regression}

\subsection{Recap: Train, Validation, and Test Sets}

The machine learning workflow involves three distinct datasets, each serving a specific purpose:

\begin{figure}[h]
\centering
\begin{tikzpicture}[
    node distance=1.5cm and 2.5cm,
    box/.style={rectangle, minimum width=2.5cm, minimum height=0.8cm, align=center, font=\large\bfseries},
    computer/.style={rectangle, draw, minimum width=2.2cm, minimum height=1.8cm, align=center},
    arrow/.style={->, >=stealth, very thick}
]

% Top row: Training set -> ML Computer -> Program
\node[box] (train) {Training set};
\node[computer, right=3cm of train] (ml1) {ML};
\node[box, right=3cm of ml1] (program1) {Program};

% Draw computer screen
\draw[thick] ([xshift=-0.6cm, yshift=0.3cm]ml1.center) rectangle ([xshift=0.6cm, yshift=0.7cm]ml1.center);
\draw[thick] ([xshift=-0.7cm, yshift=-0.7cm]ml1.center) -- ([xshift=0.7cm, yshift=-0.7cm]ml1.center);

% Arrows for top row
\draw[arrow] (train) -- (ml1);
\draw[arrow] (ml1) -- (program1);

% Middle row: Validation evaluation with feedback
\node[computer, below=3.5cm of ml1] (ml2) {};
\node[box, right=2.5cm of ml2] (val) {Validation set};

% Draw computer screen for ml2
\draw[thick] ([xshift=-0.6cm, yshift=0.3cm]ml2.center) rectangle ([xshift=0.6cm, yshift=0.7cm]ml2.center);
\draw[thick] ([xshift=-0.7cm, yshift=-0.7cm]ml2.center) -- ([xshift=0.7cm, yshift=-0.7cm]ml2.center);

% Arrow from Program to validation computer
\draw[arrow] (program1.south) -- (ml2.north);

% Bidirectional arrows between validation computer and validation set
\draw[arrow] (ml2.east) -- (val.west);
\draw[arrow] (val.west) -- (ml2.east);

% Thumbs down (red) - feedback to training
\node[left=1.5cm of train, yshift=-2.5cm, fill=red!70!black, circle, minimum size=1.2cm] (thumbdown) {\textcolor{white}{\Large\textbf{X}}};
\draw[->, very thick, red!70!black] (thumbdown.north) to[out=90, in=180] ([yshift=-0.3cm]train.west);
\draw[->, very thick, red!70!black] ([xshift=-0.5cm]ml2.west) to[out=180, in=0] (thumbdown.east);

% Thumbs up (green) - proceed to testing
\node[left=1.5cm of train, yshift=-5.5cm, fill=green!60!black, circle, minimum size=1.2cm] (thumbup) {\textcolor{white}{\huge$\checkmark$}};
\node[box, below=1.5cm of thumbup] (program2) {Program};

\draw[->, very thick, green!60!black] (thumbup.south) -- (program2.north);
\draw[->, very thick, green!60!black] ([xshift=-0.5cm]ml2.west) to[out=180, in=90] (thumbup.north);

% Bottom row: Test data -> Computer -> Check model performance
\node[box, right=1.5cm of program2] (test) {Test data};
\node[computer, right=2.5cm of test] (ml3) {};
\node[box, right=2.5cm of ml3] (check) {Check model\\performance};

% Draw computer screen for ml3
\draw[thick] ([xshift=-0.6cm, yshift=0.3cm]ml3.center) rectangle ([xshift=0.6cm, yshift=0.7cm]ml3.center);
\draw[thick] ([xshift=-0.7cm, yshift=-0.7cm]ml3.center) -- ([xshift=0.7cm, yshift=-0.7cm]ml3.center);

% Arrows for bottom row
\draw[arrow] (program2) -- (test);
\draw[arrow] (test) -- (ml3);
\draw[arrow] (ml3) -- (check);

\end{tikzpicture}
\caption{Machine Learning Workflow: Training set trains the model, validation set evaluates and provides feedback (thumbs down = retrain, thumbs up = proceed to testing), test set checks final performance}
\end{figure}

\subsubsection{The Three Datasets}

\begin{enumerate}
    \item \textbf{Training Set}: Used to train the machine learning algorithm
    \begin{itemize}
        \item The model learns patterns and relationships from this data
        \item Parameters are optimized to minimize training error
    \end{itemize}
    
    \item \textbf{Validation Set}: Used during the learning phase
    \begin{itemize}
        \item Also called the \textit{mini test set}
        \item Helps select the better performing algorithm among alternatives
        \item Used to tune hyperparameters of the model
        \item Provides feedback to adjust the model before final testing
    \end{itemize}
    
    \item \textbf{Test Set}: Used for final model evaluation
    \begin{itemize}
        \item Provides an unbiased assessment of model performance
        \item The model has never seen this data during training or validation
        \item Tells us how well the model will perform on new, real-world data
    \end{itemize}
\end{enumerate}

\subsection{The Importance of the Validation Set}

The validation set plays a crucial role in the machine learning pipeline and must be kept \textbf{separate from the training set}.

\subsubsection{Why Separate the Validation Set?}

The validation set is used for two main purposes:

\begin{enumerate}
    \item \textbf{Algorithm Selection}: To pick the better performing algorithm
    \begin{itemize}
        \item Compare different model architectures (e.g., decision trees vs. neural networks)
        \item Select the model that performs best on unseen data
    \end{itemize}
    
    \item \textbf{Hyperparameter Tuning}: To decide the hyperparameters of an algorithm
    \begin{itemize}
        \item Hyperparameters are configuration settings that are not learned from data
        \item Examples: learning rate, regularization strength ($\lambda$), number of layers, tree depth
        \item The validation set helps find the optimal hyperparameter values
    \end{itemize}
\end{enumerate}

\begin{remark}
\textbf{ATTENTION}: Splitting the datasets into training and validation sets can be done randomly to avoid bias. However, there are some \textbf{specific rules} to apply when performing this split to ensure reliable model evaluation.
\end{remark}

\subsection{Specific Rules for Dataset Splitting}

When splitting data into training, validation, and testing sets, you can have conflicting priorities. The key is to balance these priorities appropriately.

\begin{figure}[h]
\centering
\begin{tikzpicture}[
    node distance=0.8cm,
    box/.style={rectangle, draw, minimum width=10cm, minimum height=0.8cm, align=center}
]

% All Data
\node[box, fill=cyan!40] (all) {\textbf{All Data}};

% Three splits
\node[box, fill=green!30, below=of all, xshift=-3cm, minimum width=4.5cm] (train) {\textbf{Training}};
\node[box, fill=violet!30, below=of all, xshift=1cm, minimum width=2.5cm] (val) {\textbf{Validation}};
\node[box, fill=red!30, below=of all, xshift=4.5cm, minimum width=2cm] (test) {\textbf{Test}};

% Descriptions
\node[below=0.3cm of train, align=center, font=\small] {Models learn the task};
\node[below=0.3cm of val, align=center, font=\small] {Which model\\is the best?};
\node[below=0.3cm of test, align=center, font=\small] {How good\\is this model\\truly?};

% Arrows
\draw[->, thick] (all.south) -- (train.north);
\draw[->, thick] (all.south) -- (val.north);
\draw[->, thick] (all.south) -- (test.north);

\end{tikzpicture}
\caption{Dataset splitting: Training for learning, Validation for model selection, Test for final evaluation}
\end{figure}

\subsubsection{Priority 1: Accurate Error Estimation}

\textbf{Goal}: Estimate future error (i.e., validation/testing error) as accurately as possible.

\textbf{How to achieve this}: By making the \textbf{validation set as big as possible}.

\textit{(High confidence interval)}

\begin{itemize}
    \item A larger validation set provides more reliable error estimates
    \item Reduces variance in performance metrics
    \item Gives higher confidence that the measured performance reflects true generalization
\end{itemize}

\subsubsection{Priority 2: Accurate Classifier Learning}

\textbf{Goal}: Learn classifier as accurately as possible.

\textbf{How to achieve this}: By making the \textbf{training set as big as possible}.

\textit{(Better estimates, maybe better generalization)}

\begin{itemize}
    \item More training data allows the model to learn more robust patterns
    \item Reduces overfitting by providing diverse examples
    \item Generally leads to better generalization on unseen data
\end{itemize}

\subsubsection{The Critical Rule}

\begin{center}
\colorbox{red!20}{\parbox{0.9\textwidth}{
\centering
\textbf{CRITICAL REQUIREMENT}\\[0.3cm]
Training and validation/testing instances \textbf{CANNOT OVERLAP !!!!}
}}
\end{center}

\begin{itemize}
    \item If the same data appears in both training and validation/test sets, the evaluation is invalid
    \item The model would be tested on data it has already seen during training
    \item This leads to overly optimistic performance estimates that don't reflect real-world performance
    \item Data leakage between sets is one of the most common mistakes in machine learning
\end{itemize}

\begin{observation}
The split between training and validation/test sets represents a fundamental tradeoff:
\begin{itemize}
    \item Larger training set $\rightarrow$ Better model learning, but less reliable error estimation
    \item Larger validation/test set $\rightarrow$ More reliable error estimation, but potentially worse model learning
\end{itemize}

Common splits include 70-15-15, 80-10-10, or 60-20-20 (train-validation-test), depending on the total dataset size and specific requirements.
\end{observation}

\section{Cross Validation}

\subsection{When Do We Need Cross Validation?}

In cases where we don't have enough data to randomly split between training (e.g., 60\% of the total data), validation (e.g., 20\% of the total data), and testing (e.g., 20\% of the total data), we need to use \textbf{cross-validation}.

\begin{definition}[Cross Validation]
Cross-validation is a resampling technique that involves reusing a portion of the training set itself to validate performance by retraining not just one, but $N$ models.
\end{definition}

\subsection{Key Characteristics of Cross Validation}

\begin{itemize}
    \item \textbf{Multiple models}: Cross-validation involves training $N$ models instead of just one
    
    \item \textbf{Computational cost}: Performing cross-validation requires $N$ training sessions and can be more computationally demanding
    
    \item \textbf{Better use of limited data}: When data is scarce, cross-validation allows us to use the available data more efficiently for both training and validation
\end{itemize}

\subsection{Fundamental Rules of Cross Validation}

Cross validation must respect certain constraints to ensure valid model evaluation:

\begin{enumerate}
    \item \textbf{No overlap}: Training and validation cannot overlap, but $n_{\text{train}} + n_{\text{validation}} = \text{constant}$
    
    \item \textbf{Each instance used once per fold}: At each fold (step), you use each instance only in one set, so there is no overlapping within a single fold
    
    \item \textbf{Every sample participates}: Every sample is in both training and testing but not at the same time
    \begin{itemize}
        \item This reduces the chances of getting a biased training set
        \item Ensures all data contributes to both training and validation across different folds
    \end{itemize}
\end{enumerate}

\subsection{K-Fold Cross Validation}

K-fold cross validation is the most common form of cross validation, where the dataset is divided into $k$ equal parts (folds).

\subsubsection{How K-Fold Cross Validation Works}

In the case of k-fold cross-validation:

\begin{enumerate}
    \item We split our dataset into $k$ groups (folds)
    
    \item We perform training $k$ times on $(N/k)(k-1)$ data points
    
    \item We measure performance on the last $N/k$ data points (the validation fold)
    
    \item Our cross-validation performance will be the \textbf{average of the performance of the $k$ tests}
\end{enumerate}

\begin{remark}
\textbf{Important constraint}: Training set and validation set cannot overlap, but the sum of training and validation data is a constant (the total dataset size).
\end{remark}

\subsubsection{Visual Representation of K-Fold Cross Validation}

\begin{figure}[h]
\centering
\begin{tikzpicture}[
    fold/.style={rectangle, draw, minimum width=1.5cm, minimum height=0.8cm},
    val/.style={fill=blue!40},
    train/.style={fill=gray!20}
]

% Labels
\node[anchor=east] at (-0.5, 0) {1st};
\node[anchor=east] at (-0.5, -1.2) {2nd};
\node[anchor=east] at (-0.5, -2.4) {3rd};
\node[anchor=east] at (-0.5, -3.6) {4th};
\node[anchor=east] at (-0.5, -4.8) {5th};

\node[anchor=south, font=\small] at (1.5, 1) {Validation Fold};
\node[anchor=south, font=\small] at (5.5, 1) {Training Fold};

% Fold 1
\node[fold, val] at (0, 0) {};
\node[fold, train] at (1.5, 0) {};
\node[fold, train] at (3, 0) {};
\node[fold, train] at (4.5, 0) {};
\node[fold, train] at (6, 0) {};

% Fold 2
\node[fold, train] at (0, -1.2) {};
\node[fold, val] at (1.5, -1.2) {};
\node[fold, train] at (3, -1.2) {};
\node[fold, train] at (4.5, -1.2) {};
\node[fold, train] at (6, -1.2) {};

% Fold 3
\node[fold, train] at (0, -2.4) {};
\node[fold, train] at (1.5, -2.4) {};
\node[fold, val] at (3, -2.4) {};
\node[fold, train] at (4.5, -2.4) {};
\node[fold, train] at (6, -2.4) {};

% Fold 4
\node[fold, train] at (0, -3.6) {};
\node[fold, train] at (1.5, -3.6) {};
\node[fold, train] at (3, -3.6) {};
\node[fold, val] at (4.5, -3.6) {};
\node[fold, train] at (6, -3.6) {};

% Fold 5
\node[fold, train] at (0, -4.8) {};
\node[fold, train] at (1.5, -4.8) {};
\node[fold, train] at (3, -4.8) {};
\node[fold, train] at (4.5, -4.8) {};
\node[fold, val] at (6, -4.8) {};

% Y-axis label
\node[rotate=90, anchor=south] at (-2, -2.4) {K iterations (K-Folds)};

\end{tikzpicture}
\caption{K-Fold Cross Validation: Each row represents one iteration where a different fold serves as validation (blue) while the rest serve as training (gray)}
\end{figure}

\subsection{Example: 5-Fold Cross Validation}

Let's examine a concrete example with 5 folds:

\begin{itemize}
    \item \textbf{Step 1}: Randomly split the data into 5 folds
    \item \textbf{Step 2}: Test on each fold while training on 4 other folds (80\% train, 20\% test)
    \item \textbf{Step 3}: Average the results over 5 folds
\end{itemize}

\begin{figure}[h]
\centering
\begin{tikzpicture}[
    fold/.style={rectangle, draw, minimum width=1.2cm, minimum height=0.7cm},
    val/.style={fill=violet!40},
    train/.style={fill=cyan!40}
]

% Iteration labels
\node[anchor=east, font=\small] at (-0.5, 0) {1st};
\node[anchor=east, font=\small] at (-0.5, -1.2) {2nd};
\node[anchor=east, font=\small] at (-0.5, -2.4) {3rd};
\node[anchor=east, font=\small] at (-0.5, -3.6) {4th};
\node[anchor=east, font=\small] at (-0.5, -4.8) {5th};

% Fold 1
\node[fold, val] at (0, 0) {};
\node[fold, train] at (1.3, 0) {};
\node[fold, train] at (2.6, 0) {};
\node[fold, train] at (3.9, 0) {};
\node[fold, train] at (5.2, 0) {};
\node[anchor=west, font=\small] at (6.5, 0) {Performance 1};

% Fold 2
\node[fold, train] at (0, -1.2) {};
\node[fold, val] at (1.3, -1.2) {};
\node[fold, train] at (2.6, -1.2) {};
\node[fold, train] at (3.9, -1.2) {};
\node[fold, train] at (5.2, -1.2) {};
\node[anchor=west, font=\small] at (6.5, -1.2) {Performance 2};

% Fold 3
\node[fold, train] at (0, -2.4) {};
\node[fold, train] at (1.3, -2.4) {};
\node[fold, val] at (2.6, -2.4) {};
\node[fold, train] at (3.9, -2.4) {};
\node[fold, train] at (5.2, -2.4) {};
\node[anchor=west, font=\small] at (6.5, -2.4) {Performance 3};

% Fold 4
\node[fold, train] at (0, -3.6) {};
\node[fold, train] at (1.3, -3.6) {};
\node[fold, train] at (2.6, -3.6) {};
\node[fold, val] at (3.9, -3.6) {};
\node[fold, train] at (5.2, -3.6) {};
\node[anchor=west, font=\small] at (6.5, -3.6) {Performance 4};

% Fold 5
\node[fold, train] at (0, -4.8) {};
\node[fold, train] at (1.3, -4.8) {};
\node[fold, train] at (2.6, -4.8) {};
\node[fold, train] at (3.9, -4.8) {};
\node[fold, val] at (5.2, -4.8) {};
\node[anchor=west, font=\small] at (6.5, -4.8) {Performance 5};

% Arrow and final result
\draw[->, thick] (9, -1.2) -- (9, -3.6);
\node[anchor=west, font=\small, align=left] at (9.5, -2.4) {Average\\Performance\\of all the folds};

% Legend
\node[fold, val, minimum width=0.8cm] at (1, -6.2) {};
\node[anchor=west, font=\small] at (1.6, -6.2) {Validation Fold};
\node[fold, train, minimum width=0.8cm] at (4.5, -6.2) {};
\node[anchor=west, font=\small] at (5.1, -6.2) {Training Fold};

\end{tikzpicture}
\caption{5-Fold Cross Validation Example: Each fold serves as validation once, and the final performance is the average of all 5 iterations}
\end{figure}

\subsubsection{The Process in Detail}

For each of the 5 iterations:

\begin{enumerate}
    \item \textbf{Iteration 1}: Use fold 1 as validation, folds 2-5 as training $\rightarrow$ Get Performance 1
    \item \textbf{Iteration 2}: Use fold 2 as validation, folds 1,3-5 as training $\rightarrow$ Get Performance 2
    \item \textbf{Iteration 3}: Use fold 3 as validation, folds 1-2,4-5 as training $\rightarrow$ Get Performance 3
    \item \textbf{Iteration 4}: Use fold 4 as validation, folds 1-3,5 as training $\rightarrow$ Get Performance 4
    \item \textbf{Iteration 5}: Use fold 5 as validation, folds 1-4 as training $\rightarrow$ Get Performance 5
\end{enumerate}

\textbf{Final Cross-Validation Performance}:
\[
\text{CV Performance} = \frac{1}{5} \sum_{i=1}^{5} \text{Performance}_i
\]

\subsection{Advantages of Cross Validation}

\begin{itemize}
    \item \textbf{Better use of limited data}: Every data point is used for both training and validation
    
    \item \textbf{More reliable performance estimates}: Averaging over multiple folds reduces variance in the performance estimate
    
    \item \textbf{Reduces bias}: Every sample participates in validation, reducing the risk of a biased validation set
    
    \item \textbf{Detects overfitting}: If performance varies significantly across folds, it may indicate overfitting or data quality issues
\end{itemize}

\subsection{Disadvantages of Cross Validation}

\begin{itemize}
    \item \textbf{Computationally expensive}: Requires training $k$ models instead of one
    
    \item \textbf{Time-consuming}: Training time is multiplied by the number of folds
    
    \item \textbf{Not suitable for very large datasets}: When data is abundant, a simple train-validation-test split is more efficient
\end{itemize}

\begin{observation}
The choice of $k$ (number of folds) involves a tradeoff:
\begin{itemize}
    \item \textbf{Larger $k$}: More accurate performance estimate, but more computationally expensive
    \item \textbf{Smaller $k$}: Faster computation, but less reliable performance estimate
    \item Common choices: $k = 5$ or $k = 10$
    \item Extreme case: $k = N$ (Leave-One-Out Cross Validation) uses each single sample as validation once
\end{itemize}
\end{observation}

\section{Leave-One-Out Cross Validation}

\subsection{What is Leave-One-Out Cross Validation?}

Leave-One-Out Cross Validation (LOOCV) is an extreme case of k-fold cross validation that is particularly useful when we have very few data points.

\begin{definition}[Leave-One-Out Cross Validation]
LOOCV is $n$-fold cross validation where $n$ equals the total number of samples. In each iteration, we train on all $(n-1)$ samples and test on exactly 1 instance.
\end{definition}

\subsection{How LOOCV Works}

The process involves:

\begin{enumerate}
    \item Take the entire dataset of $n$ samples
    \item For each sample $i$ (where $i = 1, 2, \ldots, n$):
    \begin{itemize}
        \item Use sample $i$ as the validation set (1 sample)
        \item Use all other samples $(n-1)$ as the training set
        \item Train the model and evaluate on the single validation sample
    \end{itemize}
    \item Repeat this process $n$ times (once for each sample)
    \item Average the performance across all $n$ experiments
\end{enumerate}

\begin{figure}[h]
\centering
\begin{tikzpicture}[
    box/.style={rectangle, draw, minimum width=7cm, minimum height=0.6cm},
    val/.style={fill=gray!60},
    train/.style={fill=white}
]

% Title
\node[anchor=south] at (3.5, 3.5) {\small Total number of examples};
\draw[<->] (0, 3.3) -- (7, 3.3);

% Experiment 1
\node[anchor=east, font=\small] at (-0.3, 2.5) {Experiment 1};
\node[box, train] at (3.5, 2.5) {};
\node[rectangle, draw, fill=gray!60, minimum width=0.3cm, minimum height=0.6cm] at (0.15, 2.5) {};

% Experiment 2
\node[anchor=east, font=\small] at (-0.3, 1.7) {Experiment 2};
\node[box, train] at (3.5, 1.7) {};
\node[rectangle, draw, fill=gray!60, minimum width=0.3cm, minimum height=0.6cm] at (0.45, 1.7) {};

% Experiment 3
\node[anchor=east, font=\small] at (-0.3, 0.9) {Experiment 3};
\node[box, train] at (3.5, 0.9) {};
\node[rectangle, draw, fill=gray!60, minimum width=0.3cm, minimum height=0.6cm] at (0.75, 0.9) {};

% Dots
\node at (3.5, 0.3) {$\vdots$};

% Experiment N
\node[anchor=east, font=\small] at (-0.3, -0.3) {Experiment N};
\node[box, train] at (3.5, -0.3) {};
\node[rectangle, draw, fill=gray!60, minimum width=0.3cm, minimum height=0.6cm] at (6.85, -0.3) {};

% Arrow pointing to validation
\draw[->, thick] (7.5, -0.3) -- (7.2, -0.3);
\node[anchor=west, font=\small] at (7.6, -0.3) {Single validation example};

\end{tikzpicture}
\caption{Leave-One-Out Cross Validation: Each experiment uses exactly one sample for validation (gray) and all others for training (white)}
\end{figure}

\subsection{Pros and Cons of LOOCV}

\subsubsection{Advantages}

\begin{itemize}
    \item \textbf{Best possible classifier}: The model is learned from $n-1$ training examples, which is the maximum possible training data
    
    \item \textbf{Maximizes training data}: Each model uses almost all available data for training
    
    \item \textbf{Unbiased estimate}: Provides a nearly unbiased estimate of model performance
    
    \item \textbf{Deterministic}: No randomness in the splitting process (every sample is used exactly once as validation)
\end{itemize}

\subsubsection{Disadvantages}

\begin{itemize}
    \item \textbf{High computational cost}: Must re-learn everything for $n$ times, where $n$ is the total number of samples
    
    \item \textbf{Time-consuming}: For large datasets, LOOCV becomes impractical
    
    \item \textbf{High variance}: Performance estimates can have high variance, especially with small datasets
\end{itemize}

\begin{remark}
LOOCV is most appropriate when:
\begin{itemize}
    \item The dataset is very small (e.g., $n < 50$)
    \item Computational resources are available
    \item You need the most accurate performance estimate possible
\end{itemize}

For larger datasets, standard k-fold cross validation (with $k = 5$ or $k = 10$) is typically preferred due to its better balance between computational cost and performance estimation accuracy.
\end{remark}

\section{Stratification in Cross Validation}

\subsection{The Problem: Imbalanced Classes}

In many real-world datasets, class labels are not evenly distributed. For example, in a medical diagnosis dataset, healthy patients might vastly outnumber sick patients. When performing cross validation on such imbalanced datasets, random splitting can lead to folds with very different class distributions.

\subsection{What is Stratification?}

\begin{definition}[Stratification]
Stratification is a technique to keep class labels \textbf{balanced across training and validation sets} during cross validation. Instead of randomly dividing the dataset, we ensure that each fold maintains approximately the same proportion of samples for each class as the complete dataset.
\end{definition}

\subsection{How Stratified K-Fold Works}

The stratification process works as follows:

\begin{enumerate}
    \item \textbf{Divide by class}: Instead of taking the dataset and dividing it randomly into $K$ parts, take the dataset and divide it into individual classes
    
    \item \textbf{Split each class}: Then for each class, divide the instances into $K$ parts
    
    \item \textbf{Assemble folds}: Assemble the $i$-th part from all classes to make the $i$-th fold
    
    \item \textbf{Important}: It is still random! The splitting within each class is random, but the proportions are maintained
\end{enumerate}

\begin{figure}[h]
\centering
\begin{tikzpicture}

% Title
\node[font=\Large\bfseries] at (4, 7.8) {StratifiedKFold};

% Y-axis label
\node[rotate=90, font=\large] at (-1.3, 3.5) {Sample index};

% X-axis label
\node[font=\large] at (4, -1.2) {CV iteration};

% Column labels
\node[font=\normalsize] at (0.4, -0.5) {class};
\node[font=\normalsize] at (1.5, -0.5) {1};
\node[font=\normalsize] at (2.5, -0.5) {2};
\node[font=\normalsize] at (3.5, -0.5) {3};
\node[font=\normalsize] at (4.5, -0.5) {4};
\node[font=\normalsize] at (5.5, -0.5) {5};

% Class column (first column) - 3 colors
\fill[cyan!40] (0, 0) rectangle (0.8, 2.3);
\fill[orange!60] (0, 2.3) rectangle (0.8, 5);
\fill[brown!70] (0, 5) rectangle (0.8, 7);

% Fold 1
\fill[blue!60] (1.2, 0) rectangle (1.8, 3.1);
\fill[red!60] (1.2, 3.1) rectangle (1.8, 3.4);
\fill[blue!60] (1.2, 3.4) rectangle (1.8, 6);
\fill[red!60] (1.2, 6) rectangle (1.8, 7);

% Fold 2
\fill[blue!60] (2.2, 0) rectangle (2.8, 1);
\fill[red!60] (2.2, 1) rectangle (2.8, 1.2);
\fill[blue!60] (2.2, 1.2) rectangle (2.8, 2.3);
\fill[red!60] (2.2, 2.3) rectangle (2.8, 2.6);
\fill[blue!60] (2.2, 2.6) rectangle (2.8, 4);
\fill[red!60] (2.2, 4) rectangle (2.8, 5.1);
\fill[blue!60] (2.2, 5.1) rectangle (2.8, 7);

% Fold 3
\fill[blue!60] (3.2, 0) rectangle (3.8, 1.2);
\fill[red!60] (3.2, 1.2) rectangle (3.8, 1.4);
\fill[blue!60] (3.2, 1.4) rectangle (3.8, 3.4);
\fill[red!60] (3.2, 3.4) rectangle (3.8, 3.7);
\fill[blue!60] (3.2, 3.7) rectangle (3.8, 4.7);
\fill[red!60] (3.2, 4.7) rectangle (3.8, 6.1);
\fill[blue!60] (3.2, 6.1) rectangle (3.8, 7);

% Fold 4
\fill[blue!60] (4.2, 0) rectangle (4.8, 1.2);
\fill[red!60] (4.2, 1.2) rectangle (4.8, 1.4);
\fill[blue!60] (4.2, 1.4) rectangle (4.8, 2.4);
\fill[red!60] (4.2, 2.4) rectangle (4.8, 2.7);
\fill[blue!60] (4.2, 2.7) rectangle (4.8, 3.7);
\fill[red!60] (4.2, 3.7) rectangle (4.8, 4);
\fill[blue!60] (4.2, 4) rectangle (4.8, 5);
\fill[red!60] (4.2, 5) rectangle (4.8, 6.1);
\fill[blue!60] (4.2, 6.1) rectangle (4.8, 7);

% Fold 5
\fill[blue!60] (5.2, 0) rectangle (5.8, 1.2);
\fill[red!60] (5.2, 1.2) rectangle (5.8, 1.5);
\fill[blue!60] (5.2, 1.5) rectangle (5.8, 2.7);
\fill[red!60] (5.2, 2.7) rectangle (5.8, 3.1);
\fill[blue!60] (5.2, 3.1) rectangle (5.8, 3.7);
\fill[red!60] (5.2, 3.7) rectangle (5.8, 4.1);
\fill[blue!60] (5.2, 4.1) rectangle (5.8, 4.7);
\fill[red!60] (5.2, 4.7) rectangle (5.8, 6.1);
\fill[blue!60] (5.2, 6.1) rectangle (5.8, 7);

% Draw borders for all columns
\draw[thick] (0, 0) rectangle (0.8, 7);
\draw[thick] (1.2, 0) rectangle (1.8, 7);
\draw[thick] (2.2, 0) rectangle (2.8, 7);
\draw[thick] (3.2, 0) rectangle (3.8, 7);
\draw[thick] (4.2, 0) rectangle (4.8, 7);
\draw[thick] (5.2, 0) rectangle (5.8, 7);

% Legend
\node[rectangle, fill=red!60, draw, minimum width=0.6cm, minimum height=0.4cm] at (7.5, 6) {};
\node[anchor=west, font=\normalsize] at (8.1, 6) {Validation};

\node[rectangle, fill=blue!60, draw, minimum width=0.6cm, minimum height=0.4cm] at (7.5, 5.2) {};
\node[anchor=west, font=\normalsize] at (8.1, 5.2) {Training set};

\end{tikzpicture}
\caption{Stratified K-Fold: Each fold maintains the same class proportions across all folds}
\end{figure}

\subsection{Example: Stratified 5-Fold Cross Validation}

Let's consider a concrete scenario with an imbalanced dataset containing 300 samples, divided into three classes (A, B, and C), where Class A has more samples than the other two.

\subsubsection{Imbalanced Dataset}

\begin{itemize}
    \item \textbf{Class A}: 200 samples (66.7\%)
    \item \textbf{Class B}: 50 samples (16.7\%)
    \item \textbf{Class C}: 50 samples (16.7\%)
    \item \textbf{Total}: 300 samples
\end{itemize}

\subsubsection{Split the Dataset into Five Folds with Stratification}

Each fold will have 60 samples total, maintaining the class proportions:

\begin{itemize}
    \item \textbf{Fold 1}: 60 samples (Class A: 40, Class B: 10, Class C: 10)
    \item \textbf{Fold 2}: 60 samples (Class A: 40, Class B: 10, Class C: 10)
    \item \textbf{Fold 3}: 60 samples (Class A: 40, Class B: 10, Class C: 10)
    \item \textbf{Fold 4}: 60 samples (Class A: 40, Class B: 10, Class C: 10)
    \item \textbf{Fold 5}: 60 samples (Class A: 40, Class B: 10, Class C: 10)
\end{itemize}

Notice that each fold maintains the same 66.7\%-16.7\%-16.7\% class distribution as the original dataset.

\subsubsection{Train and Validate}

The cross-validation process proceeds as follows:

\begin{enumerate}
    \item \textbf{Iteration 1}: Train on Folds 1, 2, 3, and 4, Validate on Fold 5
    \item \textbf{Iteration 2}: Train on Folds 2, 3, 4, and 5, Validate on Fold 1
    \item \textbf{Iteration 3}: Train on Folds 1, 3, 4, and 5, Validate on Fold 2
    \item \textbf{Iteration 4}: Train on Folds 1, 2, 4, and 5, Validate on Fold 3
    \item \textbf{Iteration 5}: Train on Folds 1, 2, 3, and 5, Validate on Fold 4
\end{enumerate}

\subsubsection{Average the Performance Metrics}

\begin{itemize}
    \item Calculate performance metrics for each iteration
    \item Average these metrics to evaluate the model's performance robustly
\end{itemize}

\subsection{Why Stratification Matters}

\begin{itemize}
    \item \textbf{Prevents biased folds}: Without stratification, some folds might have very few (or no) examples of minority classes
    
    \item \textbf{Consistent evaluation}: Each fold represents the overall dataset distribution, leading to more reliable performance estimates
    
    \item \textbf{Better for imbalanced datasets}: Particularly important when dealing with class imbalance
    
    \item \textbf{Reduces variance}: Performance estimates have lower variance across folds
\end{itemize}

\begin{remark}
\textbf{When to use stratification}:
\begin{itemize}
    \item Always use stratification for classification problems with imbalanced classes
    \item It's generally a good practice even for balanced datasets
    \item Most machine learning libraries (e.g., scikit-learn) provide stratified cross-validation functions
    \item For regression problems, stratification is not applicable (since there are no discrete classes)
\end{itemize}
\end{remark}

\begin{observation}
Stratification ensures that:
\begin{itemize}
    \item Each fold is a good representative of the whole dataset
    \item Minority classes are adequately represented in both training and validation sets
    \item Performance metrics are more stable and reliable across folds
    \item The model sees a balanced distribution of classes during each training iteration
\end{itemize}
\end{observation}

\section{Evaluation Measures}

Once we have trained our models and performed cross-validation, we need to evaluate their performance. Different tasks require different evaluation metrics.

\subsection{Why Do We Need Evaluation Measures?}

Evaluation measures serve two primary purposes:

\begin{enumerate}
    \item \textbf{To decide if our model is performing well}: Assess whether the model meets the requirements for the task
    
    \item \textbf{To decide out of many models, which one performs better than the other}: Compare different models and select the best one for deployment
\end{enumerate}

\subsection{Evaluation Measures by Task Type}

Different machine learning tasks require different evaluation approaches:

\subsubsection{Classification}

\textbf{Question}: How often does our model classify something right/wrong?

Classification metrics measure the correctness of predictions, focusing on whether the predicted class matches the actual class.

\subsubsection{Regression}

\textbf{Question}: How close is our model to what we are trying to predict?

Regression metrics measure the distance or error between predicted continuous values and actual values.

\subsubsection{Clustering}

\textbf{Question}: How well does our model describe our data?

Clustering evaluation is generally very hard because there's often no ground truth to compare against. Metrics must assess the quality of discovered patterns.

\section{Evaluation Measures for Classification}

\subsection{Types of Classification Problems}

A classification model takes a set of features as input and produces one or more classes as output.

\begin{definition}[Classification Types]
Classification can be categorized into different types based on the number of classes:

\begin{itemize}
    \item \textbf{Binary Classification}: In the case of two classes (0/1) or one vs rest (e.g., apples vs other fruits)
    
    \item \textbf{Multi-class Classification}: In the case of having $N$ classes to choose from (e.g., cat, dog, wolf, etc.)
    
    \item \textbf{Multi-label Classification}: A problem where an example can belong to multiple classes simultaneously
\end{itemize}
\end{definition}

\subsection{The Confusion Matrix}

The confusion matrix is a fundamental tool for evaluating binary classification models. It provides a complete picture of how the model performs across different types of predictions.

\subsubsection{Confusion Matrix for Binary Classification}

\begin{figure}[h]
\centering
\begin{tikzpicture}[
    cell/.style={rectangle, draw, thick, minimum width=3.5cm, minimum height=2.2cm, align=center, font=\normalsize},
    header/.style={rectangle, draw, thick, fill=gray!50, align=center, font=\bfseries}
]

% Top header - Predicted Label
\node[font=\Large\bfseries] at (4, 4.5) {Predicted Label};

% Column headers (horizontal text)
\node[header, minimum width=3.5cm, minimum height=1cm] at (2, 3.5) {\textcolor{blue!70}{Positive}};
\node[header, minimum width=3.5cm, minimum height=1cm] at (6, 3.5) {\textcolor{green!60!black}{Negative}};

% Left header - Actual Label
\node[font=\Large\bfseries, rotate=90] at (-2.5, 0.85) {Actual Label};

% Row headers (vertical text)
\node[header, minimum width=2.2cm, minimum height=1.2cm, rotate=90] at (-0.5, 2) {\textcolor{blue!70}{Positive}};
\node[header, minimum width=2.2cm, minimum height=1.2cm, rotate=90] at (-0.5, -0.3) {\textcolor{green!60!black}{Negative}};

% Matrix cells (all same size)
\node[cell, fill=cyan!40] at (2, 2) {TRUE POSITIVE\\[0.3cm]\textbf{TP}};
\node[cell, fill=yellow!40] at (6, 2) {FALSE NEGATIVE\\[0.3cm]\textbf{FN}};
\node[cell, fill=pink!40] at (2, -0.3) {FALSE POSITIVE\\[0.3cm]\textbf{FP}};
\node[cell, fill=green!30] at (6, -0.3) {TRUE NEGATIVE\\[0.3cm]\textbf{TN}};

% Note box
\node[anchor=west, align=left, font=\small, text=red!70!black, draw, thick, rounded corners, fill=red!5, inner sep=8pt] at (8.5, 0.85) {
    \textit{We want to have}\\[0.1cm]
    \textit{large values in}\\[0.1cm]
    \textbf{TP and TN}\\[0.1cm]
    \textit{while smaller}\\[0.1cm]
    \textit{values in}\\[0.1cm]
    \textbf{FP and FN}
};

\end{tikzpicture}
\caption{Confusion Matrix for Binary Classification}
\end{figure}

\subsubsection{Understanding the Confusion Matrix}

The confusion matrix has four components:

\begin{itemize}
    \item \textbf{True Positive (TP)}: The model correctly predicted the positive class
    \begin{itemize}
        \item Actual label: Positive
        \item Predicted label: Positive
        \item \textcolor{blue!70}{Good prediction!}
    \end{itemize}
    
    \item \textbf{True Negative (TN)}: The model correctly predicted the negative class
    \begin{itemize}
        \item Actual label: Negative
        \item Predicted label: Negative
        \item \textcolor{green!60!black}{Good prediction!}
    \end{itemize}
    
    \item \textbf{False Positive (FP)}: The model incorrectly predicted positive (Type I error)
    \begin{itemize}
        \item Actual label: Negative
        \item Predicted label: Positive
        \item \textcolor{red}{Bad prediction!} (False alarm)
    \end{itemize}
    
    \item \textbf{False Negative (FN)}: The model incorrectly predicted negative (Type II error)
    \begin{itemize}
        \item Actual label: Positive
        \item Predicted label: Negative
        \item \textcolor{red}{Bad prediction!} (Missed detection)
    \end{itemize}
\end{itemize}

\begin{remark}
\textbf{Goal}: We want to maximize TP and TN (correct predictions) while minimizing FP and FN (incorrect predictions).
\end{remark}

\subsection{Classification Error and Accuracy}

From the confusion matrix, we can derive several important metrics.

\subsubsection{Classification Error}

\begin{definition}[Classification Error]
The classification error (also called misclassification rate) measures the proportion of incorrect predictions:

\[
\text{Classification Error} = \frac{FP + FN}{TP + TN + FP + FN}
\]

This represents the fraction of all predictions that were wrong.
\end{definition}

\subsubsection{Accuracy}

\begin{definition}[Accuracy]
Accuracy measures the proportion of correct predictions:

\[
\text{Accuracy} = 1 - \text{Error} = \frac{TP + TN}{TP + TN + FP + FN}
\]

This represents the fraction of all predictions that were correct.
\end{definition}

\subsection{The Problem with Accuracy: Imbalanced Classes}

While accuracy is a basic measure of "goodness" of a classifier, it can be very misleading when dealing with imbalanced classes.

\begin{center}
\colorbox{red!20}{\parbox{0.9\textwidth}{
\centering
\textbf{ATTENTION!}\\[0.2cm]
Accuracy is \textbf{very misleading} if we have \textbf{imbalanced classes}
}}
\end{center}

\subsubsection{Example: The Imbalanced Dataset Problem}

Consider a dataset with severe class imbalance:
\begin{itemize}
    \item 90 samples for class "NO"
    \item 10 samples for class "YES"
\end{itemize}

\textbf{Naive Strategy}: A trivial classifier that predicts "NO" for every sample would achieve:
\[
\text{Accuracy} = \frac{90}{100} = 90\%
\]

\textbf{The Problem}: Despite having 90\% accuracy, this model is completely useless! It cannot classify any data belonging to the "YES" class. The model has learned nothing meaningful about the data.

\begin{observation}
This example illustrates why accuracy alone is insufficient for evaluating classifiers, especially with imbalanced datasets. We need additional metrics that provide a more nuanced view of model performance across all classes.
\end{observation}

\begin{remark}
When dealing with imbalanced datasets, always consider:
\begin{itemize}
    \item Using stratified cross-validation
    \item Examining the confusion matrix in detail
    \item Using additional metrics beyond accuracy (precision, recall, F1-score)
    \item Considering class-specific performance
    \item Using techniques like oversampling, undersampling, or class weights
\end{itemize}
\end{remark}

\subsection{Misses and False Alarms}

Beyond accuracy, we need more specific metrics to understand model performance. These metrics focus on different aspects of classification errors.

\subsubsection{False Alarm Rate (False Positive Rate)}

\begin{definition}
\textbf{False Alarm Rate} = \textbf{False Positive Rate} = $\frac{\text{FP}}{\text{FP} + \text{TN}}$

This represents the \textit{percentage of negative samples we misclassified as positive}.
\end{definition}

\subsubsection{Miss Rate (False Negative Rate)}

\begin{definition}
\textbf{Miss Rate} = \textbf{False Negative Rate} = $\frac{\text{FN}}{\text{TP} + \text{FN}}$

This represents the \textit{percentage of positive samples we misclassified as negative}.
\end{definition}

\subsubsection{Recall (True Positive Rate)}

\begin{definition}
\textbf{Recall} = \textbf{True Positive Rate} = $\frac{\text{TP}}{\text{TP} + \text{FN}}$

This represents the \textit{percentage of positive samples we classified correctly}. 

It is also known as \textbf{Sensitivity} and can be expressed as: Recall = 1 - Miss Rate
\end{definition}

\begin{observation}
Recall answers the question: "Of all the actual positive cases, how many did we identify?"
\end{observation}

\subsubsection{Precision}

\begin{definition}
\textbf{Precision} = $\frac{\text{TP}}{\text{TP} + \text{FP}}$

This represents the \textit{percentage positive out of what we predicted was positive}.
\end{definition}

\begin{observation}
Precision answers the question: "Of all the cases we predicted as positive, how many were actually positive?"
\end{observation}

\subsection{Cost of the Task}

In practice, we typically do not use evaluation metrics like accuracy, recall, or precision alone. Instead, we declare a combination of them together based on the specific requirements of the task.

\subsubsection{The Need for a Single Evaluation Measure}

\begin{itemize}
    \item In order to optimize a learner automatically (i.e., during training), we need a \textbf{single evaluation measure}
    \item How do we decide that single metric?
    \begin{itemize}
        \item \textbf{Domain specific!} – It depends on the task
    \end{itemize}
\end{itemize}

\subsubsection{Weighting Errors by Cost}

Depending on the \textbf{cost of the task}, we can decide whether we take care more about having:
\begin{itemize}
    \item \textit{Less false positives}, or
    \item \textit{Less false negatives}
\end{itemize}

This is achieved through weighting them:

\begin{equation}
\text{Cost} = C_{\text{FP}} \cdot \text{FP} + C_{\text{FN}} \cdot \text{FN}
\end{equation}

where $C_{\text{FP}}$ and $C_{\text{FN}}$ are the costs associated with false positives and false negatives, respectively.

\begin{example}
\textbf{Medical Diagnosis (Cancer Detection)}:
\begin{itemize}
    \item False Negative (missing a cancer diagnosis): Very high cost – patient doesn't receive treatment
    \item False Positive (false alarm): Lower cost – patient undergoes additional tests
    \item Therefore: $C_{\text{FN}} \gg C_{\text{FP}}$, we prioritize high recall
\end{itemize}
\end{example}

\begin{example}
\textbf{Spam Email Filter}:
\begin{itemize}
    \item False Positive (marking important email as spam): High cost – user misses important messages
    \item False Negative (spam gets through): Lower cost – minor annoyance
    \item Therefore: $C_{\text{FP}} > C_{\text{FN}}$, we prioritize high precision
\end{itemize}
\end{example}

\subsection{F-Measure (F1 Score)}

The F-measure combines precision and recall into a single metric, providing a balance between the two.

\begin{definition}
The \textbf{F1 Score} is the harmonic mean of precision and recall:

\begin{equation}
F_1 = 2 \cdot \frac{\text{Precision} \times \text{Recall}}{\text{Precision} + \text{Recall}}
\end{equation}
\end{definition}

\begin{center}
\colorbox{green!15}{\parbox{0.8\textwidth}{
\centering
\textit{Harmonic mean of precision and recall}\\[0.1cm]
One of the most frequently used metrics in machine learning
}}
\end{center}

\subsubsection{Properties of F1 Score}

\begin{itemize}
    \item \textbf{(+)} If you do some mathematics, you will see that the F1 measure is sort of an accuracy without TN
    \item \textbf{(+)} Used frequently in information retrieval systems
    \item \textbf{(+)} Provides a single number that balances precision and recall
    \item \textbf{Range}: $F_1 \in [0, 1]$, where 1 is perfect and 0 is the worst
\end{itemize}

\begin{remark}
The harmonic mean (used in F1) is always less than or equal to the arithmetic mean. This means F1 score gives more weight to the lower value between precision and recall, ensuring both metrics need to be reasonably high for a good F1 score.
\end{remark}

\subsubsection{Generalized F-Measure}

The F1 score can be generalized to $F_\beta$ score, which allows adjusting the relative importance of precision and recall:

\begin{equation}
F_\beta = (1 + \beta^2) \cdot \frac{\text{Precision} \times \text{Recall}}{\beta^2 \cdot \text{Precision} + \text{Recall}}
\end{equation}

where:
\begin{itemize}
    \item $\beta < 1$: Emphasizes precision
    \item $\beta > 1$: Emphasizes recall
    \item $\beta = 1$: Equal weight (standard F1 score)
\end{itemize}

\subsection{Multiclass Classification -- Evaluation}

For multiclass classification problems (more than two classes), we need to extend our evaluation metrics.

\subsubsection{Multiclass Confusion Matrix}

\begin{itemize}
    \item $n_{ij}$ is the number of examples with actual label (true label) $y_i$ and predicted label $y_j$
    \item The \textbf{main diagonal} contains true positives for each class
    \item The \textbf{sum of off-diagonal elements along a column} is the number of false positives for the column label
    \item The \textbf{sum of off-diagonal elements along a row} is the number of false negatives for the row label
\end{itemize}

\subsubsection{Multiclass Metrics}

For each class $i$, we can compute:

\begin{equation}
FP_i = \sum_{j \neq i} n_{ji} \quad \quad FN_i = \sum_{j \neq i} n_{ij}
\end{equation}

\begin{equation}
\text{Precision}_i = \frac{n_{ii}}{n_{ii} + FP_i} \quad \quad \text{Recall}_i = \frac{n_{ii}}{n_{ii} + FN_i}
\end{equation}

\begin{equation}
\text{Multiclass Accuracy (MAcc)} = \frac{\sum_i n_{ii}}{\sum_i \sum_j n_{ij}}
\end{equation}

\subsubsection{Macro vs. Micro Averaging}

When reporting overall metrics for multiclass problems, we have two main approaches:

\begin{itemize}
    \item \textbf{Macro-averaging}: Compute metric for each class independently, then take the average
    \begin{equation}
    \text{Macro-Precision} = \frac{1}{C} \sum_{i=1}^{C} \text{Precision}_i
    \end{equation}
    This gives equal weight to each class regardless of size
    
    \item \textbf{Micro-averaging}: Aggregate the contributions of all classes to compute the average metric
    \begin{equation}
    \text{Micro-Precision} = \frac{\sum_i TP_i}{\sum_i (TP_i + FP_i)}
    \end{equation}
    This gives equal weight to each sample
\end{itemize}

\subsection{ROC Curve -- Receiver Operating Characteristic}

The ROC curve is a powerful tool for evaluating and comparing binary classifiers, especially when the decision threshold can be varied.

\subsubsection{What is the ROC Curve?}

The \textbf{ROC curve} is a graphical plot that illustrates the diagnostic ability of a binary classifier as its discrimination threshold is varied.

\begin{itemize}
    \item \textbf{X-axis}: False Positive Rate (FPR) = $\frac{FP}{FP + TN}$
    \item \textbf{Y-axis}: True Positive Rate (TPR) = Recall = $\frac{TP}{TP + FN}$
\end{itemize}

\subsubsection{Uses of ROC Curves}

The ROC curve allows us to:
\begin{itemize}
    \item \textbf{Compare different classifiers}: Different models can be plotted on the same ROC space
    \item \textbf{Estimate the false positive rate / negative rate depending on the thresholds} that we set to classify
    \item Visualize the trade-off between sensitivity (recall) and specificity
\end{itemize}

\subsubsection{Understanding the ROC Space}

\begin{itemize}
    \item \textbf{Perfect Classifier}: A point at (0, 1) – 100\% TPR, 0\% FPR
    \item \textbf{Random Classifier}: Lies on the diagonal line from (0,0) to (1,1)
    \item \textbf{Better Classifiers}: Curves that bow toward the upper left corner
    \item \textbf{Worse Classifiers}: Curves below the diagonal (worse than random guessing)
\end{itemize}

\subsubsection{Area Under the ROC Curve (AUC)}

\begin{definition}
The \textbf{Area Under the ROC Curve (AUC)} is a single scalar value that summarizes the performance of a classifier across all possible thresholds.

\begin{equation}
\text{AUC} \in [0, 1]
\end{equation}
\end{definition}

\begin{center}
\colorbox{orange!15}{\parbox{0.85\textwidth}{
\centering
\textbf{Goal:} Maximize the Area Under the ROC Curve (AUC)
}}
\end{center}

\subsubsection{Interpreting AUC Values}

\begin{itemize}
    \item \textbf{AUC = 1.0}: Perfect classifier
    \item \textbf{AUC = 0.9 - 1.0}: Excellent
    \item \textbf{AUC = 0.8 - 0.9}: Good
    \item \textbf{AUC = 0.7 - 0.8}: Fair
    \item \textbf{AUC = 0.6 - 0.7}: Poor
    \item \textbf{AUC = 0.5}: No better than random guessing
    \item \textbf{AUC < 0.5}: Worse than random (predictions are inverted)
\end{itemize}

\begin{observation}
The AUC can be interpreted as the probability that the classifier will rank a randomly chosen positive instance higher than a randomly chosen negative instance.
\end{observation}

\subsubsection{Advantages of ROC and AUC}

\begin{itemize}
    \item \textbf{Threshold-independent}: Evaluates the model across all possible thresholds
    \item \textbf{Class imbalance robust}: Unlike accuracy, AUC is less sensitive to class imbalance
    \item \textbf{Single metric}: Provides a single value for model comparison
    \item \textbf{Visual interpretation}: The ROC curve provides intuitive visualization of classifier performance
\end{itemize}

\end{document}
