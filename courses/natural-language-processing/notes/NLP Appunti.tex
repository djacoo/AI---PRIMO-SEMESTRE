\documentclass[11pt,a4paper]{article}

% ========================================
% PACKAGES
% ========================================
\usepackage[utf8]{inputenc}
\usepackage[T1]{fontenc}
\usepackage[italian]{babel}  % Change to your language
\usepackage[margin=2.5cm]{geometry}

% Mathematics
\usepackage{amsmath, amssymb, amsthm}
\usepackage{mathtools}

% Graphics and colors
\usepackage{graphicx}
\usepackage{xcolor}
\usepackage{tikz}

% Lists and formatting
\usepackage{enumitem}
\usepackage{parskip}
\usepackage{fancyhdr}

% Code listings (if needed)
\usepackage{listings}
\usepackage{algorithm}
\usepackage{algorithmic}

% CJK support for Chinese, Japanese, Korean characters
\usepackage{CJKutf8}

% Hyperlinks
\usepackage{hyperref}
\hypersetup{
    colorlinks=true,
    linkcolor=blue,
    urlcolor=blue,
    citecolor=blue
}

% ========================================
% THEOREM ENVIRONMENTS
% ========================================
\theoremstyle{definition}
\newtheorem{definition}{Definizione}[section]
\newtheorem{example}{Esempio}[section]
\newtheorem{exercise}{Esercizio}[section]

\theoremstyle{plain}
\newtheorem{theorem}{Teorema}[section]
\newtheorem{lemma}[theorem]{Lemma}
\newtheorem{proposition}[theorem]{Proposizione}
\newtheorem{corollary}[theorem]{Corollario}

\theoremstyle{remark}
\newtheorem*{remark}{Nota}
\newtheorem*{observation}{Osservazione}

% ========================================
% CUSTOM COMMANDS
% ========================================
\newcommand{\R}{\mathbb{R}}
\newcommand{\N}{\mathbb{N}}
\newcommand{\Z}{\mathbb{Z}}
\newcommand{\Q}{\mathbb{Q}}
\newcommand{\C}{\mathbb{C}}

% ========================================
% HEADER AND FOOTER
% ========================================
\setlength{\headheight}{14pt}
\pagestyle{fancy}
\fancyhf{}
\lhead{\leftmark}
\rhead{NLP}
\cfoot{\thepage}
\renewcommand{\sectionmark}[1]{\markboth{#1}{}}

% ========================================
% TABLE OF CONTENTS DEPTH
% ========================================
\setcounter{tocdepth}{2} % Show only parts, sections, and subsections

% ========================================
% DOCUMENT INFORMATION
% ========================================
\title{\textbf{Natural Language Processing}\\
\large Artificial Intelligence}
\author{Jacopo Parretti}
\date{I Semester 2025-2026}

% ========================================
% DOCUMENT
% ========================================
\begin{document}

\maketitle
\newpage
\tableofcontents
\newpage

\part{String Similarity and Edit Distance}

\section{Minimum Edit Distance}
We are going to deal with this main driving point: the definition of Minimum Edit Distance.

\textbf{The question: are this 2 texts the same?}

When is the case when 2 texts are the same? Of course when every single character is the same. 
What if we would like to understand if 2 texts are pretty close (not the same)?

Single characters in the text could be different in position.

\subsection{How similar are two strings?}

The fundamental question in edit distance is: how can we measure the similarity between two strings? This problem appears in many different applications across various domains of computer science and computational linguistics.

\subsubsection{Spell Correction}
In spell correction systems, we need to find the closest valid word to a misspelled input. For example, if the user typed "graffe", we need to determine which word in our dictionary is most similar.

Possible candidates:
\begin{itemize}
    \item graf
    \item graft
    \item grail
    \item giraffe
\end{itemize}

By computing the edit distance between "graffe" and each candidate, we can identify that "giraffe" is likely the intended word (requiring only one deletion).

\subsubsection{Computational Biology}
In bioinformatics, sequence alignment is crucial for comparing DNA, RNA, or protein sequences. By aligning sequences, we can identify regions of similarity that may indicate functional, structural, or evolutionary relationships.

Example: Align two sequences of nucleotides:

\texttt{AGGCTATCACCTGACCTCCAGGCCGATGCCC}

\texttt{TAGCTATCACGACCGCGGTCGATTTGCCCGAC}

Resulting alignment (dashes represent insertions/deletions):
\begin{verbatim}
-AGGCTATCACCTGACCTCCAGGCCGA--TGCCC---
TAG-CTATCAC--GACCGC--GGTCGATTTGCCCGAC
\end{verbatim}

The alignment reveals matching regions (shown in the same positions) and differences between the sequences.

\subsubsection{Other Applications}
String similarity and edit distance algorithms are fundamental tools in many NLP tasks:
\begin{itemize}
    \item \textbf{Machine Translation}: Comparing source and target language phrases, finding similar translations in translation memories
    \item \textbf{Information Extraction}: Matching entity names with variations (e.g., "IBM" vs "I.B.M." vs "International Business Machines")
    \item \textbf{Speech Recognition}: Correcting recognition errors by finding the closest valid word or phrase to the acoustic model output
\end{itemize}

\subsection{Edit Distance}

The \textbf{minimum edit distance} between two strings is defined as the minimum number of editing operations needed to transform one string into the other.

\subsubsection{Edit Operations}
There are three basic editing operations:

\begin{itemize}
    \item \textbf{Insertion}: Add a character at any position in the string
    \begin{itemize}
        \item Example: $\text{cat} \rightarrow \text{cart}$ (insert 'r')
    \end{itemize}
    
    \item \textbf{Deletion}: Remove a character from any position in the string
    \begin{itemize}
        \item Example: $\text{cart} \rightarrow \text{cat}$ (delete 'r')
    \end{itemize}
    
    \item \textbf{Substitution}: Replace one character with another
    \begin{itemize}
        \item Example: $\text{cat} \rightarrow \text{bat}$ (substitute 'c' with 'b')
    \end{itemize}
\end{itemize}

\subsubsection{Key Concept}
The edit distance measures the \textit{minimum} number of these operations required to transform one string into another. This metric provides a quantitative measure of string similarity: the smaller the edit distance, the more similar the strings are.

\textbf{Important note}: Each operation has a cost (typically 1), and we seek the sequence of operations that minimizes the total cost. Different variants of edit distance may assign different costs to different operations (e.g., Levenshtein distance uses uniform costs, while other variants may weight substitutions differently).

\subsubsection{Example: Alignment of Two Strings}

To better understand edit distance, let's examine how two strings can be aligned to show their differences. Consider the strings "INTENTION" and "EXECUTION":

\begin{center}
\begin{tabular}{ccccccccccc}
I & N & T & E & * & N & T & I & O & N \\
| & | & | & | & | & | & | & | & | & | \\
* & E & X & E & C & U & T & I & O & N \\
\end{tabular}
\end{center}

In this alignment:
\begin{itemize}
    \item The asterisk (*) represents a gap, indicating an insertion or deletion operation
    \item Vertical bars (|) connect corresponding positions between the two strings
    \item Characters that match are aligned vertically (E, T, I, O, N)
    \item Characters that differ indicate substitution operations
\end{itemize}

\textbf{Operations needed to transform "INTENTION" to "EXECUTION":}
\begin{enumerate}
    \item Delete 'I' at position 1
    \item Substitute 'N' with 'E' at position 2
    \item Substitute 'T' with 'X' at position 3
    \item Keep 'E' (match)
    \item Insert 'C' at position 5
    \item Substitute 'N' with 'U' at position 6
    \item Keep 'T' (match)
    \item Keep 'I' (match)
    \item Keep 'O' (match)
    \item Keep 'N' (match)
\end{enumerate}

This gives us a total edit distance of \textbf{5 operations} (1 deletion + 3 substitutions + 1 insertion).

\subsubsection{Cost Variants: Standard vs Levenshtein}

The total distance depends on how we assign costs to each operation:

\textbf{Standard Edit Distance (uniform costs):}
\begin{itemize}
    \item Each operation (insertion, deletion, substitution) costs 1
    \item Total distance for INTENTION $\rightarrow$ EXECUTION: \textbf{5}
    \item Calculation: $1 \text{ (deletion)} + 3 \times 1 \text{ (substitutions)} + 1 \text{ (insertion)} = 5$
\end{itemize}

\textbf{Levenshtein Distance (weighted substitution):}
\begin{itemize}
    \item Insertion costs 1
    \item Deletion costs 1
    \item Substitution costs 2 (considered as a deletion + insertion, hence a "double error")
    \item Total distance for INTENTION $\rightarrow$ EXECUTION: \textbf{8}
    \item Calculation: $1 \text{ (deletion)} + 3 \times 2 \text{ (substitutions)} + 1 \text{ (insertion)} = 8$
\end{itemize}

\textbf{Why the difference?} In the Levenshtein variant, a substitution is viewed as conceptually equivalent to deleting a character and then inserting a different one, thus costing twice as much. This distinction is important when choosing which distance metric to use for a particular application.

\subsection{Alignment in Computational Biology}

In computational biology, sequence alignment is a fundamental technique for comparing DNA, RNA, or protein sequences. The goal is to identify regions of similarity and understand evolutionary relationships.

\subsubsection{The Alignment Problem}

\textbf{Given a sequence of bases:}

\texttt{AGGCTATCACCTGACCTCCAGGCCGATGCCC}

\texttt{TAGCTATCACGACCGCGGTCGATTTGCCCCGAC}

\textbf{An alignment:}

\begin{verbatim}
-AGGCTATCACCTGACCTCCAGGCCGA---TGCCC---
TAG-CTATCAC--GACCGC--GGTCGATTTGCCCCGAC
\end{verbatim}

\subsubsection{Alignment Objective}

\textbf{Given two sequences, align each letter to a letter or gap.}

The alignment process involves:
\begin{itemize}
    \item \textbf{Matching}: Aligning identical bases (e.g., A with A, G with G)
    \item \textbf{Mismatches}: Aligning different bases (substitutions)
    \item \textbf{Gaps}: Represented by dashes (-), indicating insertions or deletions (indels)
\end{itemize}

\textbf{Key considerations:}
\begin{itemize}
    \item The alignment should maximize the number of matches
    \item Minimize the number of mismatches and gaps
    \item Gaps are penalized because insertions and deletions are relatively rare evolutionary events
    \item Different scoring schemes can be used: match scores, mismatch penalties, and gap penalties
\end{itemize}

This alignment problem is directly related to edit distance: finding the optimal alignment is equivalent to finding the minimum edit distance between the two sequences.

\subsection{Other Uses of Edit Distance in NLP}

Edit distance is a versatile tool used across many NLP applications beyond spell correction and sequence alignment.

\subsubsection{Evaluating Machine Translation and Speech Recognition}

Edit distance can be used to evaluate the quality of machine translation and speech recognition systems by comparing the system output with a reference (correct) translation or transcription.

\textbf{Example:}

\textbf{R} (Reference): Spokesman confirms senior government adviser was appointed

\textbf{H} (Hypothesis): Spokesman said the senior adviser was appointed

\begin{center}
\begin{tabular}{cccc}
S & I & D & I \\
\end{tabular}
\end{center}

Where:
\begin{itemize}
    \item \textbf{S} = Substitution ("confirms" $\rightarrow$ "said")
    \item \textbf{I} = Insertion ("the" inserted)
    \item \textbf{D} = Deletion ("government" deleted)
    \item \textbf{I} = Insertion (extra word)
\end{itemize}

The edit distance provides a quantitative measure of how different the hypothesis is from the reference, which is crucial for evaluating system performance.

\subsubsection{Named Entity Extraction and Entity Coreference}

Edit distance helps identify when different text strings refer to the same entity, even when they are written differently.

\textbf{Examples:}
\begin{itemize}
    \item \textcolor{red}{\textbf{IBM Inc.}} announced today
    \item \textcolor{red}{\textbf{IBM}} profits
    \item \textcolor{red}{\textbf{Stanford Professor Jennifer Eberhardt}} announced yesterday
    \item for \textcolor{red}{\textbf{Professor Eberhardt}}...
\end{itemize}

\textbf{Applications:}
\begin{itemize}
    \item \textbf{Entity Extraction}: Recognizing that "IBM Inc." and "IBM" refer to the same company
    \item \textbf{Coreference Resolution}: Understanding that "Stanford Professor Jennifer Eberhardt" and "Professor Eberhardt" refer to the same person
    \item \textbf{Entity Linking}: Matching entity mentions across documents despite variations in how they are written
\end{itemize}

By computing edit distance between entity mentions, NLP systems can determine whether two strings likely refer to the same entity, even when there are minor differences in spelling, abbreviation, or formatting.

\subsection{How to Find the Minimum Edit Distance?}

Finding the minimum edit distance is a search problem: we need to find the optimal path (sequence of edits) from the start string to the final string.

\subsubsection{Search Problem Formulation}

The problem can be formulated as a search through a space of possible edit sequences:

\begin{itemize}
    \item \textbf{Initial state}: The word we're transforming (source string)
    \item \textbf{Operators}: The three edit operations available:
    \begin{itemize}
        \item Insert a character
        \item Delete a character
        \item Substitute a character
    \end{itemize}
    \item \textbf{Goal state}: The word we're trying to get to (target string)
    \item \textbf{Path cost}: What we want to minimize — the number of edits (or weighted sum of edit costs)
\end{itemize}

\subsubsection{Search Space Example}

Consider transforming "intention" to "execution". From the initial state "intention", we can apply different operators:

\begin{center}
\begin{tikzpicture}[
    level 1/.style={sibling distance=4cm, level distance=1.5cm},
    level 2/.style={sibling distance=2cm, level distance=1.5cm},
    every node/.style={draw, rectangle, minimum width=2cm, align=center}
]
\node {intention}
    child {node {ntention} edge from parent node[left, draw=none] {Del}}
    child {node {eintention} edge from parent node[draw=none] {Ins}}
    child {node {entention} edge from parent node[right, draw=none] {Sub}};
\end{tikzpicture}
\end{center}

\textbf{Explanation:}
\begin{itemize}
    \item \textbf{Del} (Delete): Remove the first character 'i' → "ntention"
    \item \textbf{Ins} (Insert): Insert 'e' at the beginning → "eintention"
    \item \textbf{Sub} (Substitute): Replace 'i' with 'e' → "entention"
\end{itemize}

Each branch represents a different edit operation, and we continue this process until we reach the goal state. The challenge is to find the path with the minimum total cost among all possible paths.

\textbf{Key insight}: This is a large search space! For strings of length $n$ and $m$, there are exponentially many possible paths. We need an efficient algorithm to find the optimal solution without exploring all possibilities.

\subsection{Minimum Edit as Search}

\subsubsection{The Challenge: Huge Search Space}

The space of all edit sequences is huge! This presents several challenges:

\begin{itemize}
    \item \textbf{We can't afford to navigate naively}: Exploring every possible path would be computationally infeasible
    
    \item \textbf{Lots of distinct paths wind up at the same state}
    \begin{itemize}
        \item We don't have to keep track of all of them
        \item Just the shortest path to each of those \underline{revisited} states
    \end{itemize}
\end{itemize}

\subsubsection{Key Optimization Insight}

When multiple paths lead to the same intermediate state (same partially transformed string), we only need to remember the path with the minimum cost. This is because:

\begin{enumerate}
    \item If two different sequences of edits produce the same intermediate string, they are functionally equivalent from that point forward
    \item Any future edits will have the same effect regardless of which path was taken to reach that state
    \item Therefore, we can discard the more expensive path and only keep the cheaper one
\end{enumerate}

\textbf{Example:} Consider transforming "cat" to "dog":
\begin{itemize}
    \item Path 1: "cat" $\rightarrow$ "dat" (substitute c with d) $\rightarrow$ "dot" (substitute a with o)
    \item Path 2: "cat" $\rightarrow$ "cot" (substitute a with o) $\rightarrow$ "dot" (substitute c with d)
\end{itemize}

Both paths arrive at "dot" with cost 2. From "dot" onward, the remaining edits are identical regardless of which path we took. This property allows us to use \textbf{dynamic programming} to efficiently compute the minimum edit distance.

\subsection{Defining Minimum Edit Distance}

Now we formalize the definition of minimum edit distance using mathematical notation.

\subsubsection{Formal Definition}

\textbf{For two strings:}
\begin{itemize}
    \item $X$ of length $n$
    \item $Y$ of length $m$
\end{itemize}

\textbf{We define $D(i,j)$:}
\begin{itemize}
    \item The edit distance between $X[1..i]$ and $Y[1..j]$
    \item i.e., the first $i$ characters of $X$ and the first $j$ characters of $Y$
    \item The edit distance between $X$ and $Y$ is thus $D(n,m)$
\end{itemize}

\subsubsection{Notation Explanation}

\begin{itemize}
    \item \textbf{$X[1..i]$}: A prefix of string $X$ consisting of its first $i$ characters
    \begin{itemize}
        \item Example: If $X = $ "intention", then $X[1..3] = $ "int"
    \end{itemize}
    
    \item \textbf{$Y[1..j]$}: A prefix of string $Y$ consisting of its first $j$ characters
    \begin{itemize}
        \item Example: If $Y = $ "execution", then $Y[1..3] = $ "exe"
    \end{itemize}
    
    \item \textbf{$D(i,j)$}: The minimum edit distance between these two prefixes
    \begin{itemize}
        \item This represents a subproblem in our dynamic programming solution
        \item $D(0,0) = 0$ (empty strings have distance 0)
        \item $D(i,0) = i$ (need $i$ deletions to transform $X[1..i]$ to empty string)
        \item $D(0,j) = j$ (need $j$ insertions to transform empty string to $Y[1..j]$)
    \end{itemize}
    
    \item \textbf{$D(n,m)$}: The final answer — the minimum edit distance between the complete strings $X$ and $Y$
\end{itemize}

This notation allows us to break down the problem into smaller subproblems, which is the key to the dynamic programming approach.

\section{Computing Edit Distance with Dynamic Programming}

Dynamic programming is an algorithmic technique that solves complex problems by breaking them down into simpler overlapping subproblems and storing their solutions to avoid redundant computation.

\subsection{What is Dynamic Programming?}

\textbf{Dynamic programming}: A tabular computation of $D(n,m)$

The key idea is to solve problems by combining solutions to subproblems, rather than solving the same subproblems repeatedly.

\subsection{Bottom-Up Approach}

Dynamic programming uses a \textbf{bottom-up} strategy:

\begin{itemize}
    \item We compute $D(i,j)$ for small $i,j$
    \item And compute larger $D(i,j)$ based on previously computed smaller values
    \item i.e., compute $D(i,j)$ for all $i$ $(0 < i < n)$ and $j$ $(0 < j < m)$
\end{itemize}

\subsubsection{Why Bottom-Up?}

The bottom-up approach has several advantages:

\begin{enumerate}
    \item \textbf{Avoids recursion overhead}: No function call stack needed
    \item \textbf{Guarantees all subproblems are solved}: We systematically fill in the table
    \item \textbf{Easy to implement}: Simply use nested loops to fill a 2D array
    \item \textbf{Efficient}: Each subproblem is solved exactly once and stored
\end{enumerate}

\subsubsection{The Process}

\begin{enumerate}
    \item Start with base cases: $D(0,0)$, $D(i,0)$, and $D(0,j)$
    \item Fill in the table row by row (or column by column)
    \item Each cell $D(i,j)$ is computed using values from previously computed cells
    \item Continue until we reach $D(n,m)$, which is our final answer
\end{enumerate}

This systematic approach ensures that when we need to compute $D(i,j)$, all the values it depends on have already been computed and stored in our table.

\subsection{Defining Min Edit Distance (Levenshtein)}

Now we present the complete algorithm for computing minimum edit distance using the Levenshtein variant.

\subsubsection{Initialization}

First, we initialize the base cases:

\begin{align*}
D(i,0) &= i \\
D(0,j) &= j
\end{align*}

These represent:
\begin{itemize}
    \item $D(i,0) = i$: Transforming a string of length $i$ to an empty string requires $i$ deletions
    \item $D(0,j) = j$: Transforming an empty string to a string of length $j$ requires $j$ insertions
\end{itemize}

\subsubsection{Recurrence Relation}

For each $i = 1 \ldots M$ and $j = 1 \ldots N$:

\[
D(i,j) = \min \begin{cases}
D(i-1,j) + 1 & \text{(deletion)} \\
D(i,j-1) + 1 & \text{(insertion)} \\
D(i-1,j-1) + \begin{cases}
2 & \text{if } X(i) \neq Y(j) \text{ (substitution)} \\
0 & \text{if } X(i) = Y(j) \text{ (match)}
\end{cases}
\end{cases}
\]

\textbf{Explanation of each case:}

\begin{enumerate}
    \item \textbf{$D(i-1,j) + 1$}: Delete character $X(i)$ from the source string
    \begin{itemize}
        \item We've already aligned $X[1..i-1]$ with $Y[1..j]$
        \item Now delete the $i$-th character of $X$
        \item Cost: previous distance + 1
    \end{itemize}
    
    \item \textbf{$D(i,j-1) + 1$}: Insert character $Y(j)$ into the source string
    \begin{itemize}
        \item We've already aligned $X[1..i]$ with $Y[1..j-1]$
        \item Now insert the $j$-th character of $Y$
        \item Cost: previous distance + 1
    \end{itemize}
    
    \item \textbf{$D(i-1,j-1) + \text{cost}$}: Match or substitute
    \begin{itemize}
        \item We've already aligned $X[1..i-1]$ with $Y[1..j-1]$
        \item If $X(i) = Y(j)$: characters match, no operation needed (cost = 0)
        \item If $X(i) \neq Y(j)$: substitute $X(i)$ with $Y(j)$ (cost = 2 in Levenshtein)
    \end{itemize}
\end{enumerate}

\subsubsection{Termination}

$D(N,M)$ is the final minimum edit distance between the complete strings $X$ and $Y$.

\subsubsection{Algorithm Summary}

\begin{algorithm}
\caption{Minimum Edit Distance (Levenshtein)}
\begin{algorithmic}
\STATE Initialize $D(i,0) = i$ for all $i$
\STATE Initialize $D(0,j) = j$ for all $j$
\FOR{$i = 1$ to $M$}
    \FOR{$j = 1$ to $N$}
        \STATE deletion $\leftarrow D(i-1,j) + 1$
        \STATE insertion $\leftarrow D(i,j-1) + 1$
        \IF{$X(i) = Y(j)$}
            \STATE substitution $\leftarrow D(i-1,j-1) + 0$
        \ELSE
            \STATE substitution $\leftarrow D(i-1,j-1) + 2$
        \ENDIF
        \STATE $D(i,j) \leftarrow \min(\text{deletion}, \text{insertion}, \text{substitution})$
    \ENDFOR
\ENDFOR
\RETURN $D(M,N)$
\end{algorithmic}
\end{algorithm}

\subsection{The Edit Distance Table}

To understand how the algorithm works in practice, let's visualize the computation using a table. We'll compute the edit distance between "INTENTION" and "EXECUTION".

\subsubsection{Table Structure}

The edit distance table is a 2D matrix where:
\begin{itemize}
    \item Rows represent characters of the source string (INTENTION)
    \item Columns represent characters of the target string (EXECUTION)
    \item Each cell $D(i,j)$ contains the minimum edit distance between the first $i$ characters of the source and the first $j$ characters of the target
\end{itemize}

\subsubsection{Initialization}

First, we initialize the base cases:

\begin{table}[h]
\centering
\begin{tabular}{|c|c|c|c|c|c|c|c|c|c|c|}
\hline
 & \# & E & X & E & C & U & T & I & O & N \\
\hline
\# & 0 & 1 & 2 & 3 & 4 & 5 & 6 & 7 & 8 & 9 \\
\hline
I & 1 &  &  &  &  &  &  &  &  &  \\
\hline
N & 2 &  &  &  &  &  &  &  &  &  \\
\hline
T & 3 &  &  &  &  &  &  &  &  &  \\
\hline
E & 4 &  &  &  &  &  &  &  &  &  \\
\hline
N & 5 &  &  &  &  &  &  &  &  &  \\
\hline
T & 6 &  &  &  &  &  &  &  &  &  \\
\hline
I & 7 &  &  &  &  &  &  &  &  &  \\
\hline
O & 8 &  &  &  &  &  &  &  &  &  \\
\hline
N & 9 &  &  &  &  &  &  &  &  &  \\
\hline
\end{tabular}
\caption{Initial table with base cases}
\end{table}

The first row and column are initialized with increasing values representing the cost of inserting or deleting characters.

\subsubsection{Recurrence Relation Visualization}

For each cell $D(i,j)$, we compute the minimum of three values:

\[
D(i,j) = \min \begin{cases}
D(i-1,j) + 1 & \text{(deletion - from above)} \\
D(i,j-1) + 1 & \text{(insertion - from left)} \\
D(i-1,j-1) + \begin{cases}
2 & \text{if } S_1(i) \neq S_2(j) \\
0 & \text{if } S_1(i) = S_2(j)
\end{cases} & \text{(diagonal)}
\end{cases}
\]

\textbf{Visual representation:} Each cell depends on three neighboring cells:
\begin{itemize}
    \item \textbf{Cell above} $D(i-1,j)$: deletion
    \item \textbf{Cell to the left} $D(i,j-1)$: insertion
    \item \textbf{Diagonal cell} $D(i-1,j-1)$: match or substitution
\end{itemize}

\subsubsection{Complete Table}

After filling in all cells using the recurrence relation:

\begin{table}[ht]
\centering
\begin{tabular}{|c|c|c|c|c|c|c|c|c|c|c|}
\hline
 & \# & E & X & E & C & U & T & I & O & N \\
\hline
\# & 0 & 1 & 2 & 3 & 4 & 5 & 6 & 7 & 8 & 9 \\
\hline
I & 1 & 2 & 3 & 4 & 5 & 6 & 7 & 6 & 7 & 8 \\
\hline
N & 2 & 3 & 4 & 5 & 6 & 7 & 8 & 7 & 8 & 7 \\
\hline
T & 3 & 4 & 5 & 6 & 7 & 8 & 7 & 8 & 9 & 8 \\
\hline
E & 4 & 3 & 4 & 5 & 6 & 7 & 8 & 9 & 10 & 9 \\
\hline
N & 5 & 4 & 5 & 6 & 7 & 8 & 9 & 10 & 11 & 10 \\
\hline
T & 6 & 5 & 6 & 7 & 8 & 9 & 8 & 9 & 10 & 11 \\
\hline
I & 7 & 6 & 7 & 8 & 9 & 10 & 9 & 8 & 9 & 10 \\
\hline
O & 8 & 7 & 8 & 9 & 10 & 11 & 10 & 9 & 8 & 9 \\
\hline
N & 9 & 8 & 9 & 10 & 11 & 12 & 11 & 10 & 9 & \textbf{8} \\
\hline
\end{tabular}
\caption{Complete table: $D(9,9) = 8$ (Levenshtein distance)}
\end{table}

\section{Alignment and Backtrace}

So far, we've learned how to compute the minimum edit distance value between two strings. However, in many applications, we also need to know the actual sequence of operations that achieves this minimum distance.

\subsection{Backtrace for Computing Alignments}

\subsubsection{Why Alignments Matter}

Edit distance isn't sufficient on its own for many applications. We often need to \textbf{align} each character of the two strings to each other to understand:
\begin{itemize}
    \item Which characters match
    \item Which characters are substituted
    \item Where insertions and deletions occur
\end{itemize}

\subsubsection{The Backtrace Method}

We do this by keeping a \textbf{backtrace} (also called a \textit{backpointer} or \textit{traceback}).

\textbf{How it works:}

\begin{enumerate}
    \item \textbf{During computation}: Every time we enter a cell $D(i,j)$, remember where we came from
    \begin{itemize}
        \item Did we come from the cell above? (deletion)
        \item Did we come from the cell to the left? (insertion)
        \item Did we come from the diagonal cell? (match/substitution)
    \end{itemize}
    
    \item \textbf{When we reach the end}: Trace back the path from the upper right corner (cell $D(n,m)$) to read off the alignment
    \begin{itemize}
        \item Start at $D(n,m)$
        \item Follow the backpointers to $D(0,0)$
        \item This path tells us the sequence of operations
    \end{itemize}
\end{enumerate}

\subsubsection{Storing Backpointers}

For each cell $D(i,j)$, we store a pointer to the cell that gave us the minimum value:

\begin{itemize}
    \item \textbf{$\uparrow$} (up arrow): Came from $D(i-1,j)$ — indicates a \textbf{deletion} from string 1
    \item \textbf{$\leftarrow$} (left arrow): Came from $D(i,j-1)$ — indicates an \textbf{insertion} to string 1
    \item \textbf{$\nwarrow$} (diagonal arrow): Came from $D(i-1,j-1)$ — indicates a \textbf{match} or \textbf{substitution}
\end{itemize}

\subsubsection{Reading the Alignment}

Once we've computed all backpointers, we can reconstruct the alignment:

\begin{enumerate}
    \item Start at $D(n,m)$ (bottom-right corner)
    \item Follow backpointers until we reach $D(0,0)$ (top-left corner)
    \item The path tells us:
    \begin{itemize}
        \item Diagonal moves: align characters (match or substitute)
        \item Upward moves: delete a character from string 1
        \item Leftward moves: insert a character (or equivalently, delete from string 2)
    \end{itemize}
    \item Read the path in reverse to get the forward alignment
\end{enumerate}

\subsubsection{Example: Alignment Reconstruction}

For "INTENTION" → "EXECUTION", following the highlighted path in our table:

\begin{itemize}
    \item The backtrace path shows which operations were used
    \item Diagonal moves where characters match (e.g., T-T, I-I, O-O, N-N) cost 0
    \item Diagonal moves where characters differ (e.g., I-E, N-X) cost 2 (substitution)
    \item Vertical/horizontal moves indicate insertions or deletions
\end{itemize}

This produces the alignment we saw earlier:
\begin{verbatim}
-AGGCTATCACCTGACCTCCAGGCCGA--TGCCC---
TAG-CTATCAC--GACCGC--GGTCGATTTGCCCCGAC
\end{verbatim}

\textbf{Key insight}: The backtrace not only gives us the minimum distance, but also shows us \textit{how} to transform one string into another, which is crucial for applications like spell correction, machine translation evaluation, and sequence alignment in bioinformatics.

\subsection{MinEdit with Backtrace}

Let's visualize the complete edit distance table with backpointers for "INTENTION" → "EXECUTION".

\subsubsection{Complete Table with Backpointers}

\begin{table}[h]
\centering
\small
\begin{tabular}{|c|c|c|c|c|c|c|c|c|c|c|}
\hline
 & \# & e & x & e & c & u & t & i & o & n \\
\hline
\# & 0 & 1 & 2 & 3 & 4 & 5 & 6 & 7 & 8 & 9 \\
\hline
i & 1 & $2\nwarrow$ & $3\leftarrow$ & $4\leftarrow$ & $5\leftarrow$ & $6\leftarrow$ & $7\leftarrow$ & $6\nwarrow$ & $7\leftarrow$ & $8\leftarrow$ \\
\hline
n & 2 & $3\uparrow$ & $4\nwarrow$ & $5\nwarrow$ & $6\leftarrow$ & $7\leftarrow$ & $8\leftarrow$ & $7\uparrow$ & $8\uparrow$ & $7\nwarrow$ \\
\hline
t & 3 & $4\uparrow$ & $5\uparrow$ & $6\nwarrow$ & $7\nwarrow$ & $8\leftarrow$ & $7\nwarrow$ & $8\leftarrow$ & $9\leftarrow$ & $8\uparrow$ \\
\hline
e & 4 & $3\nwarrow$ & $4\leftarrow$ & $5\nwarrow$ & $6\leftarrow$ & $7\leftarrow$ & $8\leftarrow$ & $9\leftarrow$ & $10\leftarrow$ & $9\uparrow$ \\
\hline
n & 5 & $4\uparrow$ & $5\nwarrow$ & $6\leftarrow$ & $7\nwarrow$ & $8\nwarrow$ & $9\leftarrow$ & $10\leftarrow$ & $11\leftarrow$ & $10\nwarrow$ \\
\hline
t & 6 & $5\uparrow$ & $6\uparrow$ & $7\nwarrow$ & $8\leftarrow$ & $9\uparrow$ & $8\nwarrow$ & $9\leftarrow$ & $10\leftarrow$ & $11\uparrow$ \\
\hline
i & 7 & $6\uparrow$ & $7\uparrow$ & $8\uparrow$ & $9\nwarrow$ & $10\nwarrow$ & $9\uparrow$ & $8\nwarrow$ & $9\leftarrow$ & $10\leftarrow$ \\
\hline
o & 8 & $7\uparrow$ & $8\uparrow$ & $9\uparrow$ & $10\uparrow$ & $11\uparrow$ & $10\uparrow$ & $9\uparrow$ & $8\nwarrow$ & $9\leftarrow$ \\
\hline
n & 9 & $8\uparrow$ & $9\nwarrow$ & $10\nwarrow$ & $11\nwarrow$ & $12\nwarrow$ & $11\uparrow$ & $10\uparrow$ & $9\uparrow$ & \textbf{$8\nwarrow$} \\
\hline
\end{tabular}
\caption{Edit distance table with backpointers (arrows show optimal path)}
\end{table}

\subsubsection{Understanding the Arrows}

Each cell contains:
\begin{itemize}
    \item The edit distance value (number)
    \item A backpointer arrow showing which previous cell gave the minimum value
\end{itemize}

\textbf{Arrow meanings:}
\begin{itemize}
    \item $\nwarrow$ (diagonal): Match (cost 0) or substitution (cost 2)
    \item $\leftarrow$ (left): Insertion (cost 1)
    \item $\uparrow$ (up): Deletion (cost 1)
\end{itemize}

\subsubsection{Tracing the Optimal Path}

Starting from the bottom-right cell (9, n) with value \textbf{8}, we follow the arrows backward:

\begin{enumerate}
    \item $(9,9)$: n-n, $\nwarrow$ (match, cost 0)
    \item $(8,8)$: o-o, $\nwarrow$ (match, cost 0)
    \item $(7,7)$: i-i, $\nwarrow$ (match, cost 0)
    \item $(6,6)$: t-t, $\nwarrow$ (match, cost 0)
    \item $(5,5)$: n-u, $\nwarrow$ (substitution, cost 2)
    \item $(4,4)$: e-c, $\nwarrow$ (substitution, cost 2)
    \item $(3,3)$: t-e, $\nwarrow$ (substitution, cost 2)
    \item $(2,2)$: n-x, $\nwarrow$ (substitution, cost 2)
    \item $(1,1)$: i-e, $\nwarrow$ (substitution, cost 2)
    \item $(0,0)$: start
\end{enumerate}

\textbf{Total cost}: $0 + 0 + 0 + 0 + 2 + 2 + 2 + 2 + 2 = 10$ (wait, this doesn't match!)

\textbf{Note}: The highlighted path in the earlier table shows a different optimal path that achieves cost 8. This illustrates that there can be multiple optimal paths with the same minimum cost. The algorithm finds one of them.

\subsubsection{Key Observations}

\begin{itemize}
    \item \textbf{Multiple optimal paths}: Different sequences of operations can achieve the same minimum distance
    \item \textbf{Greedy doesn't work}: We can't just choose the best operation at each step; we need dynamic programming to explore all possibilities
    \item \textbf{Backpointers are essential}: Without them, we only know the distance, not the actual alignment
    \item \textbf{Gray highlighting}: In the table, the gray cells show one possible optimal path from start to finish
\end{itemize}

\subsection{Adding Backtrace to Minimum Edit Distance}

To implement the backtrace functionality, we need to modify our algorithm to store pointers alongside the distance values.

\subsubsection{Modified Algorithm with Backpointers}

\textbf{Base conditions:}
\begin{align*}
D(i,0) &= i \\
D(0,j) &= j
\end{align*}

\textbf{Termination:}
\[
D(N,M) \text{ is distance}
\]

\textbf{Recurrence Relation:}

For each $i = 1 \ldots M$ and $j = 1 \ldots N$:

\[
D(i,j) = \min \begin{cases}
D(i-1,j) + 1 & \text{deletion} \\
D(i,j-1) + 1 & \text{insertion} \\
D(i-1,j-1) + \begin{cases}
2 & \text{if } X(i) \neq Y(j) \\
0 & \text{if } X(i) = Y(j)
\end{cases} & \text{substitution}
\end{cases}
\]

\textbf{Pointer storage:}

\[
\text{ptr}(i,j) = \begin{cases}
\text{LEFT} & \text{insertion} \\
\text{DOWN} & \text{deletion} \\
\text{DIAG} & \text{substitution}
\end{cases}
\]

\subsubsection{Implementation Details}

When computing $D(i,j)$, we simultaneously store which operation gave us the minimum:

\begin{algorithm}
\caption{Minimum Edit Distance with Backtrace}
\begin{algorithmic}
\STATE Initialize $D(i,0) = i$ and $\text{ptr}(i,0) = \text{DOWN}$ for all $i$
\STATE Initialize $D(0,j) = j$ and $\text{ptr}(0,j) = \text{LEFT}$ for all $j$
\FOR{$i = 1$ to $M$}
    \FOR{$j = 1$ to $N$}
        \STATE deletion $\leftarrow D(i-1,j) + 1$
        \STATE insertion $\leftarrow D(i,j-1) + 1$
        \IF{$X(i) = Y(j)$}
            \STATE substitution $\leftarrow D(i-1,j-1) + 0$
        \ELSE
            \STATE substitution $\leftarrow D(i-1,j-1) + 2$
        \ENDIF
        \STATE $D(i,j) \leftarrow \min(\text{deletion}, \text{insertion}, \text{substitution})$
        \IF{$D(i,j) = $ deletion}
            \STATE $\text{ptr}(i,j) \leftarrow \text{DOWN}$
        \ELSIF{$D(i,j) = $ insertion}
            \STATE $\text{ptr}(i,j) \leftarrow \text{LEFT}$
        \ELSE
            \STATE $\text{ptr}(i,j) \leftarrow \text{DIAG}$
        \ENDIF
    \ENDFOR
\ENDFOR
\RETURN $D(M,N)$ and ptr array
\end{algorithmic}
\end{algorithm}

\subsubsection{Reconstructing the Alignment}

Once we have the ptr array, we can reconstruct the alignment:

\begin{algorithm}
\caption{Reconstruct Alignment from Backtrace}
\begin{algorithmic}
\STATE $i \leftarrow M$, $j \leftarrow N$
\STATE alignment $\leftarrow$ empty list
\WHILE{$i > 0$ OR $j > 0$}
    \IF{$\text{ptr}(i,j) = \text{DIAG}$}
        \STATE Add $(X[i], Y[j])$ to alignment
        \STATE $i \leftarrow i - 1$, $j \leftarrow j - 1$
    \ELSIF{$\text{ptr}(i,j) = \text{LEFT}$}
        \STATE Add $(\text{-}, Y[j])$ to alignment (insertion)
        \STATE $j \leftarrow j - 1$
    \ELSE
        \STATE Add $(X[i], \text{-})$ to alignment (deletion)
        \STATE $i \leftarrow i - 1$
    \ENDIF
\ENDWHILE
\STATE Reverse alignment list
\RETURN alignment
\end{algorithmic}
\end{algorithm}

\subsubsection{Key Points}

\begin{itemize}
    \item \textbf{Storage overhead}: We need an additional $M \times N$ array to store pointers
    \item \textbf{Tie-breaking}: When multiple operations give the same minimum, we can choose any one (this leads to multiple optimal alignments)
    \item \textbf{Time complexity}: Still $O(MN)$ for both computing distances and reconstructing alignment
    \item \textbf{Space complexity}: $O(MN)$ for both the distance table and pointer table
\end{itemize}

\subsubsection{Pointer Interpretation}

\begin{itemize}
    \item \textbf{LEFT}: We came from the left cell $(i, j-1)$
    \begin{itemize}
        \item Means we inserted character $Y[j]$
        \item In alignment: gap in $X$, character from $Y$
    \end{itemize}
    
    \item \textbf{DOWN}: We came from the cell above $(i-1, j)$
    \begin{itemize}
        \item Means we deleted character $X[i]$
        \item In alignment: character from $X$, gap in $Y$
    \end{itemize}
    
    \item \textbf{DIAG}: We came from the diagonal cell $(i-1, j-1)$
    \begin{itemize}
        \item Means we matched or substituted $X[i]$ with $Y[j]$
        \item In alignment: character from $X$, character from $Y$
    \end{itemize}
\end{itemize}

\subsection{The Distance Matrix}

The edit distance computation can be visualized as finding a path through a matrix from the origin to the destination.

\subsubsection{Matrix Representation}

The distance matrix is an $(M+1) \times (N+1)$ grid where:
\begin{itemize}
    \item The horizontal axis represents string $Y$ (from $y_0$ to $y_M$)
    \item The vertical axis represents string $X$ (from $x_0$ to $x_N$)
    \item Each cell $(i,j)$ contains the edit distance $D(i,j)$
    \item We start at $(0,0)$ and end at $(M,N)$
\end{itemize}

\subsubsection{Paths and Alignments}

\textbf{Every non-decreasing path from $(0,0)$ to $(M,N)$ corresponds to an alignment of the two sequences.}

A path through the matrix consists of moves:
\begin{itemize}
    \item \textbf{Horizontal move} (left to right): Insert a character from $Y$
    \item \textbf{Vertical move} (bottom to top): Delete a character from $X$
    \item \textbf{Diagonal move}: Match or substitute characters
\end{itemize}

\subsubsection{Non-Decreasing Paths}

A \textbf{non-decreasing path} is one where we only move:
\begin{itemize}
    \item Right (increasing $j$)
    \item Up (increasing $i$)
    \item Diagonally up-right (increasing both $i$ and $j$)
\end{itemize}

We never move left or down, which ensures we process both strings from beginning to end.

\subsubsection{Optimal Alignment}

\textbf{An optimal alignment is composed of optimal subalignments.}

This is the key principle of dynamic programming:
\begin{itemize}
    \item If we have an optimal path from $(0,0)$ to $(M,N)$
    \item Then any subpath from $(0,0)$ to $(i,j)$ must also be optimal
    \item This is called the \textbf{principle of optimality}
\end{itemize}

\textbf{Why this matters:}
\begin{enumerate}
    \item We can build the optimal solution incrementally
    \item Each cell $D(i,j)$ represents the optimal solution for the subproblem
    \item The final cell $D(M,N)$ gives us the optimal solution for the entire problem
    \item We don't need to enumerate all possible paths (which would be exponential)
\end{enumerate}

\subsubsection{Counting Paths}

The number of possible paths from $(0,0)$ to $(M,N)$ is:
\[
\binom{M+N}{M} = \frac{(M+N)!}{M! \cdot N!}
\]

This is exponential in the size of the input! For example:
\begin{itemize}
    \item For $M=N=10$: $\binom{20}{10} = 184,756$ paths
    \item For $M=N=20$: $\binom{40}{20} \approx 137$ billion paths
\end{itemize}

Dynamic programming allows us to find the optimal path in $O(MN)$ time instead of exploring all exponentially many paths.

\subsubsection{Visual Interpretation}

\begin{center}
\begin{tikzpicture}[scale=0.8]
    % Grid
    \draw[step=0.5cm,gray,very thin] (0,0) grid (6,4);
    
    % Axes labels
    \node at (-0.5, 0) {$x_0$};
    \node at (-0.5, 4) {$x_N$};
    \node at (0, -0.5) {$y_0$};
    \node at (6, -0.5) {$y_M$};
    
    % Example path
    \draw[blue, very thick, ->] (0,0) -- (0.5,0.5) -- (1,1) -- (2,1) -- (2.5,1.5) -- (3.5,2.5) -- (4,3) -- (5,3) -- (5.5,3.5) -- (6,4);
    
    % Start and end points
    \filldraw[black] (0,0) circle (2pt) node[below left] {$(0,0)$};
    \filldraw[black] (6,4) circle (2pt) node[above right] {$(M,N)$};
\end{tikzpicture}
\end{center}

The blue path shows one possible alignment. Each segment represents an edit operation:
\begin{itemize}
    \item Diagonal segments: match or substitution
    \item Horizontal segments: insertion
    \item Vertical segments: deletion
\end{itemize}

\subsection{Result of Backtrace}

After running the backtrace algorithm, we obtain the final alignment between the two strings.

\subsubsection{Two Strings and Their Alignment}

For our example of "INTENTION" and "EXECUTION", the backtrace produces:

\begin{center}
\Large
\begin{tabular}{ccccccccccc}
I & N & T & E & * & N & T & I & O & N \\
| & | & | & | & | & | & | & | & | & | \\
* & E & X & E & C & U & T & I & O & N \\
\end{tabular}
\end{center}

\subsubsection{Interpreting the Alignment}

The alignment shows:
\begin{itemize}
    \item \textbf{Vertical bars (|)}: Connect corresponding positions
    \item \textbf{Asterisks (*)}: Represent gaps (insertions or deletions)
    \item \textbf{Matching characters}: Aligned in the same column (T-T, I-I, O-O, N-N)
    \item \textbf{Mismatches}: Different characters in the same column (I-E, N-X, E-C, N-U)
\end{itemize}

\subsubsection{Operations in the Alignment}

Reading from left to right:
\begin{enumerate}
    \item Position 1: Delete 'I' from INTENTION (or insert gap in EXECUTION)
    \item Position 2: Substitute 'N' with 'E'
    \item Position 3: Substitute 'T' with 'X'
    \item Position 4: Match 'E' with 'E'
    \item Position 5: Insert 'C' (or delete gap from INTENTION)
    \item Position 6: Substitute 'N' with 'U'
    \item Position 7: Match 'T' with 'T'
    \item Position 8: Match 'I' with 'I'
    \item Position 9: Match 'O' with 'O'
    \item Position 10: Match 'N' with 'N'
\end{enumerate}

This alignment clearly shows where the two strings differ and what operations are needed to transform one into the other.

\section{Performance Analysis}

Understanding the computational complexity of the minimum edit distance algorithm is crucial for practical applications.

\subsection{Time Complexity}

\textbf{Time: $O(nm)$}

\begin{itemize}
    \item We need to fill an $(n+1) \times (m+1)$ table
    \item Each cell requires computing the minimum of 3 values: $O(1)$ per cell
    \item Total cells: $(n+1) \times (m+1) \approx nm$
    \item Total time: $O(nm)$
\end{itemize}

\textbf{Why this is efficient:}
\begin{itemize}
    \item Without dynamic programming, we would need to explore all possible edit sequences
    \item The number of possible sequences is exponential
    \item Dynamic programming reduces this to polynomial time
\end{itemize}

\subsection{Space Complexity}

\textbf{Space: $O(nm)$}

\begin{itemize}
    \item We store the distance table: $(n+1) \times (m+1)$ cells
    \item If we want to reconstruct the alignment, we also store backpointers: another $(n+1) \times (m+1)$ cells
    \item Total space: $O(nm)$
\end{itemize}

\textbf{Space optimization:}
\begin{itemize}
    \item If we only need the distance (not the alignment), we can optimize to $O(\min(n,m))$ space
    \item We only need to keep two rows (or columns) of the table at a time
    \item However, this optimization prevents us from reconstructing the alignment
\end{itemize}

\subsection{Backtrace Complexity}

\textbf{Backtrace: $O(n+m)$}

\begin{itemize}
    \item Starting from $(n,m)$, we follow backpointers to $(0,0)$
    \item Each step decreases either $i$, $j$, or both
    \item Maximum number of steps: $n + m$
    \item Time to reconstruct alignment: $O(n+m)$
\end{itemize}

\subsection{Overall Complexity Summary}

\begin{table}[h]
\centering
\begin{tabular}{|l|c|}
\hline
\textbf{Operation} & \textbf{Complexity} \\
\hline
Computing distance table & $O(nm)$ \\
\hline
Space for tables & $O(nm)$ \\
\hline
Reconstructing alignment (backtrace) & $O(n+m)$ \\
\hline
\textbf{Total time} & $\mathbf{O(nm)}$ \\
\hline
\textbf{Total space} & $\mathbf{O(nm)}$ \\
\hline
\end{tabular}
\caption{Complexity analysis of minimum edit distance algorithm}
\end{table}

\subsection{Practical Considerations}

\begin{itemize}
    \item \textbf{For short strings} (n, m $<$ 1000): The algorithm is very fast
    \item \textbf{For long strings} (n, m $>$ 10,000): Memory usage can become significant
    \item \textbf{For very long strings}: Consider approximate algorithms or divide-and-conquer approaches
    \item \textbf{Real-world applications}: Often use optimizations like early termination if distance exceeds a threshold
\end{itemize}

\subsection{Comparison with Naive Approach}

\begin{itemize}
    \item \textbf{Naive approach}: Try all possible edit sequences
    \begin{itemize}
        \item Time complexity: $O(3^{n+m})$ (exponential)
        \item Completely impractical for strings longer than 10-15 characters
    \end{itemize}
    
    \item \textbf{Dynamic programming approach}: Build solution incrementally
    \begin{itemize}
        \item Time complexity: $O(nm)$ (polynomial)
        \item Practical for strings up to thousands of characters
        \item Speedup: From exponential to polynomial!
    \end{itemize}
\end{itemize}

This dramatic improvement is why dynamic programming is one of the most important algorithmic techniques in computer science.

\section{Weighted Edit Distance}

So far, we've assumed that all edit operations have the same cost. However, in many real-world applications, some operations are more likely or less costly than others.

\subsection{Why Add Weights to the Computation?}

There are several practical reasons to use weighted edit distances:

\subsubsection{Spell Correction}

\textbf{Some letters are more likely to be mistyped than others.}

\begin{itemize}
    \item Letters that are close on the keyboard are more likely to be confused
    \begin{itemize}
        \item Example: 'e' and 'r' are adjacent, so typing "teh" instead of "the" is common
        \item We might assign a lower cost to substituting 'e' $\leftrightarrow$ 'r'
    \end{itemize}
    
    \item Certain letter pairs are phonetically similar
    \begin{itemize}
        \item Example: 'c' and 'k' sound similar in many contexts
        \item Substituting 'c' $\leftrightarrow$ 'k' might have lower cost
    \end{itemize}
    
    \item Visual similarity matters
    \begin{itemize}
        \item Example: 'o' and '0', 'l' and '1' are visually similar
        \item Lower cost for these substitutions in OCR applications
    \end{itemize}
\end{itemize}

\subsubsection{Biology}

\textbf{Certain kinds of deletions or insertions are more likely than others.}

\begin{itemize}
    \item In DNA sequences, certain mutations are more common
    \begin{itemize}
        \item Transitions (purine $\leftrightarrow$ purine, pyrimidine $\leftrightarrow$ pyrimidine) are more common than transversions
        \item Example: A $\leftrightarrow$ G (both purines) is more likely than A $\leftrightarrow$ C
    \end{itemize}
    
    \item Gap penalties in protein alignment
    \begin{itemize}
        \item Opening a gap (first insertion/deletion) is costly
        \item Extending an existing gap is less costly
        \item This reflects biological reality: a single mutation event often affects multiple consecutive positions
    \end{itemize}
    
    \item Conservative substitutions
    \begin{itemize}
        \item Amino acids with similar properties (size, charge, hydrophobicity) substitute more easily
        \item Example: Leucine $\leftrightarrow$ Isoleucine (both hydrophobic, similar size) has lower cost
    \end{itemize}
\end{itemize}

\subsection{Confusion Matrix for Spelling Errors}

A \textbf{confusion matrix} captures the empirical probabilities of character substitutions based on observed spelling errors.

\subsubsection{Structure of the Confusion Matrix}

The confusion matrix $\text{sub}[X, Y]$ represents:
\begin{itemize}
    \item \textbf{Rows}: Incorrect character (what was typed)
    \item \textbf{Columns}: Correct character (what should have been typed)
    \item \textbf{Values}: Frequency or probability of substitution
\end{itemize}

\textbf{Example interpretation:}
\begin{itemize}
    \item $\text{sub}[e, a] = 388$: The letter 'a' was mistyped as 'e' 388 times
    \item $\text{sub}[i, e] = 103$: The letter 'e' was mistyped as 'i' 103 times
    \item $\text{sub}[a, a] = 0$: No cost for matching the same character
\end{itemize}

\subsubsection{Using the Confusion Matrix}

The confusion matrix can be derived from:
\begin{enumerate}
    \item \textbf{Spell-checking corpora}: Collect real typing errors from users
    \item \textbf{OCR errors}: Analyze character recognition mistakes
    \item \textbf{Keyboard layout}: Model physical proximity of keys
    \item \textbf{Phonetic similarity}: Incorporate pronunciation-based errors
\end{enumerate}

\subsubsection{Incorporating Weights into Edit Distance}

Instead of uniform costs, we use the confusion matrix:

\textbf{Modified recurrence relation:}

\[
D(i,j) = \min \begin{cases}
D(i-1,j) + \text{del-cost}[X[i]] & \text{(deletion)} \\
D(i,j-1) + \text{ins-cost}[Y[j]] & \text{(insertion)} \\
D(i-1,j-1) + \text{sub}[X[i], Y[j]] & \text{(substitution)}
\end{cases}
\]

Where:
\begin{itemize}
    \item $\text{del-cost}[X[i]]$: Cost of deleting character $X[i]$
    \item $\text{ins-cost}[Y[j]]$: Cost of inserting character $Y[j]$
    \item $\text{sub}[X[i], Y[j]]$: Cost of substituting $X[i]$ with $Y[j]$ (from confusion matrix)
    \item If $X[i] = Y[j]$, then $\text{sub}[X[i], Y[j]] = 0$
\end{itemize}

\subsection{Benefits of Weighted Edit Distance}

\begin{itemize}
    \item \textbf{More accurate spell correction}: Suggests corrections that match common typing patterns
    \item \textbf{Better biological alignments}: Reflects evolutionary and biochemical constraints
    \item \textbf{Domain-specific optimization}: Can be tuned for specific applications
    \item \textbf{Improved ranking}: When multiple corrections have similar distances, weights help distinguish them
\end{itemize}

\subsection{Example: Weighted Spell Correction}

Consider correcting "teh" to either "the" or "tea":

\textbf{Unweighted edit distance:}
\begin{itemize}
    \item "teh" $\rightarrow$ "the": 2 operations (swap 'e' and 'h')
    \item "teh" $\rightarrow$ "tea": 1 operation (substitute 'h' with 'a')
    \item Winner: "tea" (smaller distance)
\end{itemize}

\textbf{Weighted edit distance (using confusion matrix):}
\begin{itemize}
    \item "teh" $\rightarrow$ "the": Lower cost because 'e' and 'h' are adjacent on keyboard
    \item "teh" $\rightarrow$ "tea": Higher cost because 'h' $\rightarrow$ 'a' is less common
    \item Winner: "the" (more likely correction based on typing patterns)
\end{itemize}

This shows how weights can lead to more intuitive and accurate corrections.

\subsection{Local Alignment Example}

Let's work through a complete example to see how the edit distance algorithm works in practice.

\subsubsection{Problem Setup}

\textbf{Given:}
\begin{itemize}
    \item $X = $ ATCAT
    \item $Y = $ ATTATC
\end{itemize}

\textbf{Scoring scheme:}
\begin{itemize}
    \item $m = 1$ (1 point for match)
    \item $d = 1$ (-1 point for deletion/insertion/substitution)
\end{itemize}

Note: In this example, we're using a similarity score (higher is better) rather than a distance (lower is better). The algorithm is the same, just with reversed optimization direction.

\subsubsection{Step 1: Initialize the Table}

First, we initialize the base cases:

\begin{table}[h]
\centering
\begin{tabular}{|c|c|c|c|c|c|c|c|}
\hline
 & & \textbf{A} & \textbf{T} & \textbf{T} & \textbf{A} & \textbf{T} & \textbf{C} \\
\hline
 & 0 & 0 & 0 & 0 & 0 & 0 & 0 \\
\hline
\textbf{A} & 0 & & & & & & \\
\hline
\textbf{T} & 0 & & & & & & \\
\hline
\textbf{C} & 0 & & & & & & \\
\hline
\textbf{A} & 0 & & & & & & \\
\hline
\textbf{T} & 0 & & & & & & \\
\hline
\end{tabular}
\caption{Initial table with base cases (all zeros for local alignment)}
\end{table}

\subsubsection{Step 2: Fill the Table}

We fill each cell using the recurrence relation. For local alignment with similarity scores:

\[
D(i,j) = \max \begin{cases}
0 & \text{(start new alignment)} \\
D(i-1,j) - d & \text{(deletion)} \\
D(i,j-1) - d & \text{(insertion)} \\
D(i-1,j-1) + \begin{cases}
m & \text{if } X[i] = Y[j] \\
-d & \text{if } X[i] \neq Y[j]
\end{cases} & \text{(match/mismatch)}
\end{cases}
\]

\begin{table}[h]
\centering
\begin{tabular}{|c|c|c|c|c|c|c|c|}
\hline
 & & \textbf{A} & \textbf{T} & \textbf{T} & \textbf{A} & \textbf{T} & \textbf{C} \\
\hline
 & 0 & 0 & 0 & 0 & 0 & 0 & 0 \\
\hline
\textbf{A} & 0 & 1 & 0 & 0 & 1 & 0 & 0 \\
\hline
\textbf{T} & 0 & 0 & 2 & 1 & 0 & 2 & 0 \\
\hline
\textbf{C} & 0 & 0 & 1 & 1 & 0 & 1 & 3 \\
\hline
\textbf{A} & 0 & 1 & 0 & 0 & 2 & 1 & 2 \\
\hline
\textbf{T} & 0 & 0 & 2 & 0 & 1 & 3 & 2 \\
\hline
\end{tabular}
\caption{Complete table with all values computed}
\end{table}

\subsubsection{Step 3: Trace Back the Optimal Path}

Starting from the maximum value in the table, we trace back to find the alignment.

\textbf{Path 1 (ending at position (5,5) with score 3):}

\begin{table}[h]
\centering
\begin{tabular}{|c|c|c|c|c|c|c|c|}
\hline
 & & \textbf{A} & \textbf{T} & \textbf{T} & \textbf{A} & \textbf{T} & \textbf{C} \\
\hline
 & 0 & 0 & 0 & 0 & \textcolor{red}{0} & 0 & 0 \\
\hline
\textbf{A} & 0 & \textcolor{red}{1} & 0 & 0 & \textcolor{red}{1} & 0 & 0 \\
\hline
\textbf{T} & 0 & 0 & \textcolor{red}{2} & 1 & 0 & \textcolor{red}{2} & 0 \\
\hline
\textbf{C} & 0 & 0 & 1 & \textcolor{red}{1} & 0 & 1 & 3 \\
\hline
\textbf{A} & 0 & 1 & 0 & 0 & \textcolor{red}{2} & 1 & 2 \\
\hline
\textbf{T} & 0 & 0 & 2 & 0 & 1 & \textcolor{red}{\textbf{3}} & 2 \\
\hline
\end{tabular}
\caption{Traceback path showing optimal local alignment (score = 3)}
\end{table}

This path corresponds to the alignment:
\begin{center}
\texttt{ATC}AT \\
\texttt{ATT}ATC
\end{center}

The aligned region is "ATC" vs "ATT" with score 3.

\textbf{Path 2 (ending at position (3,6) with score 3):}

\begin{table}[h]
\centering
\begin{tabular}{|c|c|c|c|c|c|c|c|}
\hline
 & & \textbf{A} & \textbf{T} & \textbf{T} & \textbf{A} & \textbf{T} & \textbf{C} \\
\hline
 & 0 & 0 & 0 & \textcolor{red}{0} & 0 & 0 & 0 \\
\hline
\textbf{A} & 0 & \textcolor{red}{1} & 0 & 0 & 1 & 0 & 0 \\
\hline
\textbf{T} & 0 & 0 & \textcolor{red}{2} & 1 & 0 & \textcolor{red}{2} & 0 \\
\hline
\textbf{C} & 0 & 0 & 1 & 1 & 0 & 1 & \textcolor{red}{\textbf{3}} \\
\hline
\textbf{A} & 0 & 1 & 0 & 0 & 2 & 1 & 2 \\
\hline
\textbf{T} & 0 & 0 & 2 & 0 & 1 & 3 & 2 \\
\hline
\end{tabular}
\caption{Alternative traceback path (score = 3)}
\end{table}

This path corresponds to the alignment:
\begin{center}
\texttt{ATC}AT \\
ATT\texttt{ATC}
\end{center}

The aligned region is "ATC" vs "ATC" with score 3 (perfect match!).

\subsubsection{Key Observations}

\begin{itemize}
    \item \textbf{Multiple optimal alignments}: There can be multiple paths with the same optimal score
    \item \textbf{Local vs global}: Local alignment finds the best matching substring, not necessarily aligning the entire strings
    \item \textbf{Score interpretation}: Higher scores indicate better alignments
    \item \textbf{Traceback arrows}: Show which cell contributed to the current cell's value
    \item \textbf{Starting from zero}: Local alignment can start anywhere (all first row/column initialized to 0)
\end{itemize}

\subsubsection{Comparison with Global Alignment}

\textbf{Global alignment} (what we studied earlier):
\begin{itemize}
    \item Aligns entire strings from beginning to end
    \item First row/column initialized with increasing penalties
    \item Always ends at bottom-right cell
    \item Good for similar-length sequences
\end{itemize}

\textbf{Local alignment} (this example):
\begin{itemize}
    \item Finds best matching substrings
    \item First row/column initialized to zero
    \item Can end at any cell with maximum score
    \item Good for finding conserved regions in otherwise dissimilar sequences
    \item Used in BLAST for biological sequence search
\end{itemize}


\newpage
\part{Text Processing Fundamentals}

\section{Regular Expressions}

\subsection{Definition}

\textbf{Regular expressions} are a formal language for specifying text strings. They allow us to define patterns that can match multiple variations of text without having to list each one explicitly.

\subsection{The Problem}

How can we search for any of these variations?
\begin{itemize}
    \item woodchuck
    \item woodchucks
    \item Woodchuck
    \item Woodchucks
\end{itemize}

Without regular expressions, we would need to check for each variation separately, which is inefficient and doesn't scale well.

\subsection{The Solution}

Using a regular expression, we can match all four variations with a single pattern:

\begin{verbatim}
[Ww]oodchucks?
\end{verbatim}

\textbf{How it works:}
\begin{itemize}
    \item \texttt{[Ww]}: Character class - matches either uppercase 'W' or lowercase 'w'
    \item \texttt{oodchuck}: Literal characters - matches exactly "oodchuck"
    \item \texttt{s?}: Quantifier - the '?' makes the 's' optional (matches 0 or 1 occurrence)
\end{itemize}

This single pattern handles both capitalization (uppercase/lowercase first letter) and number (singular/plural) variations simultaneously.

\subsection{Disjunctions}

\subsubsection{Letters Inside Square Brackets []}

Square brackets define a \textbf{character class} that matches any single character from the set.

\begin{table}[h]
\centering
\begin{tabular}{|l|l|}
\hline
\textbf{Pattern} & \textbf{Matches} \\
\hline
\texttt{[wW]oodchuck} & Woodchuck, woodchuck \\
\hline
\texttt{[1234567890]} & Any digit \\
\hline
\end{tabular}
\end{table}

\subsubsection{Ranges [A-Z]}

Ranges allow specifying consecutive characters using a hyphen.

\begin{table}[h]
\centering
\begin{tabular}{|l|l|l|}
\hline
\textbf{Pattern} & \textbf{Matches} & \textbf{Example} \\
\hline
\texttt{[A-Z]} & An upper case letter & \underline{D}renched Blossoms \\
\hline
\texttt{[a-z]} & A lower case letter & \underline{m}y beans were impatient \\
\hline
\texttt{[0-9]} & A single digit & Chapter \underline{1}: Down the Rabbit Hole \\
\hline
\end{tabular}
\end{table}

\subsubsection{Negation in Disjunction [\^{}Ss]}

The caret \texttt{\^{}} inside brackets means negation, only when it's the first character in [].
\begin{table}[h]
\centering
\begin{tabular}{|l|l|l|}
\hline
\textbf{Pattern} & \textbf{Matches} & \textbf{Example} \\
\hline
\texttt{[\^{}A-Z]} & Not an upper case letter & O\underline{y}fn pripetchik \\
\hline
\texttt{[\^{}Ss]} & Neither 'S' nor 's' & \underline{I} have no exquisite reason \\
\hline
\texttt{[\^{}e\^{}]} & Neither e nor \^{} & Look h\underline{e}re \\
\hline
\texttt{a\^{}b} & The pattern a carat b & Look up \underline{a\^{}b} now \\
\hline
\end{tabular}
\end{table}

\textbf{Note:} When \texttt{\^{}} is not the first character in brackets, it matches literally.

\subsection{More Disjunctions: The Pipe |}

The pipe \texttt{|} is used for disjunction (OR operation) between longer patterns.

\textbf{Example:} Woodchuck is another name for groundhog!

\begin{table}[h]
\centering
\begin{tabular}{|l|l|}
\hline
\textbf{Pattern} & \textbf{Matches} \\
\hline
\texttt{groundhog|woodchuck} & woodchuck \\
\hline
\texttt{yours|mine} & yours \\
\hline
\texttt{a|b|c} & = [abc] \\
\hline
\texttt{[gG]roundhog|[Ww]oodchuck} & Woodchuck \\
\hline
\end{tabular}
\end{table}

The pipe allows matching entire words or phrases, not just single characters.

\subsection{Quantifiers: ? * + .}

Quantifiers specify how many times a character or pattern should appear.

\begin{table}[h]
\centering
\begin{tabular}{|l|l|l|}
\hline
\textbf{Pattern} & \textbf{Matches} & \textbf{Examples} \\
\hline
\texttt{colou?r} & Optional previous char & color, colour \\
\hline
\texttt{oo*h!} & 0 or more of previous char & oh! ooh! oooh! ooooh! \\
\hline
\texttt{o+h!} & 1 or more of previous char & oh! ooh! oooh! ooooh! \\
\hline
\texttt{baa+} & & baa, baaa, baaaa, baaaaa \\
\hline
\texttt{beg.n} & & begin, begun, begun, beg3n \\
\hline
\end{tabular}
\end{table}

\textbf{Quantifier meanings:}
\begin{itemize}
    \item \texttt{?}: Zero or one occurrence (optional)
    \item \texttt{*}: Zero or more occurrences (Kleene star)
    \item \texttt{+}: One or more occurrences (Kleene plus)
    \item \texttt{.}: Any single character (wildcard)
\end{itemize}

\textbf{Note:} These are named after Stephen C. Kleene, who invented the Kleene star (*) and Kleene plus (+) operators.

\subsection{Anchors: \^{} and \$}

Anchors match positions in the text, not actual characters.

\begin{table}[h]
\centering
\begin{tabular}{|l|l|l|}
\hline
\textbf{Pattern} & \textbf{Matches} & \textbf{Example} \\
\hline
\texttt{\^{}[A-Z]} & & \underline{P}alo Alto \\
\hline
\texttt{\^{}[\^{}A-Za-z]} & & \underline{1} "Hello" \\
\hline
\verb|\.\$| & The end. & The end\underline{.} \\
\hline
\verb|.\$| & The end? The end! & The end\underline{?} The end\underline{!} \\
\hline
\end{tabular}
\end{table}

\textbf{Anchor meanings:}
\begin{itemize}
    \item \texttt{\^{}}: Matches the start of a line
    \item \texttt{\$}: Matches the end of a line
    \item \verb|\.\$|: Matches a literal period at the end of a line
    \item \verb|.\$|: Matches any character at the end of a line
\end{itemize}

\subsection{Example: Finding "the"}

\textbf{Task:} Find all instances of the word "the" in a text.

\textbf{Attempt 1:} \texttt{the}
\begin{itemize}
    \item Problem: Misses capitalized examples (The)
\end{itemize}

\textbf{Attempt 2:} \texttt{[tT]he}
\begin{itemize}
    \item Problem: Incorrectly returns \textit{other} or \textit{theology}
\end{itemize}

\textbf{Correct solution:} \texttt{[\^{}a-zA-Z][tT]he[\^{}a-zA-Z]}
\begin{itemize}
    \item Ensures "the" is surrounded by non-letter characters (word boundaries)
\end{itemize}

\subsection{Errors in Regular Expressions}

When using regular expressions, we deal with two kinds of errors:

\subsubsection{Type I Errors: False Positives}

\textbf{Matching strings that we should not have matched}

Examples: matching \textit{there}, \textit{then}, \textit{other} when searching for "the"

\subsubsection{Type II Errors: False Negatives}

\textbf{Not matching things that we should have matched}

Example: not matching "The" (capitalized) when searching for "the"

\subsubsection{Error Reduction in NLP}

In NLP we are always dealing with these kinds of errors.

Reducing the error rate for an application often involves two antagonistic efforts:
\begin{itemize}
    \item \textbf{Increasing accuracy or precision}: Minimizing false positives
    \item \textbf{Increasing coverage or recall}: Minimizing false negatives
\end{itemize}

These two goals often conflict - improving one can worsen the other.

\subsection{Summary}

\textbf{Regular expressions play a surprisingly large role in NLP:}
\begin{itemize}
    \item Sophisticated sequences of regular expressions are often the first model for any text processing task
\end{itemize}

\textbf{For hard tasks, we use machine learning classifiers:}
\begin{itemize}
    \item But regular expressions are still used for pre-processing, or as features in the classifiers
    \item Can be very useful in capturing generalizations
\end{itemize}

\section{Substitutions and Capture Groups}

\subsection{Substitutions}

Substitution is an operation that finds text matching a pattern and replaces it with different text. This is useful for text normalization, cleaning, and transformation tasks.

\textbf{Syntax in Python and UNIX commands:}

\texttt{s/regexp1/pattern/}

where:
\begin{itemize}
    \item \texttt{s} indicates substitution
    \item \texttt{regexp1} is the pattern to search for
    \item \texttt{pattern} is the replacement text
\end{itemize}

\textbf{Example:}

\texttt{s/colour/color/}

This finds every occurrence of "colour" and replaces it with "color". This is commonly used for text normalization (e.g., converting British to American spelling).

\subsection{Capture Groups}

\textbf{Capture groups} are a powerful feature that allows you to "remember" parts of the matched text and reuse them in the replacement. This enables transformations that depend on the actual content matched, not just fixed replacements.

\subsubsection{Basic Concept}

\textbf{Task:} Put angles around all numbers

\textit{the 35 boxes} $\rightarrow$ \textit{the <35> boxes}

The challenge here is that we don't know in advance which number will appear. We need to:
\begin{enumerate}
    \item Find any number
    \item Remember what that number was
    \item Use it in the replacement
\end{enumerate}

\textbf{How it works:}
\begin{itemize}
    \item Use parentheses \texttt{()} to "capture" a pattern into a numbered register (1, 2, 3...)
    \item The captured text is stored temporarily
    \item Use \verb|\1| to refer to the contents of the first register in the replacement
\end{itemize}

\textbf{Pattern:}

\texttt{s/([0-9]+)/<\textbackslash 1>/}

\textbf{Explanation:}
\begin{itemize}
    \item \texttt{([0-9]+)}: Captures one or more digits into register 1
    \begin{itemize}
        \item The parentheses mark what to save
        \item \texttt{[0-9]+} matches the actual digits
    \end{itemize}
    \item \texttt{<\textbackslash 1>}: Replaces with the captured digits surrounded by angle brackets
    \begin{itemize}
        \item \verb|\1| is replaced with whatever was captured (e.g., "35")
        \item The angle brackets are literal characters added around it
    \end{itemize}
\end{itemize}

So if the input is "the 35 boxes", the pattern captures "35" and replaces it with "<35>".

\subsubsection{Multiple Registers}

You can use multiple capture groups in a single pattern to remember different parts of the match. Each set of parentheses creates a new numbered register.

\textbf{Pattern:}

\texttt{/the (.*) er they (.*), the \textbackslash 1er we \textbackslash 2/}

\textbf{Matches:}

\textit{the faster they ran, the faster we ran}

\textbf{But not:}

\textit{the faster they ran, the faster we ate}

\textbf{Why this works:}
\begin{itemize}
    \item \texttt{(.*)}: First capture group (register 1) - matches "fast"
    \begin{itemize}
        \item \texttt{.*} matches any characters (greedy)
        \item In this case, it captures everything before "er"
    \end{itemize}
    \item \texttt{(.*)}: Second capture group (register 2) - matches "ran"
    \begin{itemize}
        \item Captures the word after "they"
    \end{itemize}
    \item \verb|\1|: References first capture - must match "fast" again
    \item \verb|\2|: References second capture - must match "ran" again
\end{itemize}

The pattern ensures that the same words appear in both parts of the sentence. This is why "the faster they ran, the faster we ran" matches (both parts have "fast" and "ran"), but "the faster they ran, the faster we ate" doesn't match (second part has "ate" instead of "ran").

\textbf{Key insight:} Backreferences (\verb|\1|, \verb|\2|, etc.) don't just match any text - they must match exactly the same text that was captured earlier. This allows you to find repeated patterns or enforce consistency within a match.

\subsection{Non-Capturing Groups}

Sometimes we want to use parentheses for grouping terms without creating a capture register. This is useful when we need grouping for operators (like \texttt{|} or quantifiers) but don't need to reference the matched text later.

\textbf{Problem:} Parentheses have a double function - grouping terms AND capturing.

\textbf{Solution:} Non-capturing groups use \texttt{?:} after the opening parenthesis.

\textbf{Syntax:} \texttt{(?:pattern)}

\subsubsection{Example}

\textbf{Pattern:}

\texttt{/(?:some|a few) (people|cats) like some \textbackslash 1/}

\textbf{Matches:}

\textit{some cats like some cats}

\textbf{But not:}

\textit{some cats like some some}

\textbf{Explanation:}
\begin{itemize}
    \item \texttt{(?:some|a few)}: Non-capturing group - matches either "some" or "a few"
    \begin{itemize}
        \item The \texttt{?:} tells the regex engine not to save this match
        \item This is just for grouping the alternatives
    \end{itemize}
    \item \texttt{(people|cats)}: First (and only) capturing group - register 1
    \begin{itemize}
        \item This IS captured because it doesn't have \texttt{?:}
        \item Saves either "people" or "cats"
    \end{itemize}
    \item \verb|\1|: References register 1, which contains "people" or "cats"
    \begin{itemize}
        \item Note: NOT "some" or "a few" because those weren't captured
    \end{itemize}
\end{itemize}

\textbf{Why use non-capturing groups?}
\begin{itemize}
    \item \textbf{Efficiency}: Capturing has a small performance cost
    \item \textbf{Clarity}: Makes it clear which parts you actually need to reference
    \item \textbf{Register numbering}: Keeps register numbers simple when you have many groups
\end{itemize}

\subsection{Lookahead Assertions}

Lookahead assertions are special patterns that check if something appears ahead in the text without actually consuming (matching) those characters. They're called "zero-width" because they don't advance the character pointer.

\subsubsection{Positive Lookahead (?=pattern)}

\textbf{Syntax:} \texttt{(?=pattern)}

\textbf{Meaning:} True if pattern matches at this position, but doesn't consume the characters.

\textbf{Use case:} Check that something appears ahead without including it in the match.

\subsubsection{Negative Lookahead (?!pattern)}

\textbf{Syntax:} \texttt{(?!pattern)}

\textbf{Meaning:} True if pattern does NOT match at this position.

\textbf{Use case:} Ensure something doesn't appear ahead.

\subsubsection{Example: Words Not Starting with "Volcano"}

\textbf{Task:} Match, at the beginning of a line, any single word that doesn't start with "Volcano"

\textbf{Pattern:}

\texttt{\^{}(?!Volcano)[A-Za-z]+/}

\textbf{How it works:}
\begin{itemize}
    \item \texttt{\^{}}: Anchor to start of line
    \item \texttt{(?!Volcano)}: Negative lookahead - ensures "Volcano" doesn't appear here
    \begin{itemize}
        \item This checks but doesn't consume characters
        \item If "Volcano" is found, the match fails immediately
    \end{itemize}
    \item \texttt{[A-Za-z]+}: Matches one or more letters (the actual word)
    \begin{itemize}
        \item This is what actually gets matched and consumed
    \end{itemize}
\end{itemize}

\textbf{Example matches:}
\begin{itemize}
    \item "Mountain erupted" - matches "Mountain"
    \item "Earthquake struck" - matches "Earthquake"
\end{itemize}

\textbf{Does not match:}
\begin{itemize}
    \item "Volcano erupted" - fails because of negative lookahead
    \item "Volcanoes are hot" - fails because it starts with "Volcano"
\end{itemize}

\textbf{Key concept:} The lookahead \texttt{(?!Volcano)} doesn't consume "Volcano" if it were there - it just checks. This is why we still need \texttt{[A-Za-z]+} to actually match the word. Without the lookahead, we'd match any word including those starting with "Volcano".

\section{Simple Application: ELIZA}

\subsection{What is ELIZA?}

ELIZA is an early NLP system created by Joseph Weizenbaum in 1966 that imitated a Rogerian psychotherapist. It was one of the first chatbots and demonstrated how simple pattern matching could create the illusion of understanding.

\textbf{Rogerian psychotherapy} is a non-directive approach where the therapist reflects the patient's statements back to them, encouraging them to explore their own thoughts and feelings.

\subsection{How ELIZA Works}

ELIZA uses pattern matching with regular expressions to recognize patterns in user input and transform them into responses.

\textbf{Basic approach:}
\begin{enumerate}
    \item Match user input against a set of patterns
    \item Extract key parts using capture groups
    \item Transform the input into a response using substitution
\end{enumerate}

\textbf{Example interaction:}

\textbf{User:} "I need X"

\textbf{ELIZA translates to:} "What would it mean to you if you got X?"

\subsection{Example Conversation}

\textbf{User:} Men are all alike.

\textbf{ELIZA:} IN WHAT WAY

\textbf{User:} They're always bugging us about something or other.

\textbf{ELIZA:} CAN YOU THINK OF A SPECIFIC EXAMPLE

\textbf{User:} Well, my boyfriend made me come here.

\textbf{ELIZA:} YOUR BOYFRIEND MADE YOU COME HERE

\textbf{User:} He says I'm depressed much of the time.

\textbf{ELIZA:} I AM SORRY TO HEAR YOU ARE DEPRESSED

\textbf{Explanation:}
\begin{itemize}
    \item ELIZA recognizes keywords like "all", "always", "my boyfriend", "I'm depressed"
    \item It uses pre-programmed responses for each pattern
    \item Some responses reflect back what the user said (using capture groups)
    \item Others ask generic questions to keep the conversation going
\end{itemize}

\subsection{ELIZA's Pattern Rules}

Here are some of the actual regex patterns ELIZA uses:

\begin{enumerate}
    \item \textbf{Detecting depression/sadness:}
    
    \texttt{s/.* I'M (depressed|sad) .*/I AM SORRY TO HEAR YOU ARE \textbackslash 1/}
    
    \begin{itemize}
        \item Matches: "I'm depressed" or "I'm sad" anywhere in the sentence
        \item Captures: the emotion word (depressed or sad)
        \item Response: "I AM SORRY TO HEAR YOU ARE [emotion]"
    \end{itemize}
    
    \item \textbf{Reflecting "I am" statements:}
    
    \texttt{s/.* I AM (depressed|sad) .*/WHY DO YOU THINK YOU ARE \textbackslash 1/}
    
    \begin{itemize}
        \item Matches: "I am depressed" or "I am sad"
        \item Response: "WHY DO YOU THINK YOU ARE [emotion]"
    \end{itemize}
    
    \item \textbf{Responding to "all" statements:}
    
    \texttt{s/.* all .*/IN WHAT WAY?/}
    
    \begin{itemize}
        \item Matches: any sentence containing "all"
        \item Response: "IN WHAT WAY?" (prompts for specifics)
    \end{itemize}
    
    \item \textbf{Responding to "always" statements:}
    
    \texttt{s/.* always .*/CAN YOU THINK OF A SPECIFIC EXAMPLE?/}
    
    \begin{itemize}
        \item Matches: any sentence containing "always"
        \item Response: asks for concrete examples
    \end{itemize}
\end{enumerate}

\subsection{Key Techniques in ELIZA}

\subsubsection{Pattern Matching}

ELIZA uses \texttt{.*} extensively to match any text before and after keywords. This allows it to find patterns regardless of what else appears in the sentence.

\subsubsection{Capture and Transform}

By capturing parts of the user's input (like emotions or subjects), ELIZA can incorporate them into responses, making the conversation feel more personalized.

\textbf{Example:}
\begin{itemize}
    \item Input: "my boyfriend made me come here"
    \item Pattern captures: "boyfriend"
    \item Response includes: "YOUR BOYFRIEND MADE YOU COME HERE"
\end{itemize}

\subsubsection{Pronoun Switching}

ELIZA switches pronouns when reflecting statements:
\begin{itemize}
    \item "I" $\rightarrow$ "you"
    \item "my" $\rightarrow$ "your"
    \item "me" $\rightarrow$ "you"
\end{itemize}

This is done through additional substitution rules after the main pattern matching.

\subsection{Limitations of ELIZA}

Despite its impressive appearance, ELIZA has significant limitations:

\begin{itemize}
    \item \textbf{No understanding}: ELIZA doesn't understand meaning - it only matches patterns
    \item \textbf{No memory}: It doesn't remember previous parts of the conversation
    \item \textbf{No reasoning}: It can't make logical inferences or connections
    \item \textbf{Brittle patterns}: Slight variations in phrasing can cause it to fail
    \item \textbf{Generic responses}: When no pattern matches, it gives generic responses like "Tell me more"
\end{itemize}

\subsection{Historical Significance}

ELIZA was groundbreaking because:
\begin{itemize}
    \item It demonstrated that simple pattern matching could create engaging interactions
    \item It showed both the power and limitations of rule-based NLP
    \item Many users attributed human-like understanding to it, even though it had none
    \item It inspired decades of chatbot research
    \item It remains a classic example of how regular expressions can be used in NLP applications
\end{itemize}

\textbf{The ELIZA effect:} The tendency of people to attribute human-like understanding to computer programs, even when they know the program is simple. This phenomenon is still relevant today in discussions about AI and chatbots.

\section{Words and Corpora}

\subsection{How Many Words in a Sentence?}

Counting words in a sentence is not as straightforward as it might seem. Different definitions and considerations lead to different counts.

\subsubsection{Example 1: Disfluencies}

\textbf{Sentence:} "I do uh main- mainly business data processing"

\textbf{Issues:}
\begin{itemize}
    \item Contains fragments ("main-")
    \item Contains filled pauses ("uh")
\end{itemize}

\textbf{Question:} Should we count these as words? This depends on the application - for speech recognition we might keep them, for text analysis we might remove them.

\subsubsection{Example 2: Lemmas vs Wordforms}

\textbf{Sentence:} "Seuss's cat in the hat is different from other cats!"

\textbf{Key concepts:}

\begin{itemize}
    \item \textbf{Lemma}: Same stem, part of speech, and rough word sense
    \begin{itemize}
        \item "cat" and "cats" = same lemma
        \item A lemma is the dictionary form or citation form of a word
    \end{itemize}
    
    \item \textbf{Wordform}: The full inflected surface form
    \begin{itemize}
        \item "cat" and "cats" = different wordforms
        \item Wordforms are the actual forms that appear in text
    \end{itemize}
\end{itemize}

\textbf{Why this matters:} Depending on whether we count lemmas or wordforms, we get different word counts. For many NLP tasks, we need to decide which level of granularity is appropriate.

\subsubsection{Example 3: Types vs Tokens}

\textbf{Sentence:} "they lay back on the San Francisco grass and looked at the stars and their"

\textbf{Key concepts:}

\begin{itemize}
    \item \textbf{Type}: An element of the vocabulary (unique word)
    \item \textbf{Token}: An instance of that type in running text (occurrence)
\end{itemize}

\textbf{Counting:}
\begin{itemize}
    \item \textbf{15 tokens} (or 14, depending on how you count)
    \begin{itemize}
        \item Every word occurrence is a token
        \item "the" appears twice, "and" appears twice - each occurrence is a separate token
    \end{itemize}
    
    \item \textbf{13 types} (or 12, or 11?)
    \begin{itemize}
        \item Each unique word is one type
        \item "the" is counted once as a type, even though it appears twice
        \item "and" is counted once as a type, even though it appears twice
    \end{itemize}
\end{itemize}

\textbf{Why the uncertainty?} Different decisions about what counts as a word (e.g., is "San Francisco" one word or two?) lead to different counts.

\subsection{How Many Words in a Corpus?}

A \textbf{corpus} (plural: corpora) is a large collection of text used for linguistic analysis and NLP tasks.

\subsubsection{Notation}

\begin{itemize}
    \item \textbf{$N$} = number of tokens (total word occurrences)
    \item \textbf{$V$} = vocabulary = set of types
    \item \textbf{$|V|$} = size of vocabulary (number of unique words)
\end{itemize}

\subsubsection{Herdan's Law (Heap's Law)}

The relationship between vocabulary size and corpus size follows a power law:

\[
|V| = kN^\beta
\]

where:
\begin{itemize}
    \item $k$ is a constant
    \item $\beta$ is typically between 0.67 and 0.75
\end{itemize}

\textbf{Key insight:} Vocabulary size grows with more than the square root of the number of tokens, but less than linearly. This means:
\begin{itemize}
    \item As you add more text, you keep encountering new words
    \item But the rate of new words decreases as the corpus grows
    \item You never stop finding new words entirely
\end{itemize}

\subsubsection{Examples from Real Corpora}

\begin{table}[h]
\centering
\begin{tabular}{|l|r|r|}
\hline
\textbf{Corpus} & \textbf{Tokens (N)} & \textbf{Types (|V|)} \\
\hline
Switchboard phone conversations & 2.4 million & 20 thousand \\
\hline
Shakespeare & 884,000 & 31 thousand \\
\hline
COCA & 440 million & 2 million \\
\hline
Google N-grams & 1 trillion & 13+ million \\
\hline
\end{tabular}
\caption{Token and type counts in various corpora}
\end{table}

\textbf{Observations:}
\begin{itemize}
    \item \textbf{Switchboard}: Spoken language has relatively small vocabulary for its size (lots of repetition)
    \item \textbf{Shakespeare}: Rich vocabulary despite relatively small corpus
    \item \textbf{COCA} (Corpus of Contemporary American English): Large modern corpus with extensive vocabulary
    \item \textbf{Google N-grams}: Massive web-scale corpus with millions of unique words
\end{itemize}

\textbf{Practical implications:}
\begin{itemize}
    \item Larger corpora are needed to capture rare words and phenomena
    \item The type/token ratio varies significantly across domains and genres
    \item Spoken language tends to have lower type/token ratios than written language
    \item Web-scale corpora contain many rare words, misspellings, and proper nouns
\end{itemize}

\subsection{Understanding Corpora}

\subsubsection{Words Don't Appear Out of Nowhere}

A fundamental principle in corpus linguistics: \textbf{words don't appear out of nowhere}. Every text is produced in a specific context.

A text is produced by:
\begin{itemize}
    \item A specific writer(s)
    \item At a specific time
    \item In a specific variety (of language)
    \item Of a specific language
    \item For a specific function
\end{itemize}

\textbf{Why this matters:} Understanding the context of text production is crucial for interpreting corpus data and building NLP systems. The same word can have different meanings, frequencies, and usage patterns depending on these factors.

\subsection{Corpora Vary Along Multiple Dimensions}

Corpora are not uniform - they vary along several important dimensions that affect the language they contain.

\subsubsection{Language}

There are approximately \textbf{7,097 languages in the world}. Each has its own:
\begin{itemize}
    \item Vocabulary and grammar
    \item Writing system (or lack thereof)
    \item Cultural context
    \item Computational resources available
\end{itemize}

\textbf{Implication:} NLP systems trained on one language don't automatically work for others. Multilingual NLP is a major research area.

\subsubsection{Variety}

Even within a single language, there are many varieties. For example:
\begin{itemize}
    \item \textbf{African American English (AAE)} varieties
    \begin{itemize}
        \item AAE Twitter posts might include forms like "\textit{iont}" (I don't)
        \item Different phonological, grammatical, and lexical features
    \end{itemize}
\end{itemize}

\textbf{Implication:} Systems trained on standard varieties may perform poorly on other varieties, potentially causing bias and fairness issues.

\subsubsection{Code Switching}

\textbf{Code switching} occurs when speakers alternate between two or more languages within a conversation or even within a sentence.

\textbf{Examples:}

\textit{Spanish/English:}
\begin{itemize}
    \item "Por primera vez veo a @username actually being hateful! It was beautiful:)"
    \item Translation: "For the first time I get to see @username actually being hateful! it was beautiful:)"
\end{itemize}

\textit{Hindi/English:}
\begin{itemize}
    \item "dost tha or ra- hega ... dont wory ... but dherya rakhe"
    \item Translation: "he was and will remain a friend ... don't worry ... but have faith"
\end{itemize}

\textbf{Challenges:}
\begin{itemize}
    \item Difficult for monolingual NLP systems
    \item Requires understanding of multiple languages simultaneously
    \item Common in multilingual communities but often ignored in NLP datasets
\end{itemize}

\subsubsection{Genre}

Different genres have distinct linguistic characteristics:
\begin{itemize}
    \item \textbf{Newswire}: Formal, objective, structured
    \item \textbf{Fiction}: Narrative, dialogue, descriptive
    \item \textbf{Scientific articles}: Technical vocabulary, formal structure
    \item \textbf{Wikipedia}: Encyclopedic, informative, hyperlinked
\end{itemize}

\textbf{Implication:} A model trained on news articles may not work well on social media or scientific papers.

\subsubsection{Author Demographics}

The characteristics of the author affect the text:
\begin{itemize}
    \item \textbf{Age}: Younger writers may use different slang and references
    \item \textbf{Gender}: Can influence topic choice and writing style
    \item \textbf{Ethnicity}: Affects cultural references and language variety
    \item \textbf{Socioeconomic status (SES)}: Influences vocabulary and topics
\end{itemize}

\textbf{Implication:} Demographic bias in training data can lead to biased NLP systems that work better for some groups than others.

\subsection{Corpus Datasheets}

To address the complexity of corpora, researchers have proposed \textbf{corpus datasheets} - standardized documentation for datasets.

\textbf{Key references:} Gebru et al (2020), Bender and Friedman (2018)

\subsubsection{Motivation}

Questions about why and how the corpus was created:
\begin{itemize}
    \item \textbf{Why was the corpus collected?} What research question or application?
    \item \textbf{By whom?} Which researchers or organizations?
    \item \textbf{Who funded it?} Funding sources can influence corpus design
\end{itemize}

\subsubsection{Situation}

\textbf{In what situation was the text written?}

Understanding the context of text production:
\begin{itemize}
    \item Social media posts vs. published articles
    \item Spontaneous speech vs. prepared presentations
    \item Private communications vs. public documents
\end{itemize}

\subsubsection{Collection Process}

\textbf{If it is a subsample, how was it sampled? Was there consent? Pre-processing?}

Important methodological details:
\begin{itemize}
    \item \textbf{Sampling method}: Random, stratified, convenience?
    \item \textbf{Consent}: Did authors/speakers consent to their text being used?
    \item \textbf{Pre-processing}: What cleaning or normalization was applied?
    \item \textbf{Ethical considerations}: Privacy, representation, potential harms
\end{itemize}

\subsubsection{Additional Information}

Datasheets should also document:
\begin{itemize}
    \item \textbf{Annotation process}: How was the data labeled? By whom?
    \item \textbf{Language variety}: Which dialect or variety is represented?
    \item \textbf{Demographics}: Who are the authors/speakers?
    \item \textbf{Limitations}: Known biases or gaps in the corpus
\end{itemize}

\subsection{Why Corpus Datasheets Matter}

\textbf{Transparency:} Researchers and practitioners need to understand what data they're working with.

\textbf{Reproducibility:} Proper documentation enables others to replicate and build on research.

\textbf{Fairness:} Understanding corpus composition helps identify and mitigate biases.

\textbf{Appropriate use:} Knowing the context and limitations prevents misuse of corpora and models trained on them.

\textbf{Example concern:} A sentiment analysis system trained on formal news text may fail on informal social media text, or work well for one demographic group but poorly for others. Datasheets help identify these issues before deployment.


\newpage
\part{Text Normalization and Tokenization}

\section{Text Normalization}

Text normalization is a fundamental preprocessing step required by every NLP task. It involves transforming raw text into a standardized format that can be processed by computational methods.

\subsection{The Need for Text Normalization}

\textbf{Every NLP task requires text normalization.}

When working with corpora, we need to consider several factors:
\begin{itemize}
    \item \textbf{Corpus homogeneity}: A corpus is considered homogeneous when certain parameters are consistent (e.g., same language, same author, same genre)
    \item \textbf{Morphological complexity}: Due to morphology, it's often more reasonable to work with types rather than tokens
    \item \textbf{Language-specific properties}: Different languages require different normalization approaches
\end{itemize}

\subsection{The Three Main Steps of Text Normalization}

According to the slides, every NLP task requires three fundamental normalization steps:

\begin{enumerate}
    \item \textbf{Tokenizing (segmenting) words}
    \begin{itemize}
        \item Breaking text into individual word units
        \item Particularly important for agglutinating languages
        \item Determines the basic units of analysis
    \end{itemize}
    
    \item \textbf{Normalizing word formats}
    \begin{itemize}
        \item Standardizing variations of the same word
        \item Handling case differences, punctuation, etc.
        \item Converting to canonical forms
    \end{itemize}
    
    \item \textbf{Segmenting sentences}
    \begin{itemize}
        \item Dividing text into sentence units
        \item Identifying sentence boundaries
        \item Important for syntactic and semantic analysis
    \end{itemize}
\end{enumerate}

\subsection{Linguistic Properties and Tokenization}

\subsubsection{Isolating vs. Agglutinating Languages}

Different languages have different morphological properties that affect tokenization:

\begin{itemize}
    \item \textbf{Isolating languages}: Languages where typically 1 token corresponds to 1 type
    \begin{itemize}
        \item Minimal morphology (no conjugation, plurals, genders, pronouns)
        \item Example: Japanese (highly isolated)
        \item Each word tends to be a single, unchanging unit
    \end{itemize}
    
    \item \textbf{Agglutinating languages}: Languages that combine multiple morphemes into single words
    \begin{itemize}
        \item Complex morphology with many affixes
        \item Single words can express what requires multiple words in other languages
        \item Example: Japanese (also has agglutinating properties)
    \end{itemize}
\end{itemize}

\textbf{Key insight:} The property of being isolating is important to recognize because it affects how we approach tokenization and normalization.

\subsubsection{Tokens and Multi-tokens}

In practice, we encounter:
\begin{itemize}
    \item \textbf{Simple tokens}: Single word units
    \item \textbf{Multi-tokens}: Compound expressions that function as single units
    \begin{itemize}
        \item Example: "New York", "machine learning"
        \item Should these be treated as one token or multiple?
        \item Decision depends on the application
    \end{itemize}
\end{itemize}

\subsection{Space-Based Tokenization}

\subsubsection{The Simple Approach}

The most straightforward tokenization method:

\textbf{Segment off a token between instances of spaces}

This approach works for languages that use space characters to separate words:
\begin{itemize}
    \item Arabic, Cyrillic, Greek, Latin, etc., based writing systems
    \item These languages explicitly mark word boundaries with spaces
\end{itemize}

\subsubsection{Language Bias}

\textbf{Important limitation:} Space-based tokenization has a significant language bias.

\begin{itemize}
    \item \textbf{Works well for}: European languages and others using space-separated writing systems
    \item \textbf{Fails for}: Languages without spaces between words
    \begin{itemize}
        \item Chinese, Japanese, Thai: No spaces between words
        \item Requires different tokenization approaches
    \end{itemize}
    \item \textbf{Directional bias}: Assumes left-to-right reading
    \begin{itemize}
        \item Doesn't account for right-to-left languages (Arabic, Hebrew)
        \item May need adjustment for bidirectional text
    \end{itemize}
\end{itemize}

\subsection{Unix Tools for Space-Based Tokenization}

Unix provides powerful command-line tools for text processing, particularly suited for space-based tokenization.

\subsubsection{The \texttt{tr} Command}

The \texttt{tr} (translate) command is fundamental for tokenization:

\textbf{Purpose:} Transform characters in text, commonly used to separate words

\textbf{Basic operation:}
\begin{itemize}
    \item Takes a file with words separated by spaces
    \item Generates a list of tokens (one per line)
    \item Pure tokenization without additional processing
\end{itemize}

\subsubsection{Simple Tokenization in UNIX}

Following Ken Church's "UNIX for Poets" approach, here's a complete tokenization pipeline:

\begin{verbatim}
tr -sc 'A-Za-z' '\n' < shakes.txt
    | sort
    | uniq -c
\end{verbatim}

\textbf{Step-by-step explanation:}

\begin{enumerate}
    \item \textbf{\texttt{tr -sc 'A-Za-z' '\textbackslash n' < shakes.txt}}
    \begin{itemize}
        \item \texttt{-s}: Squeeze repeated characters
        \item \texttt{-c}: Complement the set (everything NOT in A-Za-z)
        \item \texttt{'A-Za-z'}: Keep only alphabetic characters
        \item \texttt{'\textbackslash n'}: Replace all non-alphabetic characters with newlines
        \item \textbf{Effect}: Changes all non-alpha characters to newlines, creating one word per line
    \end{itemize}
    
    \item \textbf{\texttt{| sort}}
    \begin{itemize}
        \item Sorts all words in alphabetical order
        \item Groups identical words together
        \item Necessary for the next step
    \end{itemize}
    
    \item \textbf{\texttt{| uniq -c}}
    \begin{itemize}
        \item \texttt{uniq}: Removes duplicate adjacent lines
        \item \texttt{-c}: Counts occurrences of each unique line
        \item \textbf{Effect}: Merges identical words and counts each type
    \end{itemize}
\end{enumerate}

\textbf{Output format:}
\begin{verbatim}
1945 A
  72 AARON
  19 ABBESS
   5 ABBOT
  ...
\end{verbatim}

Each line shows the frequency count followed by the word type.

\subsubsection{Example: The First Step (Tokenizing)}

From the slides, the first step of the pipeline:

\begin{verbatim}
tr -sc 'A-Za-z' '\n' < shakes.txt | head
\end{verbatim}

\textbf{Output:}
\begin{verbatim}
THE
SONNETS
by
William
Shakespeare
From
fairest
creatures
We
...
\end{verbatim}

Each word appears on its own line, ready for further processing.

\subsubsection{Example: The Second Step (Sorting)}

After sorting the tokenized words:

\begin{verbatim}
tr -sc 'A-Za-z' '\n' < shakes.txt | sort | head
\end{verbatim}

\textbf{Output:}
\begin{verbatim}
A
A
A
A
A
A
A
A
A
...
\end{verbatim}

All identical words are now grouped together, preparing for counting.

\subsection{The Bag of Words Model}

\subsubsection{Definition}

The \textbf{bag of words} model is a simplified representation of text that:
\begin{itemize}
    \item Treats text as an unordered collection of words
    \item Counts word frequencies
    \item Ignores word order, grammar, and structure
\end{itemize}

\textbf{Mathematical representation:} Similar to multisets in set theory
\begin{itemize}
    \item Example: "aabbba" becomes $\{a: 3, b: 3\}$
    \item Each unique word (type) is associated with its count
\end{itemize}

\subsubsection{Limitations of Bag of Words}

Despite its widespread use, the bag of words model has significant limitations:

\begin{itemize}
    \item \textbf{Loss of word order}
    \begin{itemize}
        \item "The dog bit the man" vs "The man bit the dog" are identical
        \item Cannot capture syntactic relationships
    \end{itemize}
    
    \item \textbf{Ignores morphology}
    \begin{itemize}
        \item "run", "runs", "running" are treated as completely different words
        \item No understanding of word relationships
    \end{itemize}
    
    \item \textbf{Problems with agglutinating languages}
    \begin{itemize}
        \item Complex morphology creates explosion of unique forms
        \item Loses semantic relationships between related forms
    \end{itemize}
    
    \item \textbf{Directional bias}
    \begin{itemize}
        \item Assumes left-to-right processing
        \item Doesn't work well for right-to-left languages (Arabic, Hebrew)
    \end{itemize}
    
    \item \textbf{No semantic understanding}
    \begin{itemize}
        \item Cannot distinguish between different senses of the same word
        \item No understanding of context or meaning
    \end{itemize}
\end{itemize}

\textbf{Despite these limitations}, bag of words remains useful for:
\begin{itemize}
    \item Document classification
    \item Information retrieval
    \item Simple text analysis tasks
    \item Baseline models for comparison
\end{itemize}

\subsection{Practical Considerations}

\subsubsection{When to Use Space-Based Tokenization}

Space-based tokenization is appropriate when:
\begin{itemize}
    \item Working with languages that use spaces (English, Spanish, etc.)
    \item Need quick, simple preprocessing
    \item Computational resources are limited
    \item Task doesn't require sophisticated linguistic analysis
\end{itemize}

\subsubsection{When More Sophisticated Methods Are Needed}

More advanced tokenization is required for:
\begin{itemize}
    \item Languages without space separation (Chinese, Japanese, Thai)
    \item Handling contractions and compound words
    \item Preserving multi-word expressions
    \item Dealing with social media text (hashtags, mentions, emojis)
    \item Subword tokenization for neural models (BPE, WordPiece)
\end{itemize}

\section{Advanced Tokenization Techniques}

\subsection{More Counting: Merging Upper and Lower Case}

When performing frequency analysis, we often want to treat uppercase and lowercase versions of the same word as identical.

\textbf{Command:}
\begin{verbatim}
tr 'A-Z' 'a-z' < shakes.txt | tr -sc 'A-Za-z' '\n' | sort | uniq -c
\end{verbatim}

\textbf{Explanation:}
\begin{itemize}
    \item \textbf{First \texttt{tr 'A-Z' 'a-z'}}: Converts all uppercase letters to lowercase
    \begin{itemize}
        \item This merges "The" and "the" into the same token
        \item Reduces vocabulary size by eliminating case variations
    \end{itemize}
    \item \textbf{Then tokenize, sort, and count} as before
\end{itemize}

\textbf{Result:} Words that differ only in capitalization are counted together.

\subsection{Sorting the Counts}

To find the most frequent words, we can sort by frequency instead of alphabetically.

\textbf{Command:}
\begin{verbatim}
tr 'A-Z' 'a-z' < shakes.txt | tr -sc 'A-Za-z' '\n' | sort | uniq -c | sort -n -r
\end{verbatim}

\textbf{New component:}
\begin{itemize}
    \item \textbf{\texttt{sort -n -r}}: Sort numerically in reverse order
    \begin{itemize}
        \item \texttt{-n}: Numerical sort (treats numbers as numbers, not strings)
        \item \texttt{-r}: Reverse order (highest to lowest)
    \end{itemize}
\end{itemize}

\textbf{Output example:}
\begin{verbatim}
23243 the
22225 i
18618 and
16339 to
15687 of
12780 a
12163 you
10839 my
10005 in
 8954 d
\end{verbatim}

\textbf{Question from the slide:} "What happened here?"

The output shows the most frequent words in Shakespeare's works, with "the" appearing 23,243 times. Notice that:
\begin{itemize}
    \item Function words (the, i, and, to, of, a) dominate the frequency list
    \item The letter "d" appears as a separate token (likely from contractions like "I'd")
    \item This demonstrates Zipf's law: a small number of words account for most occurrences
\end{itemize}

\section{Issues in Tokenization}

Tokenization is more complex than simply splitting on spaces. Several challenging cases require careful consideration.

\subsection{Punctuation Cannot Be Blindly Removed}

Many tokens contain punctuation that carries meaning and should not be automatically stripped:

\begin{itemize}
    \item \textbf{Abbreviations}: m.p.h., Ph.D., AT\&T, cap'n
    \begin{itemize}
        \item Periods are integral to the abbreviation
        \item Removing them would destroy the meaning
    \end{itemize}
    
    \item \textbf{Prices}: \$45.55
    \begin{itemize}
        \item The period separates dollars from cents
        \item The dollar sign indicates currency
    \end{itemize}
    
    \item \textbf{Dates}: 01/02/06
    \begin{itemize}
        \item Slashes separate day, month, and year
        \item Different formats exist (US vs European)
    \end{itemize}
    
    \item \textbf{URLs}: http://www.stanford.edu
    \begin{itemize}
        \item Periods, slashes, and colons are structural
        \item Should be kept as a single token
    \end{itemize}
    
    \item \textbf{Hashtags}: \#nlproc
    \begin{itemize}
        \item The hash symbol is part of the token
        \item Common in social media text
    \end{itemize}
    
    \item \textbf{Email addresses}: someone@cs.colorado.edu
    \begin{itemize}
        \item Contains periods and @ symbol
        \item Should remain as a single token
    \end{itemize}
\end{itemize}

\textbf{Key insight:} Punctuation removal must be context-aware. Blindly stripping all punctuation destroys important information.

\subsection{Clitics}

\textbf{Clitics} are words that don't stand on their own grammatically and attach to other words.

\textbf{Examples:}
\begin{itemize}
    \item \textbf{English}: "are" in \textit{we're}, "je" in French \textit{j'ai}, "le" in \textit{l'honneur}
    \begin{itemize}
        \item These are grammatically separate words but phonologically attached
        \item Question: Should "we're" be one token or two ("we" + "are")?
    \end{itemize}
\end{itemize}

\textbf{Tokenization decisions:}
\begin{itemize}
    \item \textbf{Keep as one token}: Preserves the surface form as written
    \item \textbf{Split into two}: Normalizes to grammatical units
    \item \textbf{Choice depends on}:
    \begin{itemize}
        \item The NLP task (parsing might prefer splitting)
        \item The language (different languages have different conventions)
        \item Consistency with training data
    \end{itemize}
\end{itemize}

\subsection{Multiword Expressions (MWEs)}

\textbf{When should multiword expressions be treated as single words?}

\textbf{Examples:}
\begin{itemize}
    \item \textbf{Proper nouns}: New York
    \begin{itemize}
        \item Refers to a single entity (the city)
        \item "New" and "York" separately have different meanings
    \end{itemize}
    
    \item \textbf{Idiomatic expressions}: rock 'n' roll
    \begin{itemize}
        \item The meaning is not compositional
        \item Cannot be understood from individual words
    \end{itemize}
\end{itemize}

\textbf{Challenges:}
\begin{itemize}
    \item \textbf{Identification}: How do we automatically detect MWEs?
    \item \textbf{Ambiguity}: "New York" could also be "new" + "York" in other contexts
    \item \textbf{Variability}: Some MWEs allow internal modification ("brand new car" vs "brand car")
    \item \textbf{Language-specific}: Different languages have different MWE patterns
\end{itemize}

\textbf{Approaches:}
\begin{itemize}
    \item Use a dictionary of known MWEs
    \item Statistical methods to identify frequent collocations
    \item Named entity recognition for proper nouns
    \item Task-specific decisions based on application needs
\end{itemize}

\section{Tokenization in NLTK}

The Natural Language Toolkit (NLTK) provides sophisticated tokenization tools that handle many of the issues discussed above.

\subsection{Regular Expression Tokenization}

NLTK's \texttt{regexp\_tokenize} function allows flexible tokenization using regular expressions.

\textbf{Example from Bird, Loper and Klein (2009), \textit{Natural Language Processing with Python}:}

\begin{verbatim}
>>> text = 'That U.S.A. poster-print costs $12.40...'
>>> pattern = r'''(?x)    # set flag to allow verbose regexps
...     ([A-Z]\.)+        # abbreviations, e.g. U.S.A.
...     | \w+(-\w+)*      # words with optional internal hyphens
...     | \$?\d+(\.\d+)?%?  # currency and percentages, e.g. $12.40, 82%
...     | \.\.\.          # ellipsis
...     | [][.,;"'?():-_`]  # these are separate tokens; includes ], [
... '''
>>> nltk.regexp_tokenize(text, pattern)
['That', 'U.S.A.', 'poster-print', 'costs', '$12.40', '...']
\end{verbatim}

\textbf{Pattern breakdown:}
\begin{enumerate}
    \item \textbf{\texttt{([A-Z]\.)+}}: Abbreviations
    \begin{itemize}
        \item Matches sequences like "U.S.A."
        \item Keeps periods as part of the token
    \end{itemize}
    
    \item \textbf{Words with optional internal hyphens} (\texttt{\textbackslash w+(-\textbackslash w+)*}):
    \begin{itemize}
        \item Matches "poster-print" as a single token
        \item Handles compound words
    \end{itemize}
    
    \item \textbf{Currency and percentages} (\texttt{\textbackslash\$?\textbackslash d+(\textbackslash.\textbackslash d+)?\%?}):
    \begin{itemize}
        \item Matches "\$12.40" and "82\%"
        \item Keeps monetary amounts together
    \end{itemize}
    
    \item \textbf{Ellipsis} (\texttt{\textbackslash.\textbackslash.\textbackslash.}):
    \begin{itemize}
        \item Treats "..." as a single token
        \item Distinguishes from sentence-ending periods
    \end{itemize}
    
    \item \textbf{Separate punctuation tokens} (\texttt{[][.,;"'?():-\_`]}):
    \begin{itemize}
        \item These punctuation marks become individual tokens
        \item Includes brackets, parentheses, quotes, etc.
    \end{itemize}
\end{enumerate}

\textbf{Advantages of regex tokenization:}
\begin{itemize}
    \item Highly customizable for specific domains
    \item Can handle complex patterns
    \item Explicit rules make behavior predictable
    \item Can be tuned for different languages and text types
\end{itemize}

\textbf{Disadvantages:}
\begin{itemize}
    \item Requires careful pattern design
    \item May not generalize well to new domains
    \item Can become complex and hard to maintain
    \item Order of patterns matters (first match wins)
\end{itemize}

\section{Tokenization in Languages Without Spaces}

\subsection{The Challenge}

Many languages (like Chinese, Japanese, Thai) don't use spaces to separate words!

\textbf{Fundamental question:} How do we decide where the token boundaries should be?

\textbf{Example (Chinese):}
\begin{itemize}
    \item Written text: \begin{CJK}{UTF8}{gbsn}我爱自然语言处理\end{CJK}
    \item Possible tokenizations:
    \begin{itemize}
        \item \begin{CJK}{UTF8}{gbsn}我 / 爱 / 自然 / 语言 / 处理\end{CJK} (I / love / natural / language / processing)
        \item \begin{CJK}{UTF8}{gbsn}我 / 爱 / 自然语言 / 处理\end{CJK} (I / love / natural-language / processing)
        \item \begin{CJK}{UTF8}{gbsn}我 / 爱 / 自然语言处理\end{CJK} (I / love / natural-language-processing)
    \end{itemize}
    \item Different segmentations can be valid depending on the interpretation
    \item No spaces in the original text to guide segmentation
\end{itemize}

\subsection{Approaches to Word Segmentation}

\textbf{1. Dictionary-based methods:}
\begin{itemize}
    \item Use a lexicon of known words
    \item Find the longest matching words in the text
    \item \textbf{Problem}: Ambiguity when multiple segmentations are possible
    \item \textbf{Problem}: Out-of-vocabulary words (new terms, names)
\end{itemize}

\textbf{2. Statistical methods:}
\begin{itemize}
    \item Learn word boundaries from segmented training data
    \item Use probabilistic models (e.g., Hidden Markov Models)
    \item Choose the most likely segmentation based on learned patterns
    \item Better handles ambiguity and unknown words
\end{itemize}

\textbf{3. Neural methods:}
\begin{itemize}
    \item Use deep learning models (LSTMs, Transformers)
    \item Learn character-level representations
    \item Predict word boundaries as a sequence labeling task
    \item State-of-the-art performance for many languages
\end{itemize}

\textbf{4. Subword tokenization:}
\begin{itemize}
    \item Break words into smaller units (characters or subwords)
    \item Used in modern neural NLP (BPE, WordPiece, SentencePiece)
    \item Avoids the word segmentation problem entirely
    \item Works across languages, including those with and without spaces
\end{itemize}

\subsection{Language-Specific Challenges}

\textbf{Chinese:}
\begin{itemize}
    \item No spaces between words
    \item Characters can be words or parts of words
    \item Ambiguous segmentation is common
    \item Proper nouns and new terms are challenging
\end{itemize}

\subsection{Word Tokenization in Chinese: A Detailed Look}

\textbf{Chinese Character Structure}

Chinese words are composed of characters called \textbf{"hanzi"} (or sometimes just \textbf{"zi"}).

\textbf{Key properties:}
\begin{itemize}
    \item Each character represents a \textbf{meaning unit} called a \textbf{morpheme}
    \item Each word has on average \textbf{2.4 characters}
    \item Deciding what counts as a word is \textbf{complex and not agreed upon}
\end{itemize}

\textbf{The ambiguity problem:} Unlike English where word boundaries are marked by spaces, Chinese text is continuous, making it unclear where one word ends and another begins.

\subsection{Example: Multiple Segmentation Possibilities}

Consider the Chinese sentence: \begin{CJK}{UTF8}{gbsn}姚明进入总决赛\end{CJK} ("Yao Ming reaches the finals")

This sentence can be segmented in multiple valid ways:

\textbf{1. Three words (3-word segmentation):}
\begin{itemize}
    \item \begin{CJK}{UTF8}{gbsn}姚明\end{CJK} / \begin{CJK}{UTF8}{gbsn}进入\end{CJK} / \begin{CJK}{UTF8}{gbsn}总决赛\end{CJK}
    \item YaoMing / reaches / finals
    \item This treats "Yao Ming" as a single name token
\end{itemize}

\textbf{2. Five words (5-word segmentation):}
\begin{itemize}
    \item \begin{CJK}{UTF8}{gbsn}姚\end{CJK} / \begin{CJK}{UTF8}{gbsn}明\end{CJK} / \begin{CJK}{UTF8}{gbsn}进入\end{CJK} / \begin{CJK}{UTF8}{gbsn}总\end{CJK} / \begin{CJK}{UTF8}{gbsn}决赛\end{CJK}
    \item Yao / Ming / reaches / overall / finals
    \item This splits the name and breaks down compound words
\end{itemize}

\textbf{3. Seven characters (character-level, don't use words at all):}
\begin{itemize}
    \item \begin{CJK}{UTF8}{gbsn}姚\end{CJK} / \begin{CJK}{UTF8}{gbsn}明\end{CJK} / \begin{CJK}{UTF8}{gbsn}进\end{CJK} / \begin{CJK}{UTF8}{gbsn}入\end{CJK} / \begin{CJK}{UTF8}{gbsn}总\end{CJK} / \begin{CJK}{UTF8}{gbsn}决\end{CJK} / \begin{CJK}{UTF8}{gbsn}赛\end{CJK}
    \item Yao / Ming / enter / enter / overall / decision / game
    \item This treats each character as a separate token
\end{itemize}

\textbf{Key insight:} Each segmentation is linguistically defensible, but they lead to different analyses and different vocabulary sizes.

\subsection{Practical Approach: Character-Based Tokenization}

\textbf{Common solution:} In Chinese, it's common to just treat each character (zi) as a token.

\textbf{Advantages:}
\begin{itemize}
    \item \textbf{Very simple}: The segmentation step is trivial - just split into characters
    \item \textbf{No ambiguity}: No need to decide word boundaries
    \item \textbf{Consistent}: Works uniformly across all texts
    \item \textbf{Handles unknowns}: New words are automatically handled as character sequences
\end{itemize}

\textbf{Disadvantages:}
\begin{itemize}
    \item Loses word-level semantic information
    \item Larger sequence lengths (more tokens per sentence)
    \item May not capture multi-character word meanings effectively
\end{itemize}

\subsection{Comparison with Other Languages}

\textbf{Chinese:} Character-based tokenization is simple and effective
\begin{itemize}
    \item So the segmentation step is very simple
    \item Just split on character boundaries
\end{itemize}

\textbf{Thai and Japanese:} More complex word segmentation is required
\begin{itemize}
    \item Cannot simply use character-level tokenization
    \item Thai: Requires understanding of word boundaries within continuous script
    \item Japanese: Must handle three writing systems and grammatical particles
\end{itemize}

\textbf{Modern approach:} The standard algorithms are \textbf{neural sequence models trained by supervised machine learning}
\begin{itemize}
    \item Use deep learning (LSTMs, Transformers)
    \item Learn from annotated training data
    \item Predict word boundaries as a sequence labeling task
    \item Achieve state-of-the-art performance
\end{itemize}

\subsection{Summary: Chinese Tokenization}

\begin{itemize}
    \item \textbf{Hanzi/zi}: Chinese characters are meaning units (morphemes)
    \item \textbf{Average word length}: 2.4 characters per word
    \item \textbf{Ambiguity}: Multiple valid segmentations exist for the same text
    \item \textbf{Practical solution}: Character-level tokenization is common and simple
    \item \textbf{Alternative}: Neural sequence models for word-level segmentation
    \item \textbf{Trade-off}: Simplicity vs. linguistic accuracy
\end{itemize}

\textbf{Japanese:}
\begin{itemize}
    \item Mixes three writing systems (Hiragana, Katakana, Kanji)
    \item No spaces between words
    \item Highly agglutinative (words combine with particles)
    \item Writing system can provide some segmentation clues
\end{itemize}

\textbf{Thai:}
\begin{itemize}
    \item No spaces between words
    \item Spaces indicate phrase or sentence boundaries
    \item Requires specialized segmentation tools
\end{itemize}

\subsection{Practical Tools}

Several tools are available for segmenting languages without spaces:

\begin{itemize}
    \item \textbf{Jieba} (Chinese): Popular open-source Chinese segmenter
    \item \textbf{MeCab} (Japanese): Morphological analyzer and tokenizer
    \item \textbf{PyThaiNLP} (Thai): Thai language processing toolkit
    \item \textbf{Stanford Word Segmenter}: Multi-language statistical segmenter
    \item \textbf{SentencePiece}: Language-agnostic subword tokenizer
\end{itemize}

\section{Subword Tokenization}

\subsection{Motivation}

Instead of white-space or single-character segmentation, we can \textbf{use the data to tell us how to tokenize}.

\textbf{Subword tokenization}: Tokens can be parts of words as well as whole words.

\textbf{Advantages:}
\begin{itemize}
    \item Handles unknown words
    \item Reduces vocabulary size
    \item Captures morphology
    \item No out-of-vocabulary problem
\end{itemize}

\subsection{Three Common Algorithms}

\begin{enumerate}
    \item \textbf{Byte-Pair Encoding (BPE)} (Sennrich et al., 2016) - Used in GPT
    \item \textbf{Unigram Language Modeling} (Kudo, 2018) - Used in T5
    \item \textbf{WordPiece} (Schuster and Nakajima, 2012) - Used in BERT
\end{enumerate}

\textbf{All have 2 parts:}
\begin{enumerate}
    \item \textbf{Token learner}: Takes training corpus, induces vocabulary
    \item \textbf{Token segmenter}: Tokenizes new text using that vocabulary
\end{enumerate}

\subsection{Byte-Pair Encoding (BPE)}

\textbf{Initial vocabulary:} All individual characters $V = \{A, B, C, \ldots, a, b, c, \ldots\}$

\textbf{Algorithm - Repeat $k$ times:}
\begin{enumerate}
    \item Choose the two symbols most frequently adjacent in corpus (e.g., 'A', 'B')
    \item Add merged symbol 'AB' to vocabulary: $V \leftarrow V \cup \{\text{AB}\}$
    \item Replace every 'A' 'B' in corpus with 'AB'
\end{enumerate}

Parameter $k$ controls vocabulary size.

\subsection{BPE Algorithm Pseudocode}

\begin{algorithm}
\caption{Byte-Pair Encoding}
\begin{algorithmic}
\STATE \textbf{function} BPE(corpus $C$, merges $k$) \textbf{returns} vocab $V$
\STATE $V \leftarrow$ all unique characters in $C$
\FOR{$i = 1$ to $k$}
    \STATE $t_L, t_R \leftarrow$ Most frequent adjacent pair in $C$
    \STATE $t_{NEW} \leftarrow t_L + t_R$
    \STATE $V \leftarrow V + t_{NEW}$
    \STATE Replace $t_L, t_R$ in $C$ with $t_{NEW}$
\ENDFOR
\STATE \textbf{return} $V$
\end{algorithmic}
\end{algorithm}

\subsection{BPE Practical Details}

\textbf{Most subword algorithms run inside space-separated tokens.}

\textbf{Common approach:}
\begin{enumerate}
    \item Add end-of-word symbol '\_\_' before spaces
    \item Separate into letters
    \item Run BPE (merges don't cross word boundaries)
\end{enumerate}

\textbf{Example:} "the cat" $\rightarrow$ "the\_\_ cat\_\_" $\rightarrow$ "t h e \_\_ c a t \_\_"

\subsection{BPE Example: Step by Step}

\textbf{Original corpus:}
\begin{verbatim}
low low low low low lowest lowest newer newer newer 
newer newer newer wider wider wider new new
\end{verbatim}

\textbf{After adding end-of-word tokens (\_\_), initial vocabulary:}
\begin{center}
\texttt{\_\_, d, e, i, l, n, o, r, s, t, w}
\end{center}

\textbf{Initial corpus state:}
\begin{verbatim}
Frequency  Corpus
5          l o w __
2          l o w e s t __
6          n e w e r __
3          w i d e r __
2          n e w __
\end{verbatim}

\textbf{Iteration 1: Merge} \texttt{e r} $\rightarrow$ \texttt{er}

Most frequent adjacent pair is 'e' 'r' (appears 9 times: 6 in "newer" + 3 in "wider")

\begin{verbatim}
Vocabulary: __, d, e, i, l, n, o, r, s, t, w, er
Corpus:
5          l o w __
2          l o w e s t __
6          n e w er __
3          w i d er __
2          n e w __
\end{verbatim}

\textbf{Iteration 2: Merge} \texttt{er \_\_} $\rightarrow$ \texttt{er\_\_}

Most frequent adjacent pair is 'er' '\_\_' (appears 9 times)

\begin{verbatim}
Vocabulary: __, d, e, i, l, n, o, r, s, t, w, er, er__
Corpus:
5          l o w __
2          l o w e s t __
6          n e w er__
3          w i d er__
2          n e w __
\end{verbatim}

\textbf{Iteration 3: Merge} \texttt{n e} $\rightarrow$ \texttt{ne}

Most frequent adjacent pair is 'n' 'e' (appears 8 times: 6 in "newer" + 2 in "new")

\begin{verbatim}
Vocabulary: __, d, e, i, l, n, o, r, s, t, w, er, er__, ne
Corpus:
5          l o w __
2          l o w e s t __
6          ne w er__
3          w i d er__
2          ne w __
\end{verbatim}

\textbf{Subsequent merges:}
\begin{itemize}
    \item \textbf{(ne, w)}: $\rightarrow$ \texttt{new} - Vocabulary adds: \texttt{new}
    \item \textbf{(l, o)}: $\rightarrow$ \texttt{lo} - Vocabulary adds: \texttt{lo}
    \item \textbf{(lo, w)}: $\rightarrow$ \texttt{low} - Vocabulary adds: \texttt{low}
    \item \textbf{(new, er\_\_)}: $\rightarrow$ \texttt{newer\_\_} - Vocabulary adds: \texttt{newer\_\_}
    \item \textbf{(low, \_\_)}: $\rightarrow$ \texttt{low\_\_} - Vocabulary adds: \texttt{low\_\_}
\end{itemize}

\textbf{Final vocabulary includes:}
\begin{center}
\texttt{\_\_, d, e, i, l, n, o, r, s, t, w, er, er\_\_, ne, new, lo, low, newer\_\_, low\_\_}
\end{center}

\textbf{Final corpus:}
\begin{verbatim}
5          low__
2          low e s t __
6          newer__
3          w i d er__
2          new __
\end{verbatim}

\textbf{Key observations:}
\begin{itemize}
    \item Frequent words become single tokens (\texttt{low\_\_}, \texttt{newer\_\_})
    \item Rare words remain as character sequences (\texttt{w i d er\_\_})
    \item Common morphemes are learned (\texttt{er}, \texttt{new})
    \item Vocabulary grows incrementally with each merge
\end{itemize}

\subsection{BPE Token Segmenter Algorithm}

Once we have learned the vocabulary, we use it to tokenize new test data.

\textbf{On test data, run each merge learned from training data:}
\begin{itemize}
    \item \textbf{Greedily}
    \item \textbf{In the order we learned them}
    \item \textbf{Test frequencies don't play a role}
\end{itemize}

\textbf{Process:} Apply merges sequentially - merge every \texttt{e r} to \texttt{er}, then merge \texttt{er \_\_} to \texttt{er\_\_}, etc.

\textbf{Examples:}

\begin{itemize}
    \item Test word "\texttt{n e w e r \_\_}" would be tokenized as a full word: \texttt{newer\_\_}
    \begin{itemize}
        \item Applies merges: \texttt{e r} $\rightarrow$ \texttt{er}, then \texttt{er \_\_} $\rightarrow$ \texttt{er\_\_}, then \texttt{n e} $\rightarrow$ \texttt{ne}, then \texttt{ne w} $\rightarrow$ \texttt{new}, finally \texttt{new er\_\_} $\rightarrow$ \texttt{newer\_\_}
    \end{itemize}
    
    \item Test word "\texttt{l o w e r \_\_}" would be two tokens: \texttt{low er\_\_}
    \begin{itemize}
        \item Applies merges: \texttt{e r} $\rightarrow$ \texttt{er}, then \texttt{er \_\_} $\rightarrow$ \texttt{er\_\_}, then \texttt{l o} $\rightarrow$ \texttt{lo}, then \texttt{lo w} $\rightarrow$ \texttt{low}
        \item Result: \texttt{low} and \texttt{er\_\_} remain as separate tokens (no merge learned for this combination)
    \end{itemize}
\end{itemize}

\textbf{Key point:} The segmenter applies merges deterministically in the learned order, regardless of frequencies in the test data.

\subsection{Properties of BPE Tokens}

BPE tokens have useful linguistic properties:

\textbf{Usually include:}
\begin{itemize}
    \item \textbf{Frequent words}: Common words become single tokens
    \item \textbf{Frequent subwords}: Often morphemes like \textit{-est} or \textit{-er}
\end{itemize}

\textbf{Morphemes:} A \textbf{morpheme} is the smallest meaning-bearing unit of a language.

\textbf{Example:} The word \textit{unlikeliest} has 3 morphemes:
\begin{itemize}
    \item \textit{un-}: prefix meaning "not"
    \item \textit{likely}: root word
    \item \textit{-est}: suffix meaning "most"
\end{itemize}

\textbf{Why this matters:}
\begin{itemize}
    \item BPE naturally discovers morphological structure
    \item Related words share subword components
    \item Example: \textit{play}, \textit{playing}, \textit{played} share the \textit{play} subword
    \item Enables better generalization across morphological variants
\end{itemize}

\subsection{Modern Usage}

\textbf{Subword tokenization is standard in modern NLP:}
\begin{itemize}
    \item \textbf{GPT}: BPE
    \item \textbf{BERT}: WordPiece  
    \item \textbf{T5}: Unigram
    \item Most transformer models use subword tokenization
\end{itemize}

\textbf{Why it works:} Balances vocabulary size and sequence length, handles morphology, no unknown words problem.

\section{Word Normalization}

After tokenization, we often need to normalize words to standard forms. Word normalization involves putting words/tokens in a standard, canonical form.

\textbf{Common examples:}
\begin{itemize}
    \item \textbf{U.S.A.} or \textbf{USA} - Standardizing abbreviations
    \item \textbf{uhhuh} or \textbf{uh-huh} - Normalizing disfluencies
    \item \textbf{Fed} or \textbf{fed} - Case normalization
    \item \textbf{am, is, be, are} - Lemmatization to base form
\end{itemize}

\subsection{Case Folding}

\textbf{Case folding} means reducing all letters to lower case.

\textbf{Applications like Information Retrieval (IR):}
\begin{itemize}
    \item Users tend to use lower case in queries
    \item Improves recall by matching regardless of case
\end{itemize}

\textbf{Possible exceptions - Upper case in mid-sentence may be meaningful:}
\begin{itemize}
    \item \textit{General Motors} - Proper noun (company name)
    \item \textit{Fed} vs. \textit{fed} - Federal Reserve vs. past tense of "feed"
    \item \textit{SAIL} vs. \textit{sail} - Acronym vs. common word
\end{itemize}

\textbf{For sentiment analysis, MT, Information Extraction:}
\begin{itemize}
    \item Case is helpful (\textit{US} versus \textit{us} is important)
    \item Trade-off between generalization and precision
\end{itemize}

\subsection{Lemmatization}

\textbf{Lemmatization} represents all words as their lemma - their shared root or dictionary headword form.

\textbf{The lemma} is the canonical form of a word, typically the form you would look up in a dictionary.

\textbf{Examples:}
\begin{itemize}
    \item \textit{am, are, is} $\rightarrow$ \textit{be}
    \item \textit{car, cars, car's, cars'} $\rightarrow$ \textit{car}
    \item Spanish: \textit{quiero} ('I want'), \textit{quieres} ('you want') $\rightarrow$ \textit{querer} 'want'
    \item \textit{He is reading detective stories} $\rightarrow$ \textit{He be read detective story}
\end{itemize}

\textbf{Benefits:} Reduces vocabulary size, groups related word forms together, useful for search and IR.

\subsection{Lemmatization by Morphological Parsing}

Lemmatization is typically done through \textbf{morphological parsing}.

\textbf{Morphemes} are the small meaningful units that make up words:
\begin{itemize}
    \item \textbf{Stems}: The core meaning-bearing units
    \begin{itemize}
        \item Example: \textit{cat} in "cats", \textit{walk} in "walking"
    \end{itemize}
    \item \textbf{Affixes}: Parts that adhere to stems, often with grammatical functions
    \begin{itemize}
        \item Prefixes: \textit{un-}, \textit{re-}, \textit{pre-}
        \item Suffixes: \textit{-s}, \textit{-ing}, \textit{-ed}, \textit{-er}
    \end{itemize}
\end{itemize}

\textbf{Morphological parsers} analyze words into their component morphemes:
\begin{itemize}
    \item Parse \textit{cats} into: \textit{cat} (stem) + \textit{s} (plural suffix)
    \item Parse Spanish \textit{amaren} ('if in the future they would love') into:
    \begin{itemize}
        \item Morpheme: \textit{amar} 'to love' (stem)
        \item Features: 3PL and future subjunctive
    \end{itemize}
\end{itemize}

\textbf{Challenges:} Irregular forms (\textit{went} $\rightarrow$ \textit{go}), ambiguity (\textit{saw}), language-specific rules.

\subsection{Stemming}

\textbf{Stemming} reduces terms to stems, chopping off affixes crudely.

\textbf{Difference from lemmatization:}
\begin{itemize}
    \item Stemming is simpler and faster
    \item May produce non-words
    \item Doesn't use morphological analysis
    \item Just applies rules to remove suffixes/prefixes
\end{itemize}

\textbf{Example comparison:}

\begin{table}[h]
\centering
\begin{tabular}{|l|l|}
\hline
\textbf{Original text} & \textbf{After stemming} \\
\hline
This was not the map we found & Thi wa not the map we found \\
in Billy Bones's chest, but an & in Billi Bone s chest but an \\
accurate copy, complete in all & accur copi complet in all \\
things-names and heights and & thing name and height and \\
soundings-with the single & sound with the singl except \\
exception of the red crosses & of the red cross and the \\
and the written notes. & written note \\
\hline
\end{tabular}
\end{table}

\textbf{Notice:} Words are reduced to stems that may not be valid words (\textit{Billi}, \textit{accur}, \textit{copi}).

\subsubsection{Porter Stemmer}

The most common stemming algorithm for English.

\textbf{Based on a series of rewrite rules run in series:}
\begin{itemize}
    \item A cascade, in which output of each pass fed to next pass
    \item Rules applied sequentially
\end{itemize}

\textbf{Some sample rules:}

\begin{itemize}
    \item \texttt{ATIONAL} $\rightarrow$ \texttt{ATE} (e.g., \textit{relational} $\rightarrow$ \textit{relate})
    \item \texttt{ING} $\rightarrow$ $\epsilon$ if stem contains vowel (e.g., \textit{motoring} $\rightarrow$ \textit{motor})
    \item \texttt{SSES} $\rightarrow$ \texttt{SS} (e.g., \textit{grasses} $\rightarrow$ \textit{grass})
\end{itemize}

\textbf{Characteristics:}
\begin{itemize}
    \item Fast and simple
    \item Language-specific (different rules for each language)
    \item May over-stem or under-stem
    \item Widely used despite imperfections
\end{itemize}

\subsection{Complex Morphology in Other Languages}

Dealing with complex morphology is necessary for many languages.

\textbf{Example: Turkish word}

\textit{Uygarlastiramadiklarimizdanmissiniz}

\textbf{Meaning:} '(behaving) as if you are among those whom we could not civilize'

\textbf{Morphological breakdown:}
\begin{itemize}
    \item \textit{Uygar} 'civilized' + \textit{las} 'become'
    \item + \textit{tir} 'cause' + \textit{ama} 'not able'
    \item + \textit{dik} 'past' + \textit{lar} 'plural'
    \item + \textit{imiz} 'p1pl' + \textit{dan} 'abl'
    \item + \textit{mis} 'past' + \textit{siniz} '2pl' + \textit{casina} 'as if'
\end{itemize}

\textbf{Key insight:} Languages like Turkish, Finnish, Hungarian have highly agglutinative morphology where many morphemes combine into single words. Simple stemming is insufficient - proper morphological analysis is essential.

\section{Sentence Segmentation}

The third step of text normalization (after tokenization and word normalization) is segmenting text into sentences.

\subsection{The Challenge}

\textbf{!, ? mostly unambiguous} - Usually mark sentence boundaries

\textbf{Period "." is very ambiguous:}
\begin{itemize}
    \item \textbf{Sentence boundary}: "I went home. She stayed."
    \item \textbf{Abbreviations}: Inc., Dr., Mr., U.S.A.
    \item \textbf{Numbers}: .02\%, 4.3
\end{itemize}

\subsection{Common Algorithm}

\textbf{Tokenize first:} Use rules or ML to classify a period as either:
\begin{enumerate}
    \item \textbf{(a) Part of the word} (abbreviation, number)
    \item \textbf{(b) A sentence boundary}
\end{enumerate}

\textbf{Helpful resource:} An abbreviation dictionary can help identify periods that are part of abbreviations.

\textbf{Sentence segmentation can then often be done by rules based on this tokenization.}

\subsection{Rule-Based Approach}

\textbf{Common rules:}
\begin{itemize}
    \item Period followed by uppercase letter usually indicates sentence boundary
    \item Period at end of line is usually sentence boundary
    \item Period followed by known abbreviation is not sentence boundary
    \item Use whitelist of common abbreviations (Dr., Inc., etc.)
\end{itemize}

\subsection{Machine Learning Approach}

\textbf{Features for classification:}
\begin{itemize}
    \item Word before period
    \item Word after period
    \item Capitalization of following word
    \item Length of words around period
    \item Presence in abbreviation dictionary
\end{itemize}

\textbf{Models:} Decision trees, logistic regression, or neural networks can be trained on annotated data.

\subsection{Challenges}

\begin{itemize}
    \item \textbf{Abbreviations at end of sentence}: "The company is called Yahoo Inc." (period serves both functions)
    \item \textbf{Quotations}: "He said, 'I'm leaving.' Then he left."
    \item \textbf{Multiple punctuation}: "Really?! No way!"
    \item \textbf{Ellipsis}: "And then..."
    \item \textbf{Domain-specific}: Scientific text, legal documents have different conventions
\end{itemize}



\newpage

\part{Language Modeling}

\section{Introduction to N-grams}

\subsection{Probabilistic Language Models}

A \textbf{Language Model} (LM) is a model that assigns probabilities to sequences of words.

\textbf{Goal:} Compute the probability of a sentence or sequence of words:
\[
P(W) = P(w_1, w_2, w_3, w_4, w_5 \ldots w_n)
\]

\textbf{Related task:} Probability of an upcoming word:
\[
P(w_5 \mid w_1, w_2, w_3, w_4)
\]

A model that computes either of these is called a \textbf{language model}.

\textbf{Terminology note:} Better term would be "the grammar", but \textbf{language model} or \textbf{LM} is standard.

\subsection{Why Probabilistic Language Models?}

Language models are fundamental for many NLP applications.

\subsubsection{Machine Translation}

A language model helps choose the most natural translation.

\textbf{Example:}
\begin{itemize}
    \item $P(\text{high winds tonite}) > P(\text{large winds tonite})$
\end{itemize}

"High winds" is a more common collocation than "large winds".

\subsubsection{Spell Correction}

\textbf{Example:} "The office is about fifteen minuets from my house"

\[
P(\text{about fifteen \textbf{minutes} from}) > P(\text{about fifteen \textbf{minuets} from})
\]

Both "minuets" and "minutes" are valid words, but the context makes "minutes" much more likely.

\subsubsection{Speech Recognition}

Many word sequences sound similar when spoken.

\textbf{Example:}
\[
P(\text{I saw a van}) \gg P(\text{eyes awe of an})
\]

The language model helps disambiguate between phonetically similar sequences.

\subsubsection{Other Applications}

\begin{itemize}
    \item \textbf{Summarization}
    \item \textbf{Question-answering}
    \item \textbf{Dialogue systems}
    \item \textbf{Text generation}
    \item And many more!
\end{itemize}

\subsection{How to Compute P(W)}

How do we compute the probability of a word sequence?

\textbf{Example:} How to compute this joint probability:
\[
P(\text{its, water, is, so, transparent, that})
\]

\subsubsection{Naive Approach: Counting}

We could count how many times this exact sequence appears in a corpus:

\[
P(W) = \frac{\text{Count}(w_1, w_2, \ldots, w_n)}{\text{Total number of word sequences of length } n}
\]

\textbf{Problems:}
\begin{itemize}
    \item \textbf{Data sparsity}: Most sequences never appear in any corpus
    \item \textbf{Combinatorial explosion}: Number of possible sequences grows exponentially
    \item \textbf{No generalization}: Can't handle novel sequences
\end{itemize}

\subsubsection{Chain Rule of Probability}

\textbf{Intuition:} Break down the joint probability into conditional probabilities.

The \textbf{Chain Rule} decomposes a joint probability:

\[
P(w_1, w_2, \ldots, w_n) = P(w_1) \cdot P(w_2 \mid w_1) \cdot P(w_3 \mid w_1, w_2) \cdots P(w_n \mid w_1, \ldots, w_{n-1})
\]

Or more compactly:
\[
P(W) = \prod_{i=1}^{n} P(w_i \mid w_1, \ldots, w_{i-1})
\]

\textbf{Example:} For "its water is so transparent that":
\begin{align*}
P(\text{its, water, is, so, transparent, that}) = \; & P(\text{its}) \\
& \times P(\text{water} \mid \text{its}) \\
& \times P(\text{is} \mid \text{its, water}) \\
& \times P(\text{so} \mid \text{its, water, is}) \\
& \times P(\text{transparent} \mid \text{its, water, is, so}) \\
& \times P(\text{that} \mid \text{its, water, is, so, transparent})
\end{align*}

\textbf{Why this helps:} Conditional probabilities like $P(\text{water} \mid \text{its})$ are easier to estimate from data.

\textbf{Remaining problem:} Long contexts are still rare. This is where \textbf{N-gram models} come in.


















\end{document}


